
在默认情况下,\texttt{std::cin(std::cout)} 是极为迟缓的读入(输出)方式,而 \texttt{scanf(printf)} 比 \texttt{std::cin(std::cout)} 快得多。

可是为什么会这样呢?有没有什么办法解决读入输出缓慢的问题呢?

\subsection{关闭同步 / 解除绑定}

\subsubsection{std::ios::sync\_with\_stdio(false)}

这个函数是一个 “是否兼容 stdio” 的开关,C++ 为了兼容 C,保证程序在使用了 \texttt{printf} 和 \texttt{std::cout} 的时候不发生混乱,将输出流绑到了一起。

这其实是 C++ 为了兼容而采取的保守措施。我们可以在 IO 之前将 stdio 解除绑定,这样做了之后要注意不要同时混用 \texttt{std::cout} 和 \texttt{printf} 之类

\subsubsection{tie}

tie 是将两个 stream 绑定的函数,空参数的话返回当前的输出流指针。

在默认的情况下 \texttt{std::cin} 绑定的是 \texttt{std::cout},每次执行 \texttt{<<} 操作符的时候都要调用 \texttt{flush()},这样会增加 IO 负担。可以通过 \texttt{std::cin.tie(0)}(0 表示 NULL)来解除 \texttt{std::cin} 与 \texttt{std::cout} 的绑定,进一步加快执行效率。

\subsubsection{代码实现}

\begin{cppcode}
std::ios::sync_with_stdio(false);
std::cin.tie(0);
// 如果编译开启了 C++11 或更高版本,建议使用 std::cin.tie(nullptr);
\end{cppcode}

\subsection{读入优化}

\texttt{scanf} 和 \texttt{printf} 依然有优化的空间,这就是本章所介绍的内容——读入和输出优化。

\begin{itemize}
\item 注意,读入和输出优化均针对整数,不支持其他类型的数据
\end{itemize}

\subsubsection{原理}

众所周知,\texttt{getchar} 是用来读入 char 类型,且速度很快,用 “读入字符——转换为整形” 来代替缓慢的读入

每个整数由两部分组成——符号和数字

整数的 '+' 通常是省略的,且不会对后面数字所代表的值产生影响,而 '-' 不可省略,因此要进行判定

10 进制整数中是不含空格或除 0\textasciitilde{}9 和正负号外的其他字符的,因此在读入不应存在于整数中的字符(通常为空格)时,就可以判定已经读入结束

\subsubsection{代码实现}

\begin{cppcode}
int read() {
  int x = 0, w = 1;
  char ch = 0;
  while (ch < '0' || ch > '9') {  // ch 不是数字时
    if (ch == '-') w = -1;        // 判断是否为负
    ch = getchar();               // 继续读入
  }
  while (ch >= '0' && ch <= '9') {  // ch 是数字时
    x = (x << 3) + (x << 1) + ch - '0';  // 将新读入的数字’加’在 x 的后面
    // x<<3==x*8  x<<1==x*2  所以 (x<<3)+(x<<1) 相当于 x*10
    // x 是 int 类型,char 类型的 ch 和 ’0’ 会被自动转为其 ASCII
    // 表中序号,相当于将 ch 转化为对应数字
    ch = getchar();  // 继续读入
  }
  return x * w;  // 数字 * 正负号 = 实际数值
}
\end{cppcode}

\begin{itemize}
\item 举例 
\end{itemize}

读入 num 可写为 \texttt{num=read();}

\subsection{输出优化}

\subsubsection{原理}

同样是众所周知,\texttt{putchar} 是输出单个字符

因此将数字的每一位转化为字符输出以加速

要注意的是,负号要单独判断输出,并且每次 \%(mod)取出的是数字末位,因此要倒序输出

\subsubsection{代码实现}

\begin{cppcode}
int write(int x) {
  if (x < 0) {  // 判负 + 输出负号 + 变原数为正数
    x = -x;
    putchar('-');
  }
  if (x > 9) write(x / 10);  // 递归,将除最后一位外的其他部分放到递归中输出
  putchar(x % 10 + '0');  // 已经输出(递归)完 x 末位前的所有数字,输出末位
}
\end{cppcode}

但是递归实现常数是较大的,我们可以写一个栈来实现这个过程

\begin{cppcode}
inline void write(int x) {
  static int sta[35];
  int top = 0;
  do {
    sta[top++] = x % 10, x /= 10;
  } while (x);
  while (top) putchar(sta[--top] + 48);  // 48 是 '0'
}
\end{cppcode}

\begin{itemize}
\item 举例
\end{itemize}

输出 num 可写为 \texttt{write(num);}

\subsection{更快的读入 / 输出优化}

通过 \texttt{fread} 或者 \texttt{mmap} 可以实现更快的读入。其本质为一次性读入一个巨大的缓存区,如此比一个一个字符读入要快的多 (\texttt{getchar},\texttt{putchar})。 因为硬盘的多次读写速度是要慢于内存的,先一次性读到内存里在读入要快的多。

更通用的是 \texttt{fread},因为 \texttt{mmap} 不能在 Windows 使用。

\texttt{fread} 类似于 \texttt{scanf("%s")},不过它更为快速,而且可以一次性读入若干个字符(包括空格换行等制表符),如果缓存区足够大,甚至可以一次性读入整个文件。

对于输出,我们还有对应的 \texttt{fwrite} 函数

\begin{cppcode}
std::size_t fread(void* buffer, std::size_t size, std::size_t count,
                  std::FILE* stream);
std::size_t fwrite(const void* buffer, std::size_t size, std::size_t count,
                   std::FILE* stream);
\end{cppcode}

使用示例:\texttt{fread(Buf, 1, MAXSIZE, stdin)},如此从 stdin 文件流中读入 MAXSIZE 个大小为 1 的字符到 Buf 中。

读入之后的使用就跟普通的读入优化相似了,只需要重定义一下 getchar。它原来是从文件中读入一个 char,现在变成从 Buf 中读入一个 char,也就是头指针向后移动一位。

\begin{cppcode}
char buf[1 << 20], *p1, *p2;
#define gc()                                                               \
  (p1 == p2 && (p2 = (p1 = buf) + fread(buf, 1, 1 << 20, stdin), p1 == p2) \
       ? EOF                                                               \
       : *p1++)
\end{cppcode}

\texttt{fwrite} 也是类似的,先放入一个 \texttt{OutBuf[MAXSIZE]} 中,最后通过 \texttt{fwrite} 一次性将 \texttt{OutBuf} 输出。

参考代码:

\begin{cppcode}
namespace IO {
const int MAXSIZE = 1 << 20;
char buf[MAXSIZE], *p1, *p2;
#define gc()                                                               \
  (p1 == p2 && (p2 = (p1 = buf) + fread(buf, 1, MAXSIZE, stdin), p1 == p2) \
       ? EOF                                                               \
       : *p1++)
inline int rd() {
  int x = 0, f = 1;
  char c = nc();
  while (!isdigit(c)) {
    if (c == '-') f = -1;
    c = nc();
  }
  while (isdigit(c)) x = (x << 1) + (x << 3) + (c ^ 48), c = nc();
  return x * f;
}
char pbuf[1 << 20], *pp = pbuf;
inline void push(const char &c) {
  if (pp - pbuf == 1 << 20) fwrite(pbuf, 1, 1 << 20, stdout), pp = pbuf;
  *pp++ = c;
}
inline void write(int x) {
  static int sta[35];
  int top = 0;
  do {
    sta[top++] = x % 10, x /= 10;
  } while (x);
  while (top) push(sta[--top] + '0');
}
}  // namespace IO
\end{cppcode}

\subsection{参考}

\href{http://www.hankcs.com/program/cpp/cin-tie-with-sync_with_stdio-acceleration-input-and-output.html}{}

\href{http://meme.biology.tohoku.ac.jp/students/iwasaki/cxx/speed.html}{}
