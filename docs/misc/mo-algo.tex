
\subsection{普通莫队算法}

(主要参考了 \href{https://blog.sengxian.com/algorithms/mo-s-algorithm}{}。)

\subsubsection{概述}

莫队算法是由莫涛提出的算法,可以解决一类离线区间询问问题,适用性极为广泛。同时将其加以扩展,便能轻松处理树上路径询问以及支持修改操作。

\subsubsection{形式}

假设 $n=m$,那么对于序列上的区间询问问题,如果从 $[l,r]$ 的答案能够 $O(1)$ 扩展到 $[l-1,r],[l+1,r],[l,r+1],[l,r-1]$(即与 $[l,r]$ 相邻的区间)的答案,那么可以在 $O(n\sqrt{n})$ 的复杂度内求出所有询问的答案。

\subsubsection{实现}

离线后排序,顺序处理每个询问,暴力从上一个区间的答案转移到下一个区间答案(一步一步移动即可)。

\subsubsection{排序方法}

对于区间 $[l,r]$, 以 $l$ 所在块的编号为第一关键字,$r$ 为第二关键字从小到大排序。

\subsubsection{模板}

\begin{cppcode}
inline void move(int pos, int sign) {
  // update nowAns
}

void solve() {
  BLOCK_SIZE = int(ceil(pow(n, 0.5)));
  sort(querys, querys + m);
  for (int i = 0; i < m; ++i) {
    const query &q = querys[i];
    while (l > q.l) move(--l, 1);
    while (r < q.r) move(r++, 1);
    while (l < q.l) move(l++, -1);
    while (r > q.r) move(--r, -1);
    ans[q.id] = nowAns;
  }
}
\end{cppcode}

\subsubsection{复杂度分析}

以下的情况在 $n$ 和 $m$ 同阶的前提下讨论。

首先是分块这一步,这一步的时间复杂度毫无疑问地是 $O(\sqrt{n}\cdot\sqrt{n}\log\sqrt{n}+n\log n)=O(n\log n)$;

接着就到了莫队算法的精髓了,下面我们用通俗易懂的初中方法来证明它的时间复杂度是 $O(n\sqrt{n})$;

证:令每一块中 $L$ 的最大值为 $\max_1,\max_2,\max_3, \cdots , \max_{\lceil\sqrt{n}\rceil}$。

由第一次排序可知,$\max_1 \le \max_2 \le \cdots \le \max_{\lceil\sqrt{n}\rceil}$。

显然,对于每一块暴力求出第一个询问的时间复杂度为 $O(n)$。

考虑最坏的情况,在每一块中,$R$ 的最大值均为 $n$,每次修改操作均要将 $L$ 由 $\max_{i - 1}$ 修改至 $\max_i$ 或由 $\max_i$ 修改至 $\max_{i - 1}$。

考虑 $R$:因为 $R$ 在块中已经排好序,所以在同一块修改完它的时间复杂度为 $O(n)$。对于所有块就是 $O(n\sqrt{n})$。

重点分析 $L$:因为每一次改变的时间复杂度都是 $O(\max_i-\max_{i-1})$ 的,所以在同一块中时间复杂度为 $O(\sqrt{n}\cdot(\max_i-\max_{i-1}))$。

将每一块 $L$ 的时间复杂度合在一起,可以得到:

对于 $L$ 的总时间复杂度为

$$
\begin{aligned}
& O(\sqrt{n}(\max{}_1-1)+\sqrt{n}(\max{}_2-\max{}_1)+\sqrt{n}(\max{}_3-\max{}_2)+\cdots+\sqrt{n}(\max{}_{\lceil\sqrt{n}\rceil}-\max{}_{\lceil\sqrt{n}\rceil-1))} \\
= & O(\sqrt{n}\cdot(\max{}_1-1+\max{}_2-\max{}_1+\max{}_3-\max{}_2+\cdots+\max{}_{\lceil\sqrt{n}\rceil-1}-\max{}_{\lceil\sqrt{n}\rceil-2}+\max{}_{\lceil\sqrt{n}\rceil}-\max{}_{\lceil\sqrt{n}\rceil-1)}) \\
= & O(\sqrt{n}\cdot(\max{}_{\lceil\sqrt{n}\rceil-1}))\\
\end{aligned}
$$

(裂项求和)

由题可知 $\max_{\lceil\sqrt{n}\rceil}$ 最大为 $n$,所以 $L$ 的总时间复杂度最坏情况下为 $O(n\sqrt{n})$。

综上所述,莫队算法的时间复杂度为 $O(n\sqrt{n})$;

但是对于 $m$ 的其他取值,如 $m<n$,分块方式需要改变才能变的更优。

怎么分块呢?

我们设块长度为 $S$,那么对于任意多个在同一块内的询问,挪动的距离就是 $n$,一共 $\displaystyle \frac{n}{S}$ 个块,移动的总次数就是 $\displaystyle \frac{n^2}{S}$,移动可能跨越块,所以还要加上一个 $mS$ 的复杂度,总复杂度为 $\displaystyle O\left(\frac{n^2}{S}+mS\right)$,我们要让这个值尽量小,那么就要将这两个项尽量相等,发现 $S$ 取 $\displaystyle \frac{n}{\sqrt{m}}$ 是最优的,此时复杂度为 $\displaystyle O\left(\frac{n^2}{\displaystyle \frac{n}{\sqrt{m}}}+m\left(\frac{n}{\sqrt{m}}\right)\right)=O(n\sqrt{m})$。

\subsubsection{例题 \& 代码}

\href{https://www.lydsy.com/JudgeOnline/problem.php?id=2038}{小 Z 的袜子}

思路:莫队算法模板题。

对于区间 $[l,r]$,以 $l$ 所在块的编号为第一关键字,$r$ 为第二关键字从小到大排序。

然后从序列的第一个询问开始计算答案,第一个询问通过直接暴力算出,复杂度为 $O(n)$,后面的询问在前一个询问的基础上得到答案。

具体做法:

对于区间 $[i,i]$ ,由于区间只有一个元素,我们很容易就能知道答案。

然后一步一步从当前区间(已知答案)向下一个区间靠近。

我们设 $col[i]$ 表示当前颜色 i 出现了多少次,$ans$ 当前共有多少种可行的配对方案(有多少种可以选到一双颜色相同的袜子),表示然后每次移动的时候更新答案——设当前颜色为 $k$,如果是增长区间就是 $ans$ 加上 $C_{col[k]+1}^2-C_{col[k]}^2$,如果是缩短就是 $ans$ 减去 $C_{col[k]}^2-C_{col[k]-1}^2$。这应该很好理解。

而这个询问的答案就是 $\displaystyle \frac{ans}{C_{r-l+1}^2}$。

这里有个优化:$\displaystyle C_a^2=\frac{a (a-1)}{2}$。

所以 $\displaystyle C_{a+1}^2-C_a^2=\frac{(a+1) a}{2}-\frac{a (a-1)}{2}=\frac{a}{2}\cdot (a+1-a+1)=\frac{a}{2}\cdot 2=a$。

所以 $C_{col[k]+1}^2-C_{col[k]}^2=col[k]$。

这样我们少算了很多东西呢!

至少我的代码在 BZOJ 上测快了一倍。

还有,算 $C_a^2$ 可以用位运算,\texttt{a/2} 可以写成 \texttt{a>>1}。

算法总复杂度:$O(n\sqrt{n} )$

下面的代码中 \texttt{mot} 表示答案的分母 (mother),\texttt{sub} 表示分子,\texttt{sqn} 表示块的大小:$\sqrt{n}$,\texttt{arr} 是输入的数组,\texttt{node} 是存储询问的结构体,\texttt{tab} 是询问序列(排序后的),\texttt{col} 同上所述。

 注意:下面代码中的移动区间的 4 个 for 循环的位置很关键,不能改变它们之间的位置关系,不然会 WA(因为有那个 \texttt{++l} 和 \texttt{--r})。

代码:

\begin{cppcode}
#include <bits/stdc++.h>
#define bi(a) ((a - 1) / sqn + 1)
using namespace std;
typedef long long LL;
template <typename tp>
void read(tp& dig) {
  char c = getchar();
  dig = 0;
  while (!isdigit(c)) c = getchar();
  while (isdigit(c)) dig = dig * 10 + c - '0', c = getchar();
}
struct node {
  LL l, r, i;
};
LL n, m, sqn, arr[50005], l, r, ans, col[50005], sub[50005], mot[50005];
vector<node> tab;
bool cmp(node a, node b) {
  if (bi(a.l) == bi(b.l)) return a.r < b.r;
  return a.l < b.l;
}
LL gcd(LL a, LL b) { return !b ? a : gcd(b, a % b); }
int main() {
  read(n), read(m), sqn = sqrt(n);
  for (LL i = 1; i <= n; i++) read(arr[i]);
  for (LL i = 1, a, b; i <= m; i++)
    read(a), read(b), tab.push_back((node){a, b, i});
  sort(tab.begin(), tab.end(), cmp), l = r = tab[0].l, col[arr[l]]++;
  for (LL i = 0, gcdnum; i < tab.size(); i++) {
    for (; l < tab[i].l; l++) col[arr[l]]--, ans -= col[arr[l]];
    for (--l; l >= tab[i].l; l--) ans += col[arr[l]], col[arr[l]]++;
    for (; r > tab[i].r; r--) col[arr[r]]--, ans -= col[arr[r]];
    for (++r; r <= tab[i].r; r++) ans += col[arr[r]], col[arr[r]]++;
    sub[tab[i].i] = ans, l = tab[i].l, r = tab[i].r;
    mot[tab[i].i] = ((r - l) * (r - l + 1)) >> 1;
  }
  for (LL i = 1, gcdn; i <= m; i++)
    gcdn = gcd(sub[i], mot[i]),
    printf("%lld/%lld\n", sub[i] / gcdn, mot[i] / gcdn);
  return 0;
}
\end{cppcode}

\subsection{带修改}

请确保您已经会普通莫队算法了。

如果您还不会,请先阅读上面的 “普通莫队算法”

\subsubsection{特点}

普通莫队是不能带修改的

我们可以强行让它可以修改,就像 DP 一样,可以强行加上一维 \textbf{ 时间维 }, 表示这次操作的时间。

时间维表示经历的修改次数。

即把询问 $[l,r]$ 变成 $[l,r,time]$

那么我们的坐标也可以在时间维上移动,即 $[l,r,time]$ 多了一维可以移动的方向,可以变成:

\begin{itemize}
\item $[l-1,r,time]$
\item $[l+1,r,time]$
\item $[l,r-1,time]$
\item $[l,r+1,time]$
\item $[l,r,time-1]$
\item $[l,r,time+1]$
\end{itemize}

这样的转移也是 $O(1)$ 的,但是我们排序又多了一个关键字,再搞搞就行了

可以用和普通莫队类似的方法排序转移,做到 $O(n^{\frac{5}{3}})$

这一次我们排序的方式是以 $n^{\frac{2}{3}}$ 为一块,分成了 $n^{\frac{1}{3}}$ 块,第一关键字是左端点所在块,第二关键字是右端点所在块,第三关键字是时间。

还是来证明一下时间复杂度(默认块大小为 $\sqrt{n}$):

\begin{itemize}
\item 左右端点所在块不变,时间在排序后单调向右移,这样的复杂度是 $O(n)$
\item 若左右端点所在块改变,时间一次最多会移动 n 个格子,时间复杂度 $O(n)$
\item 左端点所在块一共有 $n^{\frac{1}{3}}$ 中,右端点也是 $n^{\frac{1}{3}}$ 种,一共 ${n^{\frac{1}{3}}}\times{n^{\frac{1}{3}}}=n^{\frac{2}{3}}$ 种,每种乘上移动的复杂度 $O(n)$,总复杂度 $O(n^{\frac{5}{3}})$
\end{itemize}

\subsubsection{例题}

\href{https://www.lydsy.com/JudgeOnline/problem.php?id=2120}{数颜色 BZOJ - 2120}

题目大意:给你一个序列,M 个操作,有两种操作:

\begin{enumerate}
\item 修改序列上某一位的数字
\item 询问区间 $[l,r]$ 中数字的种类数(多个相同的数字只算一个)
\end{enumerate}

我们不难发现,如果不带操作 1(修改)的话,我们就能轻松用普通莫队解决。

但是题目还带单点修改,所以用 \textbf{ 带修改的莫队 }。

先考虑普通莫队的做法:

\begin{itemize}
\item 每次扩大区间时,每加入一个数字,则统计它已经出现的次数,如果加入前这种数字出现次数为 $0$,则说明这是一种新的数字,答案 $+1$。然后这种数字的出现次数 $+1$。
\item 每次减小区间时,每删除一个数字,则统计它删除后的出现次数,如果删除后这种数字出现次数为 $0$,则说明这种数字已经从当前的区间内删光了,也就是当前区间减少了一种颜色,答案 $-1$。然后这种数字的出现次数 $-1$。
\end{itemize}

现在再来考虑修改:

\begin{itemize}
\item 单点修改,把某一位的数字修改掉。假如我们是从一个经历修改次数为 $i$ 的询问转移到一个经历修改次数为 $j$ 的询问上,且 $i<j$ 的话,我们就需要把第 $i+1$ 个到第 $j$ 个修改强行加上。
\item 假如 $j<i$ 的话,则需要把第 $i$ 个到第 $j+1$ 个修改强行还原。
\end{itemize}

怎么强行加上一个修改呢?假设一个修改是修改第 $pos$ 个位置上的颜色,原本 $pos$ 上的颜色为 $a$,修改后颜色为 $b$,还假设当前莫队的区间扩展到了 $[l,r]$。

\begin{itemize}
\item 加上这个修改:我们首先判断 $pos$ 是否在区间 $[l,r]$ 内。如果是的话,我们等于是从区间中删掉颜色 $a$,加上颜色 $b$,并且当前颜色序列的第 $pos$ 项的颜色改成 $b$。如果不在区间 $[l,r]$ 内的话,我们就直接修改当前颜色序列的第 $pos$ 项为 $b$。
\item 还原这个修改:等于加上一个修改第 $pos$ 项、把颜色 $b$ 改成颜色 $a$ 的修改。
\end{itemize}

因此这道题就这样用带修改莫队轻松解决啦!

代码:

\begin{cppcode}
#include <bits/stdc++.h>
#define SZ (10005)
using namespace std;
template <typename _Tp>
inline void IN(_Tp& dig) {
  char c;
  dig = 0;
  while (c = getchar(), !isdigit(c))
    ;
  while (isdigit(c)) dig = dig * 10 + c - '0', c = getchar();
}
int n, m, sqn, c[SZ], ct[SZ], c1, c2, mem[SZ][3], ans, tot[1000005], nal[SZ];
struct query {
  int l, r, i, c;
  bool operator<(const query another) const {
    if (l / sqn == another.l / sqn) {
      if (r / sqn == another.r / sqn) return i < another.i;
      return r < another.r;
    }
    return l < another.l;
  }
} Q[SZ];
void add(int a) {
  if (!tot[a]) ans++;
  tot[a]++;
}
void del(int a) {
  tot[a]--;
  if (!tot[a]) ans--;
}
char opt[10];
int main() {
  IN(n), IN(m), sqn = pow(n, (double)2 / (double)3);
  for (int i = 1; i <= n; i++) IN(c[i]), ct[i] = c[i];
  for (int i = 1, a, b; i <= m; i++)
    if (scanf("%s", opt), IN(a), IN(b), opt[0] == 'Q')
      Q[c1].l = a, Q[c1].r = b, Q[c1].i = c1, Q[c1].c = c2, c1++;
    else
      mem[c2][0] = a, mem[c2][1] = ct[a], mem[c2][2] = ct[a] = b, c2++;
  sort(Q, Q + c1), add(c[1]);
  int l = 1, r = 1, lst = 0;
  for (int i = 0; i < c1; i++) {
    for (; lst < Q[i].c; lst++) {
      if (l <= mem[lst][0] && mem[lst][0] <= r)
        del(mem[lst][1]), add(mem[lst][2]);
      c[mem[lst][0]] = mem[lst][2];
    }
    for (; lst > Q[i].c; lst--) {
      if (l <= mem[lst - 1][0] && mem[lst - 1][0] <= r)
        del(mem[lst - 1][2]), add(mem[lst - 1][1]);
      c[mem[lst - 1][0]] = mem[lst - 1][1];
    }
    for (++r; r <= Q[i].r; r++) add(c[r]);
    for (--r; r > Q[i].r; r--) del(c[r]);
    for (--l; l >= Q[i].l; l--) add(c[l]);
    for (++l; l < Q[i].l; l++) del(c[l]);
    nal[Q[i].i] = ans;
  }
  for (int i = 0; i < c1; i++) printf("%d\n", nal[i]);
  return 0;
}
\end{cppcode}

\subsection{树上莫队}

莫队只能处理线性问题,我们要把树强行压成序列

我们可以将树的括号序跑下来,把括号序分块,在括号序上跑莫队

具体怎么做呢?

dfs 一棵树,然后如果 dfs 到 x 点,就 push\_back(x),dfs 完 x 点,就直接 push\_back(-x),然后我们在挪动指针的时候

\begin{itemize}
\item 新加入的值是 x  ---> add(x)
\item 新加入的值是 - x ---> del(x)
\item 新删除的值是 x  ---> del(x)
\item 新删除的值是 - x ---> add(x)
\end{itemize}

这样的话,我们就把一棵树处理成了序列。

例题是 \href{https://www.luogu.org/problemnew/show/P4074}{\textbackslash{}[WC2013\textbackslash{}] 糖果公园}, 这题是带修改树上莫队

题意是给你一棵树, 每个点有颜色, 每次询问

$\sum_{c}val_c\sum_{i=1}^{cnt_c}w_i$

$val$ 表示该颜色的价值

$cnt$ 表示颜色出现的次数

$w$ 表示该颜色出现 $i$ 次后的价值

先把树变成序列,然后每次添加 / 删除一个点,这个点的对答案的的贡献是可以在 $O(1)$ 时间内获得的,即 $val_c\times w_{cnt_{c+1}}$

发现因为他会把起点的子树也扫了一遍,产生多余的贡献,怎么办呢?

因为扫的过程中起点的子树里的点肯定会被扫两次,但贡献为 0

所以可以开一个 $vis$ 数组,每次扫到点 x,就把 $vis_x$ 异或上 1

如果 $vis_x=0$,那这个点的贡献就可以不计

所以可以用树上莫队来求

修改的话,加上一维时间维即可, 变成带修改树上莫队

然后因为所包含的区间内可能没有 LCA,对于没有的情况要将多余的贡献删除,然后就完事了

code:

\begin{cppcode}
#include <algorithm>
#include <cmath>
#include <cstdio>
#include <iostream>

#define DEBUG printf("line:%d func:%s\n", __LINE__, __FUNCTION__);

using namespace std;

const int maxn = 200010;

int f[maxn], g[maxn], id[maxn], head[maxn], cnt, last[maxn], dep[maxn],
    fa[maxn][22], v[maxn], w[maxn];
int block, index, n, m, q;
int pos[maxn], col[maxn], app[maxn];
bool vis[maxn];
long long ans[maxn], cur;

struct edge {
  int to, nxt;
} e[maxn];
int cnt1 = 0, cnt2 = 0;  // 时间戳

struct query {
  int l, r, t, id;
  bool operator<(const query &b) const {
    return (pos[l] < pos[b.l]) || (pos[l] == pos[b.l] && pos[r] < pos[b.r]) ||
           (pos[l] == pos[b.l] && pos[r] == pos[b.r] && t < b.t);
  }
} a[maxn], b[maxn];

inline void addedge(int x, int y) {
  e[++cnt] = (edge){y, head[x]};
  head[x] = cnt;
}

void dfs(int x) {
  id[f[x] = ++index] = x;
  for (int i = head[x]; i; i = e[i].nxt) {
    if (e[i].to != fa[x][0]) {
      fa[e[i].to][0] = x;
      dep[e[i].to] = dep[x] + 1;
      dfs(e[i].to);
    }
  }
  id[g[x] = ++index] = x;  // 括号序
}

inline void swap(int &x, int &y) {
  int t;
  t = x;
  x = y;
  y = t;
}

inline int lca(int x, int y) {
  if (dep[x] < dep[y]) swap(x, y);
  if (dep[x] != dep[y]) {
    int dis = dep[x] - dep[y];
    for (int i = 20; i >= 0; i--)
      if (dis >= (1 << i)) dis -= 1 << i, x = fa[x][i];
  }  // 爬到同一高度
  if (x == y) return x;
  for (int i = 20; i >= 0; i--) {
    if (fa[x][i] != fa[y][i]) x = fa[x][i], y = fa[y][i];
  }
  return fa[x][0];
}

inline void add(int x) {
  if (vis[x])
    cur -= (long long)v[col[x]] * w[app[col[x]]--];
  else
    cur += (long long)v[col[x]] * w[++app[col[x]]];
  vis[x] ^= 1;
}

inline void modify(int x, int t) {
  if (vis[x]) {
    add(x);
    col[x] = t;
    add(x);
  } else
    col[x] = t;
}  // 在时间维上移动

int main() {
  scanf("%d%d%d", &n, &m, &q);
  for (int i = 1; i <= m; i++) scanf("%d", &v[i]);
  for (int i = 1; i <= n; i++) scanf("%d", &w[i]);
  for (int i = 1; i < n; i++) {
    int x, y;
    scanf("%d%d", &x, &y);
    addedge(x, y);
    addedge(y, x);
  }
  for (int i = 1; i <= n; i++) {
    scanf("%d", &last[i]);
    col[i] = last[i];
  }
  dfs(1);
  for (int j = 1; j <= 20; j++)
    for (int i = 1; i <= n; i++)
      fa[i][j] = fa[fa[i][j - 1]][j - 1];  // 预处理祖先
  int block = pow(index, 2.0 / 3);
  for (int i = 1; i <= index; i++) {
    pos[i] = (i - 1) / block;
  }
  while (q--) {
    int opt, x, y;
    scanf("%d%d%d", &opt, &x, &y);
    if (opt == 0) {
      b[++cnt2].l = x;
      b[cnt2].r = last[x];
      last[x] = b[cnt2].t = y;
    } else {
      if (f[x] > f[y]) swap(x, y);
      a[++cnt1] = (query){lca(x, y) == x ? f[x] : g[x], f[y], cnt2, cnt1};
    }
  }
  sort(a + 1, a + cnt1 + 1);
  int L, R, T;  // 指针坐标
  L = R = 0;
  T = 1;
  for (int i = 1; i <= cnt1; i++) {
    while (T <= a[i].t) {
      modify(b[T].l, b[T].t);
      T++;
    }
    while (T > a[i].t) {
      modify(b[T].l, b[T].r);
      T--;
    }
    while (L > a[i].l) {
      L--;
      add(id[L]);
    }
    while (L < a[i].l) {
      add(id[L]);
      L++;
    }
    while (R > a[i].r) {
      add(id[R]);
      R--;
    }
    while (R < a[i].r) {
      R++;
      add(id[R]);
    }
    int x = id[L], y = id[R];
    int llca = lca(x, y);
    if (x != llca && y != llca) {
      add(llca);
      ans[a[i].id] = cur;
      add(llca);
    } else
      ans[a[i].id] = cur;
  }
  for (int i = 1; i <= cnt1; i++) {
    printf("%lld\n", ans[i]);
  }
  return 0;
}
\end{cppcode}
