
\subsection{什么是裴蜀定理?}

裴蜀定理,又称贝祖定理。是代数几何中一个定理。

其内容是:

设 $a,b$ 是不全为零的整数, 则存在整数 $x,y$, 使得 $ax+by=\gcd(a,b)$.

\subsection{证明}

\begin{enumerate}
\item 若任何一个等于 $0$, 则 $\gcd(a,b)=a$. 这时定理显然成立.
\item 若 $a,b$ 不等于 $0$.
由于 $\gcd(a,b)=\gcd(a,-b)$,
不妨设 $a,b$ 都大于 $0$, $a\geq b,\gcd(a,b)=d$.
对 $ax+by=d$, 两边同时除以$d$, 可得 $a_1x+b_1y=1$, 其中 $(a_1,b_1)=1$.
转证 $a_1x+b_1y=1$. 由带余除法:
$$
\begin{align*}a_1 &= q_1b+r_1 &(0\leq r_1<b_1) \\ b_1 &= q_2r_1+r_2 &(0\leq r_2<r_1) \\ r_1 &= q_3r_2+r_3 &(0\leq r_3<r_2) \\ &\cdots \\ r_{n-3} &= q_{n-1}r_{n-2}+r_{n-1} \\ r_{n-2} &= q_nr_{n-1}+r_n \\ r_{n-1} &= q_{n+1}r_n\end{align*}
$$
于是, 有
$$
\gcd(a_1,b_1)=\gcd(b_1,r_1)=\gcd(r_1,r_2)=\cdots=(r_{n-1},r_n)=1
$$
故
$$
r_{n-2}=x_nr_{n-1}+1
$$
即
$$
1=r_{n-2}-x_nr_{n-1}
$$
由倒数第三个式子 $r_{n-1}=r_{n-3}-x_{n-1}r_{n-2}$ 代入上式, 得
$$
1=(1+x_nx_{n-1})r_{n-2}-x_nr_{n-3}
$$
然后用同样的办法用它上面的等式逐个地消去 $r_{n-2},\cdots,r_1$,
可证得 $1=a_1x+b_1y$.
\end{enumerate}

\subsection{应用}

\begin{NOTE}{Codeforces Round #290 (Div. 2) D. Fox And Jumping}{}
给出 $n$ 张卡片,分别有 $l_i$ 和 $c_i$。在一条无限长的纸带上,你可以选择花 $c_i$ 的钱来购买卡片 $i$,从此以后可以向左或向右跳 $l_i$ 个单位。问你至少花多少元钱才能够跳到纸带上全部位置。若不行,输出 $-1$。
\end{NOTE}


分析该问题,先考虑两个数的情况,发现想要跳到每一个格子上,必须使得这些数通过数次相加或相加得出的绝对值为 $1$,进而想到了裴蜀定理。

可以推出:如果 $a$ 与 $b$ 互质,那么一定存在两个整数 $x$ 与 $y$,使得 $ax+by=1$.

由此得出了若选择的卡牌的数通过数次相加或相减得出的绝对值为 $1$ ,那么这些数一定互质,此时可以考虑动态规划求解。

不过可以转移思想,因为这些数互质,即为 $0$ 号节点开始,每走一步求 $\gcd$(节点号, 下一个节点),同时记录代价,就成为了从 $0$ 通过不断 $\gcd$ 最后变为 $1$ 的最小代价。

由于:互质即为最大公因数为 $1$,$\gcd(0,x)=x$ 这两个定理,可以证明该算法的正确。选择优先队列优化 Dijkstra 求解。

不过还有个问题,即为需要记录是否已经买过一个卡片,开数组标记由于数据范围达到$10^9$会超出内存限制,可以想到使用 \texttt{unordered_map} (比普通的 \texttt{map} 更快地访问各个元素,迭代效率较低,详见 \href{/ds/stl/map/}{STL-map})
