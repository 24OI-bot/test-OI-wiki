
(本文转载自 \href{https://zhuanlan.zhihu.com/c_1005817911142838272}{桃酱的算法笔记},原文戳 \href{https://zhuanlan.zhihu.com/p/41867199}{链接},已获得作者授权)

\subsection{简介}

\begin{QUOTE}{}{}
沃尔什转换(Walsh Transform)是在频谱分析上作为离散傅立叶变换的替代方案的一种方法。 —— \href{https://zh.wikipedia.org/zh-cn/%E6%B2%83%E7%88%BE%E4%BB%80%E8%BD%89%E6%8F%9B}{维基百科}
\end{QUOTE}

其实这个变换在信号处理中应用很广泛,fft 是 double 类型的,但是 walsh 把信号在不同震荡频率方波下拆解,因此所有的系数都是绝对值大小相同的整数,这使得不需要作浮点数的乘法运算,提高了运算速度。

所以,FWT 和 FFT 的核心思想应该是相同的。都是对数组的变换。我们设数组 $A$ 经过快速沃尔什变换之后记作 $FWT[A]$.

那么 FWT 核心思想就是:

我们需要一个新序列 $C$,由序列 $A$ 和序列 $B$ 经过某运算规则得到,即 $C = A \cdot B$

我们先正向得到 $FWT[A], FWT[B]$

然后根据 $FWT[C]=FWT[A] \cdot FWT[B]$

在 $O(n)$ 求出 $FWT[C]$

然后逆向想运算得到原序列 $C$。时间复杂度为 $O(nlogn)$

在算法竞赛中,FWT 是用于解决对下标进行位运算卷积问题的方法。

公式: $C[i] = \sum_{i=j \bigoplus k}A[j] * B[k]$

(其中 $\bigoplus$ 是二元位运算中的某一种,$*$ 是普通乘法)

\subsection{FWT 的运算}

\subsubsection{FWT 之与($\And$)运算和或($|$)运算}

与运算和或运算的本质是差不多的,所以这里讲一下或运算,与运算也是可以自己根据公式 yy 出来的。

\subsubsection{或运算 $A_i$}

如果有 $k=i|j$,那么 $i$ 的二进制位为 $1$ 的位置和 $j$ 的二进制位为 $1$ 的位置肯定是 $k$ 的二进制位为 $1$ 的位置的子集。

现在要得到 $FWT[C] = FWT[A] * FWT[B]$,我们就要构造这个 fwt 的规则。

我们按照定义,显然可以构造 $FWT[A] = A' = \sum_{i=i|j}A[j]$,来表示 $j$ 满足二进制中 $1$ 为 $i$ 的子集。

那么显然会有 $C[i] = \sum_{i=j|k}A[j]*B[k] \Rightarrow FWT[C] = FWT[A] * FWT[B]$

那么我们接下来看 $FWT[A]$ 怎么求。

首先肯定不能枚举了,复杂度为 $O(n^2)$。既然不能整体枚举,我们就考虑分治。

我们把整个区间二分,其实二分区间之后,下标写成二进制形式是有规律可循的。

我们令 $A_0$ 表示 $A$ 的前一半,$A_1$ 表示区间的后一半,那么 $A_0$ 就是 A 下标最大值的最高位为 $0$,他的子集就是他本身的子集(因为最高位为 $0$ 了),但是 $A_1$ 的最高位是 $1$,他满足条件的子集不仅仅是他本身,还包最高位为 $0$ 的子集,即

$$
FWT[A] = merge(FWT[A_0], FWT[A_0] + FWT[A_1])
$$

其中 merge 表示像字符串拼接一样把它们拼起来, $+$ 就是普通加法,表示对应二进制位相加。

这样我们就通过二分能在 $O(logn)$ 完成拼接,每次拼接的时候要完成一次运算,也就是说在 $O(nlogn)$ 的时间复杂度得到了 $FWT[A]$。

接下来就是反演了,其实反演是很简单的,既然知道了 $A_0$ 的本身的子集是他自己 ($A_0 = FAT[A_0]$),$A_1$ 的子集是 $FAT[A_0] + FAT[A_1](A_1'= A_0' + A_1'$), 那就很简单的得出反演的递推式了:

$$
UFWT[A'] = merge(UFWT[A_0'], UFWT[A_1'] - UFWT[A_0'])
$$

\subsubsection{与运算}

与运算类比或运算可以得到类似结论

$$
FWT[A] = merge(FWT[A_0] + FWT[A_1], FWT[A_1])
$$

$$
UFWT[A'] = merge(UFWT[A_0'] - UFWT[A_1'], UFWT[A_1'])
$$

\subsubsection{异或运算}

最常考的异或运算。

异或的卷积是基于如下原理:

若我们令 $i\And j$ 中 $1$ 数量的奇偶性为 $i$ 与 $j$ 的奇偶性,那么 $i$ 与 $k$ 的奇偶性异或 $j$ 和 $k$ 的奇偶性等于 $i \operatorname{xor} j$ 和 $k$ 的奇偶性。

对于 $FWT[A]$ 的运算其实也很好得到。

公式如下:

$A[i] = \sum_{C_1}A[j] - \sum_{C_2}A[j]$ ($C_1$ 表示 $i \And j$ 奇偶性为 $0$,$C_2$ 表示 $i \And j$ 的奇偶性为 $1$)

结论:

$$
FWT[A] = merge(FWT[A_0] + FWT[A_1], FWT[A_0] - FWT[A_1])
$$

$$
UFWT[A'] - merge(\frac{FWT[A_0'] + FWT[A_1']}{2}, \frac{FWT[A_0'] - FWT[A_1']}{2})
$$

\subsubsection{同或运算}

类比异或运算给出公式:

$A[i] = \sum_{C_1}A[j] - \sum_{C_2}A[j]$ ($C_1$ 表示 $i|j$ 奇偶性为 $0$,$C_2$ 表示 $i|j$ 的奇偶性为 $1$)

$$
FWT[A] = merge(FWT[A_1] - FWT[A_0], FWT[A_1] + FWT[A_0])
$$

$$
UFWT[A'] = merge(\frac{FWT[A_1'] - FWT[A_0']}{2}, \frac{FWT[A_1'] + FWT[A_0']}{2})
$$
