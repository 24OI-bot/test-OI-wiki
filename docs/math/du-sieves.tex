
在数论题目中,常常需要根据一些 \textbf{ 积性函数 } 的性质,求出一些式子的值。

\textbf{ 积性函数 } : 对于所有互质的 $a$ 和 $b$ ,总有 $f(ab)=f(a)f(b)$ ,则称 $f(x)$ 为积性函数。

常见的积性函数有:

$d(x)=\sum_{i|n} 1$

$\sigma(x)=\sum_{i|n} i$

$\varphi(x)=\sum_{i=1}^x 1[gcd(x,i)=1]$

$\mu(x)=\begin{cases}1&\text{n=1}\\(-1)^k& \ \prod_{i=1}^k q_i=1\\0 &\ \max\{q_i\}>1\end{cases}$

积性函数有如下性质:

若 $f(x)$ , $g(x)$ 为积性函数,则

$h(x)=f(x^p)$ 

$h(x)=f^p(x)$

$h(x)=f(x)g(x)$

$h(x)=\sum_{d|x} f(d)g(\frac x d)$

中的 $h(x)$ 也为积性函数。

在莫比乌斯反演的题目中,往往要求出一些数论函数的前缀和,利用 \textbf{ 杜教筛 } 可以快速求出这些前缀和。 

\begin{NOTE}{ 例题 [P4213 【模板】杜教筛( Sum ) ](https://www.luogu.org/problemnew/show/P4213)}{}

\end{NOTE}


题目大意: 求 $S_1(n)= \sum_{i=1}^n \mu(i)$ 和 $S_2(n)= \sum_{i=1}^n \varphi(i)$  的值, $n\le 2^{31} -1$ 。

由 \textbf{ 狄利克雷卷积 } ,我们知道:

$\because \epsilon =\mu * 1$ ( $\epsilon(n)=~[n=1]$ ) 

$\therefore \epsilon (n)=\sum_{d|n} \mu(d)$ 

$S_1(n)=\sum_{i=1}^n \epsilon (i)-\sum_{i=2}^n S_1(\lfloor \frac n i \rfloor)$

$= 1-\sum_{i=2}^n S_1(\lfloor \frac n i \rfloor)$

观察到 $\lfloor \frac n i \rfloor$ 最多只有 $O(\sqrt n)$ 种取值,我们就可以应用 \textbf{ 整除分块 } (或称数论分块)来计算每一项的值了。 

直接计算的时间复杂度为 $O(n^{\frac 3 4})$ 。考虑先线性筛预处理出前 $n^{\frac 2 3}$ 项,剩余部分的时间复杂度为

$O(\int_{0}^{n^{\frac 1 3}} \sqrt{\frac{n}{x}} ~ dx)=O(n^{\frac 2 3})$ 

对于较大的值,需要用 \texttt{map} 存下其对应的值,方便以后使用时直接使用之前计算的结果。

当然也可以用杜教筛求出 $\varphi (x)$ 的前缀和,但是更好的方法是应用莫比乌斯反演:

$\sum_{i=1}^n \sum_{j=1}^n 1[gcd(i,j)=1]=\sum_{i=1}^n \sum_{j=1}^n \sum_{d|i,d|j} \mu(d)$

$=\sum_{d=1}^n \mu(d) {\lfloor \frac n d \rfloor}^2$

由于题目所求的是 $\sum_{i=1}^n \sum_{j=1}^i 1[gcd(i,j)=1]$ ,所以我们排除掉 $i=1,j=1$ 的情况,并将结果除以 $2$ 即可。 

观察到,只需求出莫比乌斯函数的前缀和,就可以快速计算出欧拉函数的前缀和了。时间复杂度 $O(n^{\frac 2 3})$ 。 

给出一种代码实现:

\begin{cppcode}
#include <algorithm>
#include <cstdio>
#include <cstring>
#include <map>
using namespace std;
const int maxn = 2000010;
typedef long long ll;
ll T, n, pri[maxn], cur, mu[maxn], sum_mu[maxn];
bool vis[maxn];
map<ll, ll> mp_mu;
ll S_mu(ll x) {
  if (x < maxn) return sum_mu[x];
  if (mp_mu[x]) return mp_mu[x];
  ll ret = 1ll;
  for (ll i = 2, j; i <= x; i = j + 1) {
    j = x / (x / i);
    ret -= S_mu(x / i) * (j - i + 1);
  }
  return mp_mu[x] = ret;
}
ll S_phi(ll x) {
  ll ret = 0ll;
  for (ll i = 1, j; i <= x; i = j + 1) {
    j = x / (x / i);
    ret += (S_mu(j) - S_mu(i - 1)) * (x / i) * (x / i);
  }
  return ((ret - 1) >> 1) + 1;
}
int main() {
  scanf("%lld", &T);
  mu[1] = 1;
  for (int i = 2; i < maxn; i++) {
    if (!vis[i]) {
      pri[++cur] = i;
      mu[i] = -1;
    }
    for (int j = 1; j <= cur && i * pri[j] < maxn; j++) {
      vis[i * pri[j]] = true;
      if (i % pri[j])
        mu[i * pri[j]] = -mu[i];
      else {
        mu[i * pri[j]] = 0;
        break;
      }
    }
  }
  for (int i = 1; i < maxn; i++) sum_mu[i] = sum_mu[i - 1] + mu[i];
  while (T--) {
    scanf("%lld", &n);
    printf("%lld %lld\n", S_phi(n), S_mu(n));
  }
  return 0;
}
\end{cppcode}
