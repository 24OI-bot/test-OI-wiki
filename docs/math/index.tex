
\subsection{数学部分简介}

在 OI/ACM 的各种比赛中,常常会有数学题的出现。

这些数学题以数论、排列组合、概率期望、多项式为代表,可以出现在几乎任何类别的题目中

举几个栗子 :

\begin{enumerate}
\item 多项式可以优化卷积形式的背包,可以做一些字符串题
\item 很多 DP 类型的题都可以结合排列组合 / 概率期望。
\end{enumerate}

\hr

\subsubsection{以下是你可以在本部分找到的知识 (部分未完成,待补充)}

\begin{enumerate}
\item 进制相关
\item 位运算 —— 二进制下的按位运算
\item 高精度 —— 当语言变量类型不足以表达需要表达的数时的处理方法
\item 整除性质 (数论)
\item 同余相关 (数论)
\item 高斯消元 (矩阵 / 概率期望)
\item 数论反演
\item 杜教筛 / 洲阁筛
\item 多项式 (FFT, NTT, FWT, 拉格朗日差值)
\item 排列组合 (Lucas, Catalan)
\item 概率与期望
\item 置换
\item 线性规划
\item 线性基
\end{enumerate}

\hr

OI 中的数学以高中,大学的数学为基础,考察选手对数学知识的掌握,利用计算机的计算能力来解决问题。

\subsubsection{NOIP 中有可能会考察的知识点}

然而 NOIP 可能考察更多的知识点,这里只是利用之前的题总结出来的,考过或者考的概率比较大的知识点。

NOIP 对数学的考察还处在一个比较简单的范围。

\begin{enumerate}
\item 进制相关 —— 通常是利用进制优化一些问题
\item 位运算 —— 状压常用
\item 高精度 —— 不包括需要利用多项式的高精度
\item 整除性质 —— $\gcd$,欧拉函数,费马小定理
\item 同余相关 —— $exgcd$,逆元,中国剩余定理
\item 概率期望 —— 概率 DP,以及有可能用到高斯消元解决的概率 DP
\item 排列组合 —— 杨辉三角,二项式定理,卢卡斯定理,卡特兰数
\end{enumerate}
