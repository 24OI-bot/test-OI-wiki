
我们说,如果存在一个整数 $k$,使得 $a = kd$,则称 $d$ 整除 $a$,记做 $d | a$,称 $a$ 是 $d$ 的倍数,如果 $d > 0$,称 $d$ 是 $a$ 的约数。特别地,任何整数都整除 $0$。

显然大于 $1$ 的正整数 $a$ 可以被 $1$ 和 $a$ 整除,如果除此之外 $a$ 没有其他的约数,则称 $a$ 是素数,又称质数。任何一个大于 $1$ 的整数如果不是素数,也就是有其他约数,就称为是合数。$1$ 既不是合数也不是素数。

素数计数函数:小于或等于 $x$ 的素数的个数,用 $\pi(x)$ 表示。随着 $x$ 的增大,有这样的近似结果:$\pi(x) \sim \frac{x}{\ln(x)}$

\subsection{素数判定}

我们自然地会想到,如何用计算机来判断一个数是不是素数呢?

\subsubsection{暴力做法}

自然可以枚举从小到大的每个数看是否能整除

\begin{cppcode}
bool isPrime(a) {
  for (int i = 2; i < a; ++i)
    if (a % i == 0) return 0;
  return 1;
}
\end{cppcode}

这样做是十分稳妥了,但是真的有必要每个数都去判断吗?

很容易发现这样一个事实:如果 $x$ 是 $a$ 的约数,那么 $\frac{a}{x}$ 也是 $a$ 的约数。

这个结论告诉我们,对于每一对 $(x, \frac{a}{x} )$,只需要检验其中的一个就好了。为了方便起见,我们之考察每一对里面小的那个数。不难发现,所有这些较小数就是 $[1, \sqrt{a}]$ 这个区间里的数。

由于 $1$ 肯定是约数,所以不检验它。

\begin{cppcode}
bool isPrime(a) {
  for (int i = 2; i * i <= a; ++i)
    if (a % i) return 0;
  return 1;
}
\end{cppcode}

\subsubsection{Miller-Rabin 素性测试}

Miller-Rabin 素性测试(Miller–Rabin primality test)是进阶的素数判定方法,具有比暴力做法更好的时间复杂度。但是代码复杂度较高,在比赛中使用较少。

\paragraph{Fermat 素性测试}

我们可以根据 \href{/math/fermat/#_1}{费马小定理} 得出一种检验素数的思路:

它的基本思想是不断地选取在 $[2, n-1]$ 中的基 $a$,并检验是否每次都有 $a^{n-1} \equiv 1 \pmod n$

\begin{cppcode}
bool millerRabin(int n) {
  for (int i = 1; i <= s; ++i) {
    int a = rand() % (n - 2) + 2;
    if (quickPow(a, n - 1, n) != 1) return 0;
  }
  return 1;
}
\end{cppcode}

很遗憾,费马小定理的逆定理并不成立,换言之,满足了 $a^{n-1} \equiv 1 \pmod n$ ,$n$ 也不一定是素数。

\paragraph{卡迈克尔数}

上面的做法中随机地选择 $a$,很大程度地降低了犯错的概率。但是仍有一类数,上面的做法并不能准确地判断。

对于合数 $n$,如果对于所有正整数 $a$,$a$ 和 $n$ 互素,都有同余式 $a^{n-1} \equiv 1 \pmod n$ 成立,则合数 $n$ 为卡迈克尔数(Carmichael Number),又称为费马伪素数。

比如,$341 = 11 \times 31$ 就是一个卡迈克尔数。

而且我们知道,若 $n$ 为卡迈克尔数,则 $m=2^{n}-1$ 也是一个卡迈克尔数,从而卡迈克尔数的个数是无穷的。

\paragraph{二次探测定理}

如果 $p$ 是奇素数,则 $x^2 \equiv 1 \bmod p$ 的解为 $x = 1$ 或者 $x = p - 1 (\bmod p)$;

\subsubsection{实现}

根据卡迈克尔数的性质,可知其一定不是 $p^e$。

不妨将费马小定理和二次探测定理结合起来使用:

将 $n−1$ 分解为 $n−1=u \times 2^t$,不断地对 $u$ 进行平方操作,若发现非平凡平方根时即可判断出其不是素数。

比较正确的 Miller Rabin:(来自 fjzzq2002)

\begin{cppcode}
bool millerRabbin(int n) {
  int a = n - 1, b = 0;
  while (a % 2 == 0) a /= 2, ++b;
  for (int i = 1, j; i <= s; ++i) {
    int x = rand() % (n - 2) + 2, v = quickPow(x, a, n);
    if (v == 1 || v == n - 1) continue;
    for (j = 0; j < b; ++j) {
      v = (long long)v * v % n;
      if (v == n - 1) break;
    }
    if (j >= b) return 0;
  }
  return 1;
}
\end{cppcode}

\subsubsection{参考}

\href{http://www.matrix67.com/blog/archives/234}{}

\href{https://blog.bill.moe/miller-rabin-notes/}{}

\subsection{反素数}

\subsubsection{定义}

如果某个正整数 $n$ 满足如下条件,则称为是反素数:

  任何小于 $n$ 的正数的约数个数都小于 $n$ 的约数个数

注:注意区分 \href{https://en.wikipedia.org/wiki/Emirp}{emirp},它是用来表示从后向前写读是素数的数。

\subsubsection{简介}

(本段转载自 \href{https://zhuanlan.zhihu.com/c_1005817911142838272}{桃酱的算法笔记},原文戳 \href{https://zhuanlan.zhihu.com/p/41759808}{链接},已获得作者授权)

其实顾名思义,素数就是因子只有两个的数,那么反素数,就是因子最多的数(并且因子个数相同的时候值最小),所以反素数是相对于一个集合来说的。

我所理解的反素数定义就是,在一个集合中,因素最多并且值最小的数,就是反素数。

那么,如何来求解反素数呢?

首先,既然要求因子数,我首先想到的就是素因子分解。把 $n$ 分解成 $n=p_{1}^{k_{1}}p_{2}^{k_{2}} \cdots p_{n}^{k_{n}}$ 的形式,其中 $p$ 是素数,$k$ 为他的指数。这样的话总因子个数就是 $(k_1+1) \times (k_2+1) \times (k_3+1) \cdots \times (k_n+1)$。

但是显然质因子分解的复杂度是很高的,并且前一个数的结果不能被后面利用。所以要换个方法。

我们来观察一下反素数的特点。

\begin{enumerate}
\item 反素数肯定是从 $2$ 开始的连续素数的幂次形式的乘积。
\item 数值小的素数的幂次大于等于数值大的素数,即 $n=p_{1}^{k_{1}}p_{2}^{k_{2}} \cdots p_{n}^{k_{n}}$ 中,有 $k_1 \geq k_2 \geq k_3 \geq \cdots \geq k_n$
\end{enumerate}

解释:

\begin{enumerate}
\item 如果不是从 $2$ 开始的连续素数,那么如果幂次不变,把素数变成数值更小的素数,那么此时因子个数不变,但是 $n$ 的数值变小了。交换到从 $2$ 开始的连续素数的时候 $n$ 值最小。
\item 如果数值小的素数的幂次小于数值大的素数的幂,那么如果把这两个素数交换位置(幂次不变),那么所得的 $n$ 因子数量不变,但是 $n$ 的值变小。
\end{enumerate}

另外还有两个问题,

\begin{enumerate}
\item 对于给定的 $n$,要枚举到哪一个素数呢?
\end{enumerate}

最极端的情况大不了就是 $n=p_{1}*p_{2} \cdots p_{n}$ ,所以只要连续素数连乘到刚好小于等于 $n$ 就可以的呢。再大了,连全都一次幂,都用不了,当然就是用不到的啦!

\begin{enumerate}
\item 我们要枚举到多少次幂呢?
\end{enumerate}

我们考虑一个极端情况,当我们最小的素数的某个幂次已经比所给的 $n$(的最大值)大的话,那么展开成其他的形式,最大幂次一定小于这个幂次。unsigned long long 的最大值是 2 的 64 次方,所以我这边习惯展开成 2 的 64 次方。

细节有了,那么我们具体如何具体实现呢?

我们可以把当前走到每一个素数前面的时候列举成一棵树的根节点,然后一层层的去找。找到什么时候停止呢?

\begin{enumerate}
\item 当前走到的数字已经大于我们想要的数字了
\item 当前枚举的因子已经用不到了(和 $1$ 重复了嘻嘻嘻)
\item 当前因子大于我们想要的因子了
\item 当前因子正好是我们想要的因子(此时判断是否需要更新最小 $ans$)
\end{enumerate}

然后 dfs 里面不断一层一层枚举次数继续往下迭代就好啦\textbackslash{}\textasciitilde{}\textasciitilde{}

\subsubsection{常见题型}

\paragraph{求因子数一定的最小数}

题目链接:\href{http://codeforces.com/problemset/problem/27/E}{}

对于这种题,我么只要以因子数为 dfs 的返回条件基准,不断更新找到的最小值就可以了

上代码:

\begin{cppcode}
#include <stdio.h>
#define ULL unsigned long long
#define INF ~0ULL
int p[16] = {2, 3, 5, 7, 11, 13, 17, 19, 23, 29, 31, 37, 41, 43, 47, 53};

ULL ans;
int n;

// depth: 当前在枚举第几个素数。num: 当前因子数。
// temp: 当前因子数量为 num
// 的时候的数值。up:上一个素数的幂,这次应该小于等于这个幂次嘛
void dfs(int depth, int temp, int num, int up) {
  if (num > n || depth >= 16) return;
  if (num == n && ans > temp) {
    ans = temp;
    return;
  }
  for (int i = 1; i <= up; i++) {
    if (temp / p[depth] > ans) break;
    dfs(depth + 1, temp = temp * p[depth], num * (i + 1), i);
  }
}

int main() {
  while (scanf("%d", &n) != EOF) {
    ans = INF;
    dfs(0, 1, 1, 64);
    printf("%d\n", ans);
  }
  return 0;
}
\end{cppcode}

\paragraph{求 n 以内因子数最多的数}

\href{http://acm.zju.edu.cn/onlinejudge/showProblem.do?problemId=1562}{}

思路同上,只不过要改改 dfs 的返回条件。注意这样的题目的数据范围,我一开始用了 int,应该是溢出了,在循环里可能就出不来了就超时了。上代码,0ms 过。注释就没必要写了上面写的很清楚了。

\begin{cppcode}
#include <cstdio>
#include <iostream>
#define ULL unsigned long long

int p[16] = {2, 3, 5, 7, 11, 13, 17, 19, 23, 29, 31, 37, 41, 43, 47, 53};
ULL n;
ULL ans, ans_num;  // ans 为 n 以内的最大反素数(会持续更新),ans_sum 为 ans
                   // 的因子数。

void dfs(int depth, ULL temp, ULL num, int up) {
  if (depth >= 16 || temp > n) return;
  if (num > ans_num) {
    ans = temp;
    ans_num = num;
  }
  if (num == ans_num && ans > temp) ans = temp;
  for (int i = 1; i <= up; i++) {
    if (temp * p[depth] > n) break;
    dfs(depth + 1, temp *= p[depth], num * (i + 1), i);
  }
  return;
}

int main() {
  while (scanf("%lld", &n) != EOF) {
    ans_num = 0;
    dfs(0, 1, 1, 60);
    printf("%lld\n", ans);
  }
  return 0;
}
\end{cppcode}
