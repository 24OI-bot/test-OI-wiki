
\subsection{简介}

莫比乌斯反演是数论中的重要内容。对于一些函数 $f(n)$,如果很难直接求出它的值,而容易求出其倍数和或约数和 $g(n)$,那么可以通过莫比乌斯反演简化运算,求得 $f(n)$ 的值。

开始学习莫比乌斯反演前,我们需要一些前置知识:\textbf{积性函数}、\textbf{Dirichlet 卷积}、\textbf{莫比乌斯函数}。

\hr

\subsection{积性函数}

\subsubsection{定义}

若 $\gcd(x,y)=1$ 且 $f(xy)=f(x)f(y)$,则 $f(n)$ 为积性函数。

\subsubsection{性质}

若 $f(x)$ 和 $g(x)$ 均为积性函数,则以下函数也为积性函数:

$$
\begin{align*}
h(x)&=f(x^p)\\
h(x)&=f^p(x)\\
h(x)&=f(x)g(x)\\
h(x)&=\sum_{d\mid x}f(d)g(\frac{x}{d})
\end{align*}
$$

\subsubsection{例子}

$$
\qquad\begin{array}
\text{约数个数函数}&d(n)=\displaystyle\sum_{d\mid n}1\\
\text{约数和函数}&\displaystyle\sigma(n)=\sum_{d\mid n}d\\
\text{约数 $k$ 次幂函数}&\displaystyle\sigma_k(n)=\sum_{d\mid n}d^k\\
\text{欧拉函数}&\displaystyle\varphi(n)=\sum_{i=1}^n [\gcd(i,n)=1]\\
\text{莫比乌斯函数}&\displaystyle\mu(n)=
\begin{cases}
1 & n=1\\
(-1)^k &c_{1,2,\cdots,k}=1\quad(n=\displaystyle\prod_{i=1}^k {p_i}^{c_i})\\
0 & c_i>1
\end{cases}
\end{array}
$$

\hr

\subsection{Dirichlet 卷积}

\subsubsection{定义}

定义两个数论函数 $f,g$ 的 $\text{Dirichlet}$ 卷积为

$$
(f*g)(n)=\sum_{d\mid n}f(d)g(\frac{n}{d})
$$

\subsubsection{性质}

$\text{Dirichlet}$ 卷积满足交换律和结合律。

其中 $\varepsilon$ 为 $\text{Dirichlet}$ 卷积的单位元(任何函数卷 $\varepsilon$ 都为其本身)

\subsubsection{例子}

$$
\begin{align*}
\varepsilon=\mu*1&\Leftrightarrow\varepsilon(n)=\sum_{d\mid n}\mu(d)\\
d=1*1&\Leftrightarrow d(n)=\sum_{d\mid n}1\\
\sigma=d*1&\Leftrightarrow\varepsilon(n)=\sum_{d\mid n}d\\
\varphi=\mu*\text{ID}&\Leftrightarrow\varphi(n)=\sum_{d\mid n}d\cdot\mu(\frac{n}{d})
\end{align*}
$$

\hr

\subsection{莫比乌斯函数}

\subsubsection{定义}

$\mu$ 为莫比乌斯函数

\subsubsection{性质}

莫比乌斯函数不但是积性函数,还有如下性质:

$$
\mu(n)=
\begin{cases}
1&n=1\\
0&n\text{ 含有平方因子}\\
(-1)^k&k\text{ 为 }n\text{ 的本质不同质因子个数}\\
\end{cases}
$$

\subsubsection{证明}

$$
\varepsilon(n)=
\begin{cases}
1&n=1\\
0&n\neq 1\\
\end{cases}
$$

其中 $\displaystyle\varepsilon(n)=\sum_{d\mid n}\mu(d)$ 即 $\varepsilon=\mu*1$

设 $\displaystyle n=\prod_{i=1}^k{p_i}^{c_i},n'=\prod_{i=1}^k p_i$

那么 $\displaystyle\sum_{d\mid n}\mu(d)=\sum_{d\mid n'}\mu(d)=\sum_{i=0}^k C_k^i\cdot(-1)^k$

根据二项式定理,易知该式子的值在 $k=0$ 即 $n=1$ 时值为 $1$ 否则为 $0$,这也同时证明了 $\displaystyle\sum_{d\mid n}\mu(d)=[n=1]$

\subsubsection{补充结论}

反演结论:$\displaystyle [gcd(i,j)=1] \Leftrightarrow\sum_{d\mid\gcd(i,j)}\mu(d)$

\begin{itemize}
\item \textbf{直接推导}:如果看懂了上一个结论,这个结论稍加思考便可以推出:如果 $\gcd(i,j)=1$ 的话,那么代表着我们按上个结论中枚举的那个 $n$ 是 $1$,也就是式子的值是 $1$,反之,有一个与 $[\gcd(i,j)=1]$ 相同的值:$0$
\item \textbf{利用 $\varepsilon$ 函数}:根据上一结论,$[\gcd(i,j)=1]\Rightarrow \varepsilon(\gcd(i,j))$,将 $\varepsilon$ 展开即可。
\end{itemize}

\subsubsection{线性筛}

由于 $\mu$ 函数为积性函数,因此可以线性筛莫比乌斯函数(线性筛基本可以求所有的积性函数,尽管方法不尽相同)。

\textbf{代码}:

\begin{cppcode}
void getMu() {
  mu[1] = 1;
  for (int i = 2; i <= n; ++i) {
    if (!flg[i]) p[++tot] = i, mu[i] = -1;
    for (int j = 1; j <= tot && i * p[j] <= n; ++j) {
      flg[i * p[j]] = 1;
      if (i % p[j] == 0) {
        mu[i * p[j]] = 0;
        break;
      }
      mu[i * p[j]] = -mu[i];
    }
  }
}
\end{cppcode}

\subsubsection{拓展}

证明

$$
\varphi*1=\text{ID}\text{(ID 函数即 } f(x)=x\text{)}
$$

将 $n$ 分解质因数:$\displaystyle n=\prod_{i=1}^k {p_i}^{c_i}$

首先,因为 $\varphi$ 是积性函数,故只要证明当 $n'=p^c$ 时 $\displaystyle\varphi*1=\sum_{d\mid n'}\varphi(\frac{n'}{d})=\text{ID}$ 成立即可。

因为 $p$ 是质数,于是 $d=p^0,p^1,p^2,\cdots,p^c$

易知如下过程:

$$
\begin{align*}
\varphi*1&=\sum_{d\mid n}\varphi(\frac{n}{d})\\
&=\sum_{i=0}^c\varphi(p^i)\\
&=1+p^0\cdot(p-1)+p^1\cdot(p-1)+\cdots+p^{c-1}\cdot(p-1)\\
&=p^c\\
&=\text{ID}\\
\end{align*}
$$

该式子两侧同时卷 $\mu$ 可得 $\displaystyle\varphi(n)=\sum_{d\mid n}d\cdot\mu(\frac{n}{d})$

\hr

\subsection{莫比乌斯反演}

\subsubsection{公式}

设 $f(n),g(n)$ 为两个数论函数。

如果有

$$
f(n)=\sum_{d\mid n}g(d)
$$

那么有

$$
g(n)=\sum_{d\mid n}\mu(d)f(\frac{n}{d})
$$

\subsubsection{证明}

\begin{itemize}
\item \textbf{暴力计算}:
\end{itemize}

$$
\sum_{d\mid n}\mu(d)f(\frac{n}{d})=\sum_{d\mid n}\mu(d)\sum_{k\mid \frac{n}{d}}g(k)=\sum_{k\mid n}g(k)\sum_{d\mid \frac{n}{k}}\mu(d)=g(n)
$$

用 $\displaystyle\sum_{d\mid n}g(d)$ 来替换 $f(\dfrac{n}{d})$,再变换求和顺序。最后一步转为的依据:$\displaystyle\sum_{d\mid n}\mu(d)=[n=1]$,因此在 $\dfrac{n}{k}=1$ 时第二个和式的值才为 $1$。此时 $n=k$,故原式等价于 $\displaystyle\sum_{k\mid n}[n=k]\cdot g(k)=g(n)$

\begin{itemize}
\item \textbf{运用卷积}:
\end{itemize}

原问题为:已知 $f=g*1$,证明 $g=f*\mu$

易知如下转化:$f*\mu=g*1*\mu\Rightarrow f*\mu=g$(其中 $1*\mu=\varepsilon$)

\hr

\subsection{问题形式}

\subsubsection{\href{https://www.lydsy.com/JudgeOnline/problem.php?id=2301}{「HAOI 2011」Problem b}}

求值(多组数据)

$$
\sum_{i=x}^{n}\sum_{j=y}^{m}[\gcd(i,j)=k]\qquad (1\leqslant T,x,y,n,m,k\leqslant 5\times 10^4)
$$

根据容斥原理,原式可以分成 $4$ 块来处理,每一块的式子都为

$$
\sum_{i=1}^{n}\sum_{j=1}^{m}[\gcd(i,j)=k]
$$

考虑化简该式子

$$
\sum_{i=1}^{\lfloor\frac{n}{k}\rfloor}\sum_{j=1}^{\lfloor\frac{m}{k}\rfloor}[\gcd(i,j)=1]
$$

因为 $\gcd(i,j)=1$ 时对答案才用贡献,于是我们可以将其替换为 $\varepsilon(\gcd(i,j))$($\varepsilon(n)$ 当且仅当 $n=1$ 时值为 $1$ 否则为 $0$ ),故原式化为

$$
\sum_{i=1}^{\lfloor\frac{n}{k}\rfloor}\sum_{j=1}^{\lfloor\frac{m}{k}\rfloor}\varepsilon(\gcd(i,j))
$$

将 $\varepsilon$ 函数展开得到

$$
\displaystyle\sum_{i=1}^{\lfloor\frac{n}{k}\rfloor}\sum_{j=1}^{\lfloor\frac{m}{k}\rfloor}\sum_{d\mid  \gcd(i,j)}\mu(d)
$$

变换求和顺序,先枚举 $d\mid gcd(i,j)$ 可得

$$
\displaystyle\sum_{d=1}^{\lfloor\frac{n}{k}\rfloor}\mu(d)\sum_{i=1}^{\lfloor\frac{n}{k}\rfloor}d\mid i\sum_{j=1}^{\lfloor\frac{m}{k}\rfloor}d\mid j
$$

(其中 $d\mid i$ 表示 $i$ 是 $d$ 的倍数时对答案有 $1$ 的贡献)

易知 $1\sim\lfloor\dfrac{n}{k}\rfloor$ 中 $d$ 的倍数有 $\lfloor\dfrac{n}{kd}\rfloor$ 个,故原式化为

$$
\displaystyle\sum_{d=1}^{\lfloor\frac{n}{k}\rfloor}\mu(d) \lfloor\frac{n}{kd}\rfloor\lfloor\frac{m}{kd}\rfloor
$$

很显然,式子可以数论分块求解(注意:过程中默认 $n\leqslant m$)。

\textbf{时间复杂度}:$\Theta(N+T\sqrt{n})$

\textbf{代码}:

\begin{cppcode}
#include <algorithm>
#include <cstdio>
const int N = 50000;
int mu[N + 5], p[N + 5];
bool flg[N + 5];
void init() {
  int tot = 0;
  mu[1] = 1;
  for (int i = 2; i <= N; ++i) {
    if (!flg[i]) {
      p[++tot] = i;
      mu[i] = -1;
    }
    for (int j = 1; j <= tot && i * p[j] <= N; ++j) {
      flg[i * p[j]] = 1;
      if (i % p[j] == 0) {
        mu[i * p[j]] = 0;
        break;
      }
      mu[i * p[j]] = -mu[i];
    }
  }
  for (int i = 1; i <= N; ++i) mu[i] += mu[i - 1];
}
int solve(int n, int m) {
  int res = 0;
  for (int i = 1, j; i <= std::min(n, m); i = j + 1) {
    j = std::min(n / (n / i), m / (m / i));
    res += (mu[j] - mu[i - 1]) * (n / i) * (m / i);
  }
  return res;
}
int main() {
  int T, a, b, c, d, k;
  init();
  for (scanf("%d", &T); T; --T) {
    scanf("%d%d%d%d%d", &a, &b, &c, &d, &k);
    printf("%d\n", solve(b / k, d / k) - solve(b / k, (c - 1) / k) -
                       solve((a - 1) / k, d / k) +
                       solve((a - 1) / k, (c - 1) / k));
  }
  return 0;
}
\end{cppcode}

\subsubsection{\href{https://www.luogu.org/problemnew/show/SP5971}{「SPOJ 5971」LCMSUM}}

求值(多组数据)

$$
\sum_{i=1}^n \text{lcm}(i,n)\qquad (1\leqslant T\leqslant 3\times 10^5,1\leqslant n\leqslant 10^6)
$$

易得原式即

$$
\sum_{i=1}^n \frac{i\cdot n}{\gcd(i,n)}
$$

根据 $\gcd(a,n)=1$ 时一定有 $\gcd(n-a,n)=1$ ,可将原式化为

$$
\frac{1}{2}\cdot(\sum_{i=1}^{n-1}\frac{i\cdot n}{\gcd(i,n)}+\sum_{i=n-1}^{1}\frac{i\cdot n}{\gcd(i,n)})+n
$$

上述式子中括号内的两个 $\sum$ 对应的项相等,故又可以化为

$$
\frac{1}{2}\cdot \sum_{i=1}^{n-1}\frac{n^2}{\gcd(i,n)}+n
$$

可以将相同的 $\gcd(i,n)$ 合并在一起计算,故只需要统计 $\gcd(i,n)=d$ 的个数。当 $\gcd(i,n)=d$ 时,$\displaystyle\gcd(\frac{i}{d},\frac{n}{d})=1$,所以 $\gcd(i,n)=d$ 的个数有 $\displaystyle\varphi(\frac{n}{d})$ 个。

故答案为

$$
 \frac{1}{2}\cdot\sum_{d\mid n}\frac{n^2\cdot\varphi(\frac{n}{d})}{d}+n
$$

变换求和顺序,设 $\displaystyle d'=\frac{n}{d}$,式子化为

$$
\frac{1}{2}n\cdot\sum_{d'\mid n}d'\cdot\varphi(d')+n
$$

设 $\displaystyle \text{g}(n)=\sum_{d\mid n} d\cdot\varphi(d)$,已知 $\text{g}$ 为积性函数,于是可以 $\Theta(n)$ 预处理。最后枚举 $d$,统计贡献即可。

\textbf{时间复杂度}:$\Theta(n\log n)$

\textbf{代码}:

\begin{cppcode}
#include <cstdio>
const int N = 1000000;
int tot, p[N + 5], phi[N + 5];
long long ans[N + 5];
bool flg[N + 5];

void solve() {
  phi[1] = 1;
  for (int i = 2; i <= N; ++i) {
    if (!flg[i]) p[++tot] = i, phi[i] = i - 1;
    for (int j = 1; j <= tot && i * p[j] <= N; ++j) {
      flg[i * p[j]] = 1;
      if (i % p[j] == 0) {
        phi[i * p[j]] = phi[i] * p[j];
        break;
      }
      phi[i * p[j]] = phi[i] * (p[j] - 1);
    }
  }
  for (int i = 1; i <= N; ++i) {
    for (int j = 1; i * j <= N; ++j) {
      ans[i * j] += 1LL * j * phi[j] / 2;
    }
  }
  for (int i = 1; i <= N; ++i) ans[i] = 1LL * i * ans[i] + i;
}
int main() {
  int T, n;
  solve();
  for (scanf("%d", &T); T; --T) {
    scanf("%d", &n);
    printf("%lld\n", ans[n]);
  }
  return 0;
}
\end{cppcode}

\subsubsection{\href{https://www.lydsy.com/JudgeOnline/problem.php?id=2154}{「BZOJ 2154」Crash 的数字表格}}

求值(对 $20101009$ 取模)

$$
\sum_{i=1}^n\sum_{j=1}^m\text{lcm}(i,j)\qquad (n,m\leqslant 10^7)
$$

易知原式等价于

$$
\sum_{i=1}^n\sum_{j=1}^m\frac{i\cdot j}{\gcd(i,j)}
$$

枚举最大公因数 $d$,显然两个数除以 $d$ 得到的数互质

$$
\sum_{i=1}^n\sum_{j=1}^m\sum_{d\mid i,d\mid j,\gcd(\frac{i}{d},\frac{j}{d})=1}\frac{i\cdot j}{d}
$$

非常经典的 $\gcd$ 式子的化法

$$
\sum_{d=1}^n d\cdot\sum_{i=1}^{\lfloor\frac{n}{d}\rfloor}\sum_{j=1}^{\lfloor\frac{m}{d}\rfloor}[\gcd(i,j)=1]\ i\cdot j
$$

后半段式子中,出现了互质数对之积的和,为了让式子更简洁就把它拿出来单独计算。于是我们记

$$
\text{sum}(n,m)=\sum_{i=1}^n\sum_{j=1}^m [\gcd(i,j)=1]\  i\cdot j
$$

接下来对 $\text{sum}(n,m)$ 进行化简。首先枚举约数,并将 $[\gcd(i,j)=1]$ 表示为 $\varepsilon(\gcd(i,j))$

$$
\sum_{d=1}^n\sum_{d\mid i}^n\sum_{d\mid j}^m\mu(d)\cdot i\cdot j
$$

设 $i=i'\cdot d$,$j=j'\cdot d$,显然式子可以变为

$$
\sum_{d=1}^n\mu(d)\cdot d^2\cdot\sum_{i=1}^{\lfloor\frac{n}{d}\rfloor}\sum_{j=1}^{\lfloor\frac{m}{d}\rfloor}i\cdot j
$$

观察上式,前半段可以预处理前缀和;后半段又是一个范围内数对之和,记

$$
g(n,m)=\sum_{i=1}^n\sum_{j=1}^m i\cdot j=\frac{n\cdot(n+1)}{2}\times\frac{m\cdot(m+1)}{2}
$$

可以 $\Theta(1)$ 求解

至此

$$
\text{sum}(n,m)=\sum_{d=1}^n\mu(d)\cdot d^2\cdot g(\lfloor\frac{n}{d}\rfloor,\lfloor\frac{m}{d}\rfloor)
$$

我们可以 $\lfloor\frac{n}{\lfloor\frac{n}{d}\rfloor}\rfloor$ 数论分块求解 $\text{sum}(n,m)$ 函数。

在求出 $\text{sum}(n,m)$ 后,回到定义 $\text{sum}$ 的地方,可得原式为

$$
\sum_{d=1}^n d\cdot\text{sum}(\lfloor\frac{n}{d}\rfloor,\lfloor\frac{m}{d}\rfloor)
$$

可见这又是一个可以数论分块求解的式子!

本题除了推式子比较复杂、代码细节较多之外,是一道很好的莫比乌斯反演练习题!(上述过程中,默认 $n\leqslant m$)

\textbf{时间复杂度}:$\Theta(n+m)$(两次数论分块)

\textbf{代码}:

\begin{cppcode}
#include <algorithm>
#include <cstdio>
using std::min;

const int N = 1e7;
const int mod = 20101009;
int n, m, mu[N + 5], p[N / 10 + 5], sum[N + 5];
bool flg[N + 5];

void init() {
  mu[1] = 1;
  int tot = 0, k = min(n, m);
  for (int i = 2; i <= k; ++i) {
    if (!flg[i]) p[++tot] = i, mu[i] = -1;
    for (int j = 1; j <= tot && i * p[j] <= k; ++j) {
      flg[i * p[j]] = 1;
      if (i % p[j] == 0) {
        mu[i * p[j]] = 0;
        break;
      }
      mu[i * p[j]] = -mu[i];
    }
  }
  for (int i = 1; i <= k; ++i)
    sum[i] = (sum[i - 1] + 1LL * i * i % mod * (mu[i] + mod)) % mod;
}
int Sum(int x, int y) {
  return (1LL * x * (x + 1) / 2 % mod) * (1LL * y * (y + 1) / 2 % mod) % mod;
}
int func(int x, int y) {
  int res = 0;
  for (int i = 1, j; i <= min(x, y); i = j + 1) {
    j = min(x / (x / i), y / (y / i));
    res = (res + 1LL * (sum[j] - sum[i - 1] + mod) * Sum(x / i, y / i) % mod) %
          mod;
  }
  return res;
}
int solve(int x, int y) {
  int res = 0;
  for (int i = 1, j; i <= min(x, y); i = j + 1) {
    j = min(x / (x / i), y / (y / i));
    res = (res +
           1LL * (j - i + 1) * (i + j) / 2 % mod * func(x / i, y / i) % mod) %
          mod;
  }
  return res;
}
int main() {
  scanf("%d%d", &n, &m);
  init();
  printf("%d\n", solve(n, m));
}
\end{cppcode}

\begin{QUOTE}{}{}
本文部分内容引用于 \href{https://algocode.net}{algocode 算法博客},特别鸣谢!
\end{QUOTE}
