
假设班里有 $10$ 个学生喜欢数学, $15$ 个学生喜欢语文, $21$ 个学生喜欢编程,班里至少喜欢一门学科的有多少个学生呢?是 $10+15+21=46$ 个吗?不是的,因为有些学生可能同时喜欢数学和语文,或者语文和编程,甚至还有可能三者都喜欢。为了叙述方便,我们把喜欢语文、数学、编程的学生集合分别用 $A,B,C$ 表示,则学生总数等于 $|A\cup B\cup C|$。刚才已经讲过,如果把这三个集合的元素个数 $|A|,|B|,|C|$ 直接加起来,会有一些元素重复统计了,因此需要扣掉 $|A\cap B|,|B\cap C|,|C\cap A|$ ,但这样一来,又有一小部分多扣了,需要加回来,即 $|A\cap B\cap C|$ 。即

$|A\cup B\cup C|=|A|+|B|+|C|-|A\cap B|-|B\cap C|-|C\cap A|+|A\cap B\cap C|$

一般地,对于任意多个集合,我们都可以列出这样一个等式,等式左边是所有集合的并的元素个数,右边是这些集合的各种搭配。每个搭配都是若干个集合的交集,且每一项前面的正负号取决于集合的个数————奇数个集合为正,偶数个集合为负。即

设 $S$ 为有限集, $A_i\in S~(i=1,2,...,n~,~n\ge 2)$ ,则有

$| \bigcup_{i=1}^n A_i | =\sum_{k=1}^n (-1)^{(k-1)} \times \sum_{1\le i_1<i_2<...<i_k\le n} |A_{i_1}\cap A_{i_2} \cap ...\cap A_{i_k}|$

\subsection{容斥原理在算法竞赛中的应用}

\begin{NOTE}{ 例题 [BZOJ 1042 \[HAOI2008\] 硬币购物](https://www.lydsy.com/JudgeOnline/problem.php?id=1042)}{}
题目大意:一共有 $4$ 种硬币,面值分别为 $c_1,c_2,c_3,c_4$ 。某人去买东西,去了 $tot$ 次。每次给出 $d1,d2,d3,d4$ ,$d_i$ 表示有 $i$ 个面值为 $c_i$ 的硬币,求购买价值为 $s$ 的物品的付款方案数。
\end{NOTE}


先用多重背包预处理出 $f(i)$ ,表示不限制钞票数量购买价格为 $i$ 的物品的方案数。由于容斥原理,我们最后的答案为

$f(s)-f(s-d_1)-f(s-d_2)-f(s-d_3)-f(s-d_4)$

$+f(s-d_1-d_2)+f(s-d_1-d_3)+f(s-d_1-d_4)+f(s-d_2-d_3)+f(s-d_2-d_4)+f(s-d_3-d_4)$

$-f(s-d_1-d_2-d_3)-f(s-d_1-d_2-d_4)-f(s-d_1-d_3-d_4)-f(s-d_2-d_3-d_4)+f(s-d_1-d_2-d_3-d_4)$

这样,我们就可以在 $O(1)$ 的时间复杂度内处理每个询问。

\subsubsection{练习}

\href{https://www.lydsy.com/JudgeOnline/problem.php?id=4665}{BZOJ 4665 小 w 的喜糖}

\href{https://www.lydsy.com/JudgeOnline/problem.php?id=4361}{BZOJ 4361 isn}
