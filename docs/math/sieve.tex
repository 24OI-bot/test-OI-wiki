
\subsection{素数筛法}

如果我们想要知道小于等于 $n$ 有多少个素数呢?

一个自然的想法是我们对于小于等于 $n$ 的每个数进行一次判定。这种暴力的做法显然不能达到最优复杂度,考虑如何优化。

考虑这样一件事情:如果 $x$ 是合数,那么 $x$ 的倍数也一定是合数。利用这个结论,我们可以避免很多次不必要的检测。

如果我们从小到大考虑每个数,然后同时把当前这个数的所有(比自己大的)倍数记为合数,那么运行结束的时候没有被标记的数就是素数了。

\begin{cppcode}
int Eratosthenes(int n) {
  int p = 0;
  for (int i = 0; i <= n; ++i) is_prime[i] = 1;
  is_prime[0] = is_prime[1] = 0;
  for (int i = 2; i <= n; ++i) {
    if (is_prime[i]) {
      prime[p++] = i;  // prime[p]是i,后置自增运算代表当前素数数量
      for (int j = 2 * i; j <= n; j += i)
        is_prime[j] = 0;  //是i的倍数的均不是素数
    }
  }
  return p;
}
\end{cppcode}

以上为 \textbf{Eratosthenes 筛法}(埃拉托斯特尼筛法),时间复杂度是 $O(n\log\log n)$。

以上做法仍有优化空间,我们发现这里面似乎会对某些数标记了很多次其为合数。有没有什么办法省掉无意义的步骤呢?

答案当然是:有!

如果能让每个合数都只被标记一次,那么时间复杂度就可以降到 $O(n)$ 了

\begin{cppcode}
void init() {
  phi[1] = 1;
  f(i, 2, MAXN) {
    if (!vis[i]) {
      phi[i] = i - 1;
      pri[cnt++] = i;
    }
    f(j, 0, cnt) {
      if ((LL)i * pri[j] >= MAXN) break;
      vis[i * pri[j]] = 1;
      if (i % pri[j]) {
        phi[i * pri[j]] = phi[i] * (pri[j] - 1);
      } else {
        // i % pri[j] == 0
        // 换言之,i 之前被 pri[j] 筛过了
        // 由于 pri 里面质数是从小到大的,所以 i 乘上其他的质数的结果一定也是
        // pri[j] 的倍数 它们都被筛过了,就不需要再筛了,所以这里直接 break
        // 掉就好了
        phi[i * pri[j]] = phi[i] * pri[j];
        break;
      }
    }
  }
}
\end{cppcode}

上面代码中的 $phi$ 数组,会在下面提到。

上面的这种\textbf{线性筛法}也称为 \textbf{Euler 筛法}(欧拉筛法)。

\subsection{筛法求欧拉函数}

注意到在线性筛中,每一个合数都是被最小的质因子筛掉。比如设 $p_1$ 是 $n$ 的最小质因子,$n' = \frac{n}{p_1}$,那么线性筛的过程中 $n$ 通过 $n' \times p_1$ 筛掉。

观察线性筛的过程,我们还需要处理两个部分,下面对$n' \bmod p_1$ 分情况讨论。

如果 $n' \bmod p_1 = 0$,那么 $n'$ 包含了 $n$ 的所有质因子。

$$
\begin{aligned}
\varphi(n) & = n \times \prod_{i = 1}^s{\frac{p_i - 1}{p_i}} \\\\
& = p_1 \times n' \times \prod_{i = 1}^s{\frac{p_i - 1}{p_i}} \\\\
& = p_1 \times \varphi(n')
\end{aligned}
$$

那如果 $n' \bmod p_1 \neq 0$ 呢,这时 $n'$ 和 $n$ 是互质的,根据欧拉函数性质,我们有:

$$
\begin{aligned}
\varphi(n) & = \varphi(p_1) \times \varphi(n') \\\\
& = (p_1 - 1) \times \varphi(n')
\end{aligned}
$$

\begin{cppcode}
void phi_table(int n, int* phi) {
  for (int i = 2; i <= n; i++) phi[i] = 0;
  phi[1] = 1;
  for (int i = 2; i <= n; i++)
    if (!phi[i])
      for (int j = i; j <= n; j += i) {
        if (!phi[j]) phi[j] = j;
        phi[j] = phi[j] / i * (i - 1);
      }
}
\end{cppcode}

\subsection{筛法求莫比乌斯函数}

\subsection{筛法求约数个数}

\subsection{其他线性函数}
