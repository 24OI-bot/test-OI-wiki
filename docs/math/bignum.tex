
什么时候需要高精度呢?就比如数据规模很大,unsigned long long 都存不下,就需要开一个字符数组来准确地表示一个数。

高精度问题包含很多小的细节,实现上也有很多讲究,暂时先不展开。

\begin{itemize}
\item 四则运算
\item 快速幂
\item 分数
\item 对数(?)
\item 开根
\item 压位高精度
\end{itemize}

放一个之前的高精度板子吧。

还有一个很好用的\href{https://paste.ubuntu.com/p/7VKYzpC7dn/}{高精度封装类} 10kb 想用可以自行下载。

\begin{cppcode}
#define MAXN 9999
// MAXN 是一位中最大的数字
#define MAXSIZE 10024
// MAXSIZE 是位数
#define DLEN 4
// DLEN 记录压几位
struct Big {
  int a[MAXSIZE], len;
  bool flag;  //标记符号'-'
  Big() {
    len = 1;
    memset(a, 0, sizeof a);
    flag = 0;
  }
  Big(const int);
  Big(const char*);
  Big(const Big&);
  Big& operator=(const Big&);  //注意这里operator有&,因为赋值有修改……
  //由于OI中要求效率
  //此处不使用泛型函数
  //故不重载
  // istream& operator>>(istream&,  BigNum&);   //重载输入运算符
  // ostream& operator<<(ostream&,  BigNum&);   //重载输出运算符
  Big operator+(const Big&) const;
  Big operator-(const Big&) const;
  Big operator*(const Big&)const;
  Big operator/(const int&) const;
  // TODO: Big / Big;
  Big operator^(const int&) const;
  // TODO: Big ^ Big;

  // TODO: Big 位运算;

  int operator%(const int&) const;
  // TODO: Big ^ Big;
  bool operator<(const Big&) const;
  bool operator<(const int& t) const;
  inline void print();
};
// README::不要随随便便把参数都变成引用,那样没办法传值
Big::Big(const int b) {
  int c, d = b;
  len = 0;
  // memset(a,0,sizeof a);
  CLR(a);
  while (d > MAXN) {
    c = d - (d / (MAXN + 1) * (MAXN + 1));
    d = d / (MAXN + 1);
    a[len++] = c;
  }
  a[len++] = d;
}
Big::Big(const char* s) {
  int t, k, index, l;
  CLR(a);
  l = strlen(s);
  len = l / DLEN;
  if (l % DLEN) ++len;
  index = 0;
  for (int i = l - 1; i >= 0; i -= DLEN) {
    t = 0;
    k = i - DLEN + 1;
    if (k < 0) k = 0;
    g(j, k, i) t = t * 10 + s[j] - '0';
    a[index++] = t;
  }
}
Big::Big(const Big& T) : len(T.len) {
  CLR(a);
  f(i, 0, len) a[i] = T.a[i];
  // TODO:重载此处?
}
Big& Big::operator=(const Big& T) {
  CLR(a);
  len = T.len;
  f(i, 0, len) a[i] = T.a[i];
  return *this;
}
Big Big::operator+(const Big& T) const {
  Big t(*this);
  int big = len;
  if (T.len > len) big = T.len;
  f(i, 0, big) {
    t.a[i] += T.a[i];
    if (t.a[i] > MAXN) {
      ++t.a[i + 1];
      t.a[i] -= MAXN + 1;
    }
  }
  if (t.a[big])
    t.len = big + 1;
  else
    t.len = big;
  return t;
}
Big Big::operator-(const Big& T) const {
  int big;
  bool ctf;
  Big t1, t2;
  if (*this < T) {
    t1 = T;
    t2 = *this;
    ctf = 1;
  } else {
    t1 = *this;
    t2 = T;
    ctf = 0;
  }
  big = t1.len;
  int j = 0;
  f(i, 0, big) {
    if (t1.a[i] < t2.a[i]) {
      j = i + 1;
      while (t1.a[j] == 0) ++j;
      --t1.a[j--];
      // WTF?
      while (j > i) t1.a[j--] += MAXN;
      t1.a[i] += MAXN + 1 - t2.a[i];
    } else
      t1.a[i] -= t2.a[i];
  }
  t1.len = big;
  while (t1.len > 1 && t1.a[t1.len - 1] == 0) {
    --t1.len;
    --big;
  }
  if (ctf) t1.a[big - 1] = -t1.a[big - 1];
  return t1;
}
Big Big::operator*(const Big& T) const {
  Big res;
  int up;
  int te, tee;
  f(i, 0, len) {
    up = 0;
    f(j, 0, T.len) {
      te = a[i] * T.a[j] + res.a[i + j] + up;
      if (te > MAXN) {
        tee = te - te / (MAXN + 1) * (MAXN + 1);
        up = te / (MAXN + 1);
        res.a[i + j] = tee;
      } else {
        up = 0;
        res.a[i + j] = te;
      }
    }
    if (up) res.a[i + T.len] = up;
  }
  res.len = len + T.len;
  while (res.len > 1 && res.a[res.len - 1] == 0) --res.len;
  return res;
}
Big Big::operator/(const int& b) const {
  Big res;
  int down = 0;
  gd(i, len - 1, 0) {
    res.a[i] = (a[i] + down * (MAXN + 1) / b);
    down = a[i] + down * (MAXN + 1) - res.a[i] * b;
  }
  res.len = len;
  while (res.len > 1 && res.a[res.len - 1] == 0) --res.len;
  return res;
}
int Big::operator%(const int& b) const {
  int d = 0;
  gd(i, len - 1, 0) d = (d * (MAXN + 1) % b + a[i]) % b;
  return d;
}
Big Big::operator^(const int& n) const {
  Big t(n), res(1);
  // TODO::快速幂这样写好丑= =//DONE:)
  int y = n;
  while (y) {
    if (y & 1) res = res * t;
    t = t * t;
    y >>= 1;
  }
  return res;
}
bool Big::operator<(const Big& T) const {
  int ln;
  if (len < T.len) return 233;
  if (len == T.len) {
    ln = len - 1;
    while (ln >= 0 && a[ln] == T.a[ln]) --ln;
    if (ln >= 0 && a[ln] < T.a[ln]) return 233;
    return 0;
  }
  return 0;
}
inline bool Big::operator<(const int& t) const {
  Big tee(t);
  return *this < tee;
}
inline void Big::print() {
  printf("%d", a[len - 1]);
  gd(i, len - 2, 0) { printf("%04d", a[i]); }
}

inline void print(Big s) {
  // s不要是引用,要不然你怎么print(a * b);
  int len = s.len;
  printf("%d", s.a[len - 1]);
  gd(i, len - 2, 0) { printf("%04d", s.a[i]); }
}
char s[100024];
\end{cppcode}
