
\subsection{事件}

\subsubsection{单位事件、事件空间、随机事件}

在一次随机试验中可能发生的不能再细分的结果被称为单位事件,用 $E$ 表示。在随机试验中可能发生的所有单位事件的集合称为事件空间,用 $S$ 来表示。例如在一次掷骰子的随机试验中,如果用获得的点数来表示单位事件,那么一共可能出现 $6$ 个单位事件,则事件空间可以表示为 $S=\{1,2,3,4,5,6\}$ 。

随机事件是事件空间 $S$ 的子集,它由事件空间 $S$ 中的单位元素构成,用大写字母 $A, B, C,\ldots$ 表示。例如在掷两个骰子的随机试验中,设随机事件 $A$ 为 “获得的点数和大于 $10$ ” ,则 $A$ 可以由下面 $3$ 个单位事件组成: $A = \{ (5,6),(6,5),(6,6)\}$ 。

\subsubsection{事件的计算}

因为事件在一定程度上是以集合的含义定义的,因此可以把集合计算方法直接应用于事件的计算,也就是说,在计算过程中,可以把事件当作集合来对待。

\textbf{ 和事件 } :相当于 \textbf{ 并集 } 。只需其中之一发生,就发生了。

\textbf{ 积事件 } :相当于 \textbf{ 交集 } 。必须要全都发生,才计算概率。

\subsection{概率}

\subsubsection{定义}

如果在相同条件下,进行了 n 次试验,事件 A 发生了$N_A$ 次,那么$\frac{N_A}{n}$ 称为事件 A 发生的概率。

\subsubsection{公理}

\textbf{ 非负性 } :对于一个事件 $A$ ,有概率 $P(A)\in [0,1]$ 。

\textbf{ 规范性 } :事件空间的概率值为 $1$,$P(S)=1$.

\textbf{ 容斥性 } :若 $P(A+B) = P(A)+P(B)$ ,则  $A$ 和 $B$ 互为独立事件。

\subsubsection{计算}

\textbf{ 全概率公式 } :若事件 $A_1,A_2,\ldots,A_n$ 构成一个完备的事件且都有正概率,即 $\forall i,j, A_i\cap A_j=\varnothing$ 且 $\displaystyle \sum_{i=1}^n A_i=1$,有 $\displaystyle P(B)=\sum_{i=1}^n P(A_i)P(B|A_i)$ 。

\textbf{ 贝叶斯定理 } : $\displaystyle P(B_i|A)=\frac{P(B_i)P(A|B_i)}{\displaystyle \sum_{j=1}^n P(B_j)P(A|B_j)}$

公式中,事件 $B_i$ 的概率为 $P(B_i)$ ,事件 $B_i$ 已发生条件下事件 $A$ 的概率为 $P(A|B_i)$ ,事件 $A$ 发生条件下事件 $B_i$ 的概率为 $P(B_i|A)$ 。

\subsection{期望}

\subsubsection{定义}

在一定区间内变量取值为有限个,或数值可以一一列举出来的变量称为离散型随机变量。一个离散性随机变量的数学期望是试验中每次可能的结果乘以其结果概率的总和。

\subsubsection{性质}

\textbf{ 全期望公式 } : $E(Y)=E[E(Y|X)]$ 。可由全概率公式证明。

\textbf{ 线性性质 } : 对于任意两个随机事件 $x,y$ ( \textbf{ 不要求相互独立 } ) ,有 $E(X+Y)=E(X)+E(Y)$ 。

\subsection{例题}

\href{https://ti.luogu.com.cn/problemset/1022}{NOIP2017 初赛 T14, T15}

\href{https://www.luogu.org/problemnew/show/P1850}{NOIP2016 换教室} (概率期望 DP)
