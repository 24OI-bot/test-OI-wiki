
\subsection{「物不知数」问题}

\begin{QUOTE}{}{}
有物不知其数,三三数之剩二,五五数之剩三,七七数之剩二。问物几何?
\end{QUOTE}

即求满足以下条件的整数:除以 $3$ 余 $2$,除以 $5$ 余 $3$,除以 $7$ 余 $2$。

该问题最早见于《孙子算经》中,并有该问题的具体解法。宋朝数学家秦九韶于 1247 年《数书九章》卷一、二《大衍类》对「物不知数」问题做出了完整系统的解答。上面具体问题的解答口诀由明朝数学家程大位在《算法统宗》中给出:

\begin{QUOTE}{}{}
三人同行七十希,五树梅花廿一支,七子团圆正半月,除百零五便得知。
\end{QUOTE}

$2\times 70+3\times 21+2\times 15=233=2\times 105+23$,故答案为 $23$。

\subsection{算法简介及过程}

中国剩余定理 (Chinese Remainder Theorem, CRT) 可求解如下形式的一元线性同余方程组(其中 $n_1, n_2, \cdots, n_k$ 两两互质):

$$
\left \{
\begin{array}{c}
x &\equiv& a_1 \pmod {n_1} \\
x &\equiv& a_2 \pmod {n_2} \\
  &\vdots& \\
x &\equiv& a_n \pmod {n_k} \\
\end{array}
\right.
$$

上面的「物不知数」问题就是一元线性同余方程组的一个实例。

\subsubsection{算法流程}

\begin{enumerate}
\item 计算所有模数的积 $n$;
\item 对于第 $i$ 个方程:
\begin{enumerate}
\item 计算 $m_i=\frac{n}{n_i}$;
\item 计算 $m_i$ 在模 $n_i$ 意义下的\href{/math/inverse/}{逆元} $m_i^{-1}$;
\item 计算 $c_i=m_im_i^{-1}$(\textbf{不要对 $n_i$ 取模})。
\end{enumerate}
\item 方程组的唯一解为:$a=\sum_{i=1}^k a_ic_i \pmod n$。
\end{enumerate}

\subsubsection{伪代码}

\vskip 0.2 in
\texttt{
1 → n\\0 → ans\\for i = 1 to k\\	n * n[i] → n\\for i = 1 to k\\	n / n[i] → m\\	inv(m, n[i]) → b               // b * m mod n[i] = 1\\	(ans + m * b) mod n → ans\\return ans}
\vskip 0.2 in

\subsection{算法的证明}

我们需要证明上面算法计算所得的 $a$ 对于任意 $i=1,2,\cdots,k$ 满足 $a\equiv a_i \pmod {n_i}$。

当 $i\neq j$ 时,有 $m_j\equiv 0 \pmod {n_i}$,故 $c_j\equiv m_j\equiv 0 \pmod {n_i}$。又有 $c_i\equiv m_i(m_i^{-1}\bmod {n_i})\equiv 1 \pmod {n_i}$,所以我们有:

$$
\begin{aligned}
a&\equiv \sum_{j=1}^k a_jc_j        &\pmod {n_i} \\
 &\equiv a_ic_i                     &\pmod {n_i} \\
 &\equiv a_im_i(m^{-1}_i \bmod n_i) &\pmod {n_i} \\
 &\equiv a_i                        &\pmod {n_i}
\end{aligned}
$$

\textbf{即对于任意 $i=1,2,\cdots,k$,上面算法得到的 $a$ 总是满足 $a\equiv a_i \pmod{n_i}$,即证明了解同余方程组的算法的正确性。}

因为我们没有对输入的 $a_i$ 作特殊限制,所以任何一组输入 $\{a_i\}$ 都对应一个解 $a$。

另外,若 $x\neq y$,则总存在 $i$ 使得 $x$ 和 $y$ 在模 $n_i$ 下不同余。

\textbf{故系数列表 $\{a_i\}$ 与解 $a$ 之间是一一映射关系,方程组总是有唯一解。}

\subsection{例}

下面演示 CRT 如何解「物不知数」问题。

\begin{enumerate}
\item $n=3\times 5\times 7=105$;
\item 三人同行\textbf{七十}希:$n_1=3, m_1=n/n_1=35, m_1^{-1}\equiv 2\pmod 3$,故 $c_1=35\times 2=70$;
\item 五树梅花\textbf{廿一}支:$n_2=5, m_2=n/n_2=21, m_2^{-1}\equiv 1\pmod 5$,故 $c_2=21\times 1=21$;
\item 七子团圆正\textbf{半月}:$n_3=7, m_3=n/n_3=15, m_3^{-1}\equiv 1\pmod 7$,故 $c_3=15\times 1=15$;
\item 所以方程组的唯一解为 $a\equiv 2\times 70+3\times 21+2\times 15\equiv 233\equiv 23 \pmod {105}$。(除\textbf{百零五}便得知)
\end{enumerate}

\subsection{应用}

某些计数问题或数论问题出于加长代码、增加难度、或者是一些其他不可告人的原因,给出的模数:\textbf{不是质数}!

但是对其质因数分解会发现它没有平方因子,也就是该模数是由一些不重复的质数相乘得到。

那么我们可以分别对这些模数进行计算,最后用 CRT 合并答案。

推荐练习:BZOJ 1951

\subsection{比较两 CRT 下整数}

考虑 CRT, 不妨假设$n_1\leq n_2 \leq ... \leq n_k$

$$
\left \{
\begin{array}{c}
x &\equiv& a_1 \pmod {n_1} \\
x &\equiv& a_2 \pmod {n_2} \\
  &\vdots& \\
x &\equiv& a_n \pmod {n_k} \\
\end{array}
\right.
$$

与 PMR(Primorial Mixed Radix) 表示

$x=b_1+b_2n_1+b_3n_1n_2...+b_kn_1n_2...n_{k-1} ,b_i\in [0,n_i)$

将数字转化到 PMR 下, 逐位比较即可

转化方法考虑依次对 PMR 取模

$$
\begin{align}
b_1&=a_1 \mod n_1\\
b_2&=(a_2-b_1)c_{1,2} \mod n_2\\
b_3&=((a_3-b_1')c_{1,3}-x_2')c_{2,3} \mod n_3\\
&...\\
b_k&=(...((a_k-b_1)c_{1,k}-b_2)c_{2,k})-...)c_{k-1,k} \mod n_k
\end{align}
$$

其中$c_{i,j}$表示$n_i$对$n_j$的逆元,$c_{i,j}n_i=1 \mod n_j$

\subsection{扩展:模数不互质的情况}

\subsubsection{两个方程}

设两个方程分别是 $x\equiv a_1 \pmod {m_1}$、$x\equiv a_2 \pmod {m_2}$;

将它们转化为不定方程:$x=m_1p+a_1=m_2q+a_2$,其中 $p, q$ 是整数,则有 $m_1p-m_2q=a_2-a_1$。

由裴蜀定理,当 $a_2-a_1$ 不能被 $\gcd(m_1,m_2)$ 整除时,无解;

其他情况下,可以通过扩展欧几里得算法解出来一组可行解 $(p, q)$;

则原来的两方程组成的模方程组的解为 $x\equiv b\pmod M$,其中 $b=m_1p+a_1$,$M=\text{lcm}(m_1, m_2)$。

\subsubsection{多个方程}

用上面的方法两两合并就可以了……

推荐练习:POJ 2891

\href{https://www.luogu.org/problemnew/show/P4777}{【模板】扩展中国剩余定理}

\href{https://www.luogu.org/problemnew/show/P4774}{\textbackslash{}[NOI2018\textbackslash{}] 屠龙勇士}

\href{https://www.luogu.org/problemnew/show/P3868}{\textbackslash{}[TJOI2009\textbackslash{}] 猜数字}

\href{https://www.luogu.org/problemnew/show/P2480}{\textbackslash{}[SDOI2010\textbackslash{}] 古代猪文}
