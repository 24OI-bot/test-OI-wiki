
\subsection{最大公约数}

最大公约数即为 Greatest Common Divisor,常缩写为 gcd

在 \href{/math/prime}{素数} 一节中,我们已经介绍了约数的概念。

一组数的公约数,是指同时是这组数中每一个数的约数的数。而最大公约数,则是指所有公约数里面最大的一个。

那么如何求最大公约数呢?我们先考虑两个数的情况。

\subsubsection{两个数的}

如果我们已知两个数 $a$ 和 $b$,如何求出二者的最大公约数呢?

不妨设 $a > b$

我们发现如果 $b$ 是 $a$ 的约数,那么 $b$ 就是二者的最大公约数。

下面讨论不能整除的情况,即 $a = b \times q + r$,其中 $r < b$。

我们通过证明可以得到$\gcd(a,b)=\gcd(b,a \bmod b)$,过程如下:

\hr

设$a=bk+c$,显然有$c=a \bmod b$。设$d|a\ \ \ d|b$,则

$c=a-bk$

$\frac{c}{d}=\frac{a}{d}-\frac{b}{d}k$

由右边的式子可知$\frac{c}{d}$为整数,即$d|c$所以对于$a,b$的公约数,它也会是$a \bmod b$的公约数。

反过来也需要证明

设$d|b\ \ \ d|(a \bmod b)$,我们还是可以像之前一样得到以下式子

$\frac{a\bmod b}{d}=\frac{a}{d}-\frac{b}{d}k$

$\frac{a\bmod b}{d}+\frac{b}{d}k=\frac{a}{d}$

因为左边式子显然为整数,所以$\frac{a}{d}$也为整数,即$d|a$,所以$b,a\bmod b$的公约数也是$a,b$的公约数。

既然两式公约数都是相同的,那么最大公约数也会相同

所以得到式子

$\gcd(a,b)=\gcd(b,a\bmod b)$

既然得到了$\gcd(a, b) = \gcd(b, r)$,这里两个数的大小是不会增大的,那么我们也就得到了关于两个数的最大公约数的一个递归求法。

\begin{cppcode}
int gcd(int a, int b) {
  if (b == 0) return a;
  return gcd(b, a % b);
}
\end{cppcode}

递归至\texttt{b==0}(即上一步的\texttt{a%b==0}) 的情况再返回值即可。

\subsubsection{多个数的}

那怎么求多个书的最大公约数呢?显然答案一定是每个数的约数,那么也一定是每相邻两个数的约数。我们采用归纳法,可以证明,每次取出两个数求出答案后再放回去,不会对所需要的答案造成影响。

\subsection{最小公倍数}

\subsubsection{两个数的}

首先我们介绍这样一个定理 —— 算术基本定理:

\begin{QUOTE}{}{}
 每一个正整数都可以表示成若干整数的乘积,这种分解方式在忽略排列次序的条件下是唯一的。
\end{QUOTE}

用数学公式来表示就是 $x = p_1^{k_1}p_2^{k_2} \cdots p_s^{k_s}$

设 $a = p_{a_1}^{k_{a_1}}p_{a_2}^{k_{a_2}} \cdots p_{a_s}^{k_{a_s}}$, $b = p_{b_1}^{k_{b_1}}p_{b_2}^{k_{b_2}} \cdots p_{b_s}^{k_{b_s}}$

我们发现,对于 $a$ 和 $b$ 的情况,二者的最大公约数等于

$p_1^{k_{\min(a_1, b_1)}}p_2^{k_{\min(a_2, b_2)}} \cdots p_s^{k_{\min(a_s, b_s)}}$

最小公倍数等于

$p_1^{k_{\max(a_1, b_1)}}p_2^{k_{\max(a_2, b_2)}} \cdots p_s^{k_{\max(a_s, b_s)}}$

由于 $a + b = \max(a, b) + \min(a, b)$

所以得到结论是 $\gcd(a, b) \times \operatorname{lcm}(a, b) = a \times b$

要求两个数的最小公倍数,先求出最大公约数即可。

\subsubsection{多个数的}

可以发现,当我们求出两个数的$gcd$时,求最小公倍数是$O(1)$的复杂度。那么对于多个数,我们其实没有必要求一个共同的最大公约数再去处理,最直接的方法就是,当我们算出两个数的$gcd$,或许在求多个数的$gcd$时候,我们将它放入序列对后面的数继续求解,那么,我们转换一下,直接将最小公倍数放入序列即可

\subsection{EXGCD - 扩展欧几里得定理}

目的:求$ax+by=\gcd(a,b)$的一组可行解

\subsection{证明}

设

$ax_1+by_1=\gcd(a,b)$

$bx_2+(a\bmod b)y_2=\gcd(b,a\bmod b)$

由欧几里得定理可知:

$\gcd(a,b)=\gcd(b,a\bmod b)$

所以

$ax_1+by_1=bx_2+(a\bmod b)y_2$

又因为

$a\bmod b=a-(\lfloor\frac{a}{b}\rfloor\times b)$

所以

$ax_1+by_1=bx_2+(a-(\lfloor\frac{a}{b}\rfloor\times b))y_2$

$ax_1+by_1=ay_2+bx_2-\lfloor\frac{a}{b}\rfloor\times by_2=ay_2+b(x_2-\lfloor\frac{a}{b}\rfloor y_2)$

因为 $a=a,b=b$ ,所以

$x_1=y_2,y_1=x_2-\lfloor\frac{a}{b}\rfloor y_2$ 

将 $x_2,y_2$ 不断代入递归求解直至 GCD(最大公约数,下同) 为 \texttt{0} 递归 \texttt{x=1,y=0} 回去求解。

\begin{cppcode}
int Exgcd(int a, int b, int &x, int &y) {
  if (!b) {
    x = 1;
    y = 0;
    return a;
  }
  int d = Exgcd(b, a % b, x, y);
  int t = x;
  x = y;
  y = t - (a / b) * y;
  return d;
}
\end{cppcode}

函数返回的值为 GCD,在这个过程中计算 $x,y$ 即可
