
\subsection{逆元简介}

如果一个线性同余方程 $ax \equiv 1 \pmod b$,则 $x$ 称为 $a \mod b$ 的逆元,记作 $a^{-1}$。

\subsection{如何求逆元}

\subsubsection{扩展欧几里得法:}

\begin{cppcode}
void ex_gcd(int a, int b, int& x, int& y) {
  if (b == 0) {
    x = 1, y = 0;
    return;
  }
  ex_gcd(b, a % b, x, y);
  int t = x;
  x = y;
  y = t - a / b * y;
  return;
}
\end{cppcode}

扩展欧几里得法和求解 \href{/math/linear-equation/}{线性同余方程} 是一个原理,在这里不展开解释。

\subsubsection{快速幂法:}

这个要运用 \href{/math/fermat/}{费马小定理}:

\begin{QUOTE}{}{}
若 $p$ 为质数,$a$ 为正整数,且 $a$、$p$ 互质,则 $a^{p-1} \equiv 1 \pmod p$。
\end{QUOTE}

因为 $ax \equiv 1 \pmod b$;

所以 $ax \equiv a^{b-1} \pmod b$(根据费马小定理);

所以 $x \equiv a^{b-2} \pmod b$;

然后我们就可以用快速幂来求了。

代码:

\begin{cppcode}
#define ll long long
inline ll poW(ll a, ll b) {
  long long ans = 1;
  a %= p;
  while (b) {
    if (b & 1) ans = ((ans * a) % p + p) % p;
    a = (a * a) % p;
    b >>= 1;
  }
  return ans % p;
}
\end{cppcode}

\subsubsection{线性求逆元}

但是如果要求的很多,以上两种方法就显得慢了,很有可能超时,所以下面来讲一下如何线性求逆元。

首先,很显然的 $1^{-1} \equiv 1 \pmod p$;

然后,设 $p=ki+j,j<i,1<i<p$,再放到 $\mod p$ 意义下就会得到:$ki+j \equiv 0 \pmod p$;

两边同时乘 $i^{-1},j^{-1}$:

$kj^{-1}+i^{-1} \equiv 0 \pmod p$;

$i^{-1} \equiv -kj^{-1}+ \pmod p$;

$i^{-1} \equiv -(\frac{p}{i}) (p \mod i)^{-1}$;

然后我们就可以推出逆元了,代码只有一行:

\begin{cppcode}
a[i] = -(p / i) * a[p % i];
\end{cppcode}

但是,有些情况下要避免出现负数,所以我们要改改代码,让它只求正整数:

\begin{cppcode}
a[i] = (p - p / i) * a[p % i] % p;
\end{cppcode}

这就是线性求逆元

\subsection{逆元练习题}

\href{https://www.luogu.org/problemnew/show/P3811}{【模板】乘法逆元}

\href{https://www.luogu.org/problemnew/show/P1082}{同余方程}

\href{https://www.lydsy.com/JudgeOnline/problem.php?id=1965}{\textbackslash{}[AHOI2005\textbackslash{}] 洗牌}

\href{https://www.luogu.org/problemnew/show/P4071}{\textbackslash{}[SDOI2016\textbackslash{}] 排列计数}
