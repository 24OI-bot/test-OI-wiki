
康托展开可以用来求一个 $1\sim n$ 的任意排列的排名。

\subsection{什么是排列的排名?}

把 $1\sim n$ 的所有排列按字典序排序,这个排列的位次就是它的排名。

\subsection{时间复杂度?}

康托展开可以在 $O(n^2)$ 的复杂度内求出一个排列的排名,在用到树状数组优化时可以做到 $O(n\log n)$ 。

\subsection{怎么实现?}

因为排列是按字典序排名的,因此越靠前的数字优先级越高。也就是说如果两个排列的某一位之前的数字都相同,那么如果这一位如果不相同,就按这一位排序。

比如 $4$ 的排列,$[2,3,1,4]<[2,3,4,1]$,因为在第 $3$ 位出现不同,则 $[2,3,1,4]$ 的排名在 $[2,3,4,1]$ 前面。

\subsection{举个栗子}

我们知道长为 $5$ 的排列 $[2,5,3,4,1]$ 大于以 $1$ 为第一位的任何排列,以 $1$ 为第一位的 $5$ 的排列有 $4!$ 种。这是非常好理解的。但是我们对第二位的 $5$ 而言,它大于\textbf{第一位与这个排列相同的,而这一位比 $5$ 小的}所有排列。不过我们要注意的是,这一位不仅要比 $5$ 小,还要满足没有在当前排列的前面出现过,不然统计就重复了。因此这一位为 $1,3$ 或 $4$ ,第一位为 $2$ 的所有排列都比它要小,数量为 $3\times 3!$。

按照这样统计下去,答案就是 $1+4!+3\times 3!+2!+1=46$。注意我们统计的是排名,因此最前面要 $+1$。

注意到我们每次要用到\textbf{当前有多少个小于它的数还没有出现},这里用树状数组统计比它小的数出现过的次数就可以了。

\subsection{逆康托展开}

因为排列的排名和排列是一一对应的,所以康托展开满足双射关系,是可逆的。可以通过类似上面的过程倒推回来。

如果我们知道一个排列的排名,就可以推出这个排列。因为 $4!$ 是严格大于 $3\times 3!+2\times 2!+1\times 1!$ 的,所以可以认为对于长度为 $5$ 的排列,排名 $x$ 除以 $4!$ 向下取整就是有多少个数小于这个排列的第一位。

\subsection{引用上面展开的例子}

首先让 $46-1=45$,$45$ 代表着有多少个排列比这个排列小。$\lfloor\frac {45}{4!}\rfloor=1$,有一个数小于它,所以第一位是 $2$。

此时让排名减去 $1\times 4!$得到$21$,$\lfloor\frac {21}{3!}\rfloor=3$,有 $3$ 个数小于它,去掉已经存在的 $2$,这一位是 $5$。

$21-3\times 3!=3$,$\lfloor\frac {3}{2!}\rfloor=1$,有一个数小于它,那么这一位就是 $3$。

让 $3-1\times 2!=1$,有一个数小于它,这一位是剩下来的第二位,$4$,剩下一位就是 $1$。即 $[2,5,3,4,1]$。

实际上我们得到了形如\textbf{有两个数小于它}这一结论,就知道它是当前第 $3$ 个没有被选上的数,这里也可以用线段树维护,时间复杂度为 $O(n\log n)$。
