
贪心算法顾名思义就是用计算机来模拟一个 “贪心” 的人做出决策的过程。

这个人每一步行动总是按某种指标选取最优的操作,他总是 \textbf{ 只看眼前,并不考虑以后可能造成的影响 } 。

可想而知,并不是所有的时候贪心法都能获得最优解,所以一般使用贪心法的时候,都要确保自己能证明其正确性。

\subsection{常见做法}

在提高组难度以下的题目中,最常见的贪心有两种。一种是:「我们将 XXX 按照某某顺序排序,然后按某种顺序(例如从小到大)处理」。另一种是:「我们每次都取 XXX 中最大 / 小的东西,并更新 XXX」,有时「XXX 中最大 / 小的东西」可以优化,比如用优先队列维护。

为啥分成两种?你可以发现,一种是离线的,一种是在线的。

\subsection{证明方法}

\st{从来都是大胆猜想,从来不会小心求证}

以下套路请按照题目自行斟酌,一般情况下一道题只会用到其中的一种方法来证明。

\begin{enumerate}
\item 运用反证法,如果交换方案中任意两个元素 / 相邻的两个元素后,答案不会变得更好,那么可以发现目前的解已经是最优解了。
\item 运用归纳法,先手算得出边界情况(例如 $n = 1$ )的最优解 $F_1$ ,然后再证明:对于每个 $n$ ,$F_{n+1}$ 都可以由 $F_{n}$ 推导出结果。
\end{enumerate}

\subsection{\href{https://goldimax.github.io/atricle.html?5b82a0a49f54540031c06bd8}{排序法}}

用排序法常见的情况是输入一个包含几个(一般一到两个)权值的数组,通过排序然后遍历模拟计算的方法求出最优值。

有些题的排序方法非常显然,如 \href{https://www.luogu.org/problemnew/show/P1209}{ luogu P1209 } 就是将输入数组差分后排序模拟求值。

然而有些时候很难直接一下子看出排序方法,比如 \href{https://www.luogu.org/problemnew/show/P1080}{ luogu P1080 } 就很容易凭直觉而错误地以 $a$ 或 $b$ 为关键字排序,过样例之后提交就发现 WA 了 QAQ。一个 \st{众所周知的} 常见办法就是尝试交换数组相邻的两个元素来\textbf{推导}出正确的排序方法。我们假设这题输入的俩个数用一个结构体来保存

\begin{cppcode}
struct {
  int a, b;
} v[n];
\end{cppcode}

用 $m$ 表示 $i$ 前面所有的 $a$ 的乘积,那么第 $i$ 个大臣得到的奖赏就是 

$$
\frac{m} {v[i].b}
$$

第 $i + 1$ 个大臣得到的奖赏就是 

$$
\frac{m \cdot v[i].a} {v[i + 1].b}
$$

如果我们交换第 $i$ 个大臣与第 $i + 1$ 个大臣的位置,那么第 $i + 1$ 个大臣得到的奖赏就是 

$$
\frac{m} {v[i + 1].b}
$$

第 $i + 1$ 个大臣得到的奖励就是 

$$
\frac{m \cdot v[i + 1].a} {v[i].b}
$$

如果交前更优当且仅当 

$$
\max (\frac{m} {v[i].b}, \frac{m \times v[i].a} {v[i + 1].b})  < \max (\frac{m} {v[i + 1].b}, \frac{m \times v[i + 1].a} {v[i].b})
$$

提取出相同的 $m$ 并约分得到 

$$
\max(\frac{1} {v[i].b}, \frac{v[i].a} {v[i + 1].b}) < \max(\frac{1} {v[i + 1].b}, \frac{v[i + 1].a} {v[i].b})
$$

然后分式化成整式得到 

$$
\max(v[i + 1].b, v[i].a \times v[i].b) < \max(v[i].b, v[i + 1].a \times v[i + 1].b)
$$

于是我们就成功得到排序函数了!

\begin{cppcode}
struct uv {
  int a, b;
  bool operator<(const uv &x) const {
    return max(x.b, a * b) < max(b, x.a * x.b);
  }
};
\end{cppcode}

\st{看上去是不是很简单呢(这题高精度卡常……)} ,如果看懂了就可以尝试下一道类似的题 \href{https://www.luogu.org/problemnew/show/P2123}{luogu P2123}(请不要翻题解……。

\subsection{后悔法}

\begin{NOTE}{ 例题 [luogu P2949 \[USACO09OPEN\] 工作调度 Work Scheduling ](https://www.luogu.org/problemnew/show/P2949)}{}

\end{NOTE}


贪心思想:

\begin{itemize}
\item \textbf{1} . 先假设每一项工作都做,将各项工作按截止时间排序后入队。      
\item \textbf{2} . 在判断第 i 项工作做与不做时,若其截至时间符合条件,则将其与队中报酬最小的元素比较,若第 i 项工作报酬较高(后悔),则 ans+=a[i].p-q.top()。      
\end{itemize}

\textbf{PS} : 用优先队列(小根堆)来维护队首元素最小。          

\subsubsection{code:}

\begin{cppcode}
#include <algorithm>
#include <cmath>
#include <cstdio>
#include <cstring>
#include <iostream>
#include <queue>
using namespace std;
struct f {
  long long d;
  long long x;
} a[100005];
bool cmp(f A, f B) { return A.d < B.d; }
priority_queue<long long, vector<long long>, greater<long long> > q;
int main() {
  long long n, i, j;
  cin >> n;
  for (i = 1; i <= n; i++) {
    scanf("%d%d", &a[i].d, &a[i].x);
  }
  sort(a + 1, a + n + 1, cmp);
  long long ans = 0;
  for (i = 1; i <= n; i++) {
    if (a[i].d <= q.size()) {
      if (q.top() < a[i].x) {
        ans += a[i].x - q.top();
        q.pop();
        q.push(a[i].x);
      }
    } else {
      ans += a[i].x;
      q.push(a[i].x);
    }
  }
  cout << ans << endl;
  return 0;
}
\end{cppcode}
