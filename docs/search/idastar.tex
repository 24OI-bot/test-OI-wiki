
学习 IDA 之前,请确保您已经学完了 \href{/search/astar}{A} 算法和 \href{/search/iterative}{迭代加深搜索} 。

\subsection{IDA 简介}

IDA,即采用迭代加深的 A 算法。相对于 A 算法,由于 IDA 改成了深度优先的方式,所以 IDA 更实用:

\begin{enumerate}
\item 不需要判重,不需要排序;
\item 空间需求减少。
\end{enumerate}

\textbf{大致框架}(伪代码):

\vskip 0.2 in
\texttt{
Procedure IDA_STAR(StartState)\\Begin\\PathLimit := H(StartState) - 1;\\Succes := False;\\Repeat\\inc(PathLimit);\\StartState.g = 0;\\Push(OpenStack , StartState);\\Repeat\\CurrentState := Pop(OpenStack);\\If Solution(CurrentState) then\\Success = True\\Elseif PathLimit >= CurrentState.g + H(CurrentState) then\\For each Child(CurrentState) do\\Push(OpenStack , Child(CurrentState));\\until Successor empty(OpenStack);\\until Success or ResourceLimtsReached;\\end;}
\vskip 0.2 in

\subsubsection{优点}

\begin{enumerate}
\item 空间开销小,每个深度下实际上是一个深度优先搜索,不过深度有限制,而 DFS 的空间消耗小是众所周知的;
\item 利于深度剪枝。
\end{enumerate}

\subsubsection{缺点}

重复搜索:回溯过程中每次 depth 变大都要再次从头搜索。

\begin{QUOTE}{}{}
其实,前一次搜索跟后一次相差是微不足道的。
\end{QUOTE}

\subsection{例题}

\begin{NOTE}{埃及分数}{}
\textbf{题目描述}
\end{NOTE}


\vskip 0.2 in
\texttt{
在古埃及,人们使用单位分数的和(即 $\frac{1}{a}$,$a$ 是自然数)表示一切有理数。例如,$\frac{2}{3}=\frac{1}{2}+\frac{1}{6}$,但不允许 $\frac{2}{3}=\frac{1}{3}+\frac{1}{3}$,因为在加数中不允许有相同的。 \\\\对于一个分数 $\frac{a}{b}$ ,表示方法有很多种,其中加数少的比加数多的好,如果加数个数相同,则最小的分数越大越好。 例如,$\frac{19}{45}=\frac{1}{5}+\frac{1}{6}+\frac{1}{18}$ 是最优方案。 \\\\输入整数 $a,b$ ($0<a<b<500$),试编程计算最佳表达式。\\\\输入样例:\\\\```text\\495 499\\```\\\\输出样例:\\\\```text\\Case 1: 495/499=1/2+1/5+1/6+1/8+1/3992+1/14970\\```}
\vskip 0.2 in

\textbf{分析}

这道题目理论上可以用回溯法求解,但是\textbf{解答树}会非常 “恐怖”—不仅深度没有明显的上界,而且加数的选择理论上也是无限的。换句话说,如果用宽度优先遍历,连一层都扩展不完,因为每一层都是\textbf{无限大}的。

解决方案是采用迭代加深搜索:从小到大枚举深度上限 $maxd$,每次执行只考虑深度不超过 $maxd$ 的节点。这样,只要解的深度优先,则一定可以在有限时间内枚举到。

深度上限 $maxd$ 还可以用来\textbf{剪枝}。 按照分母递增的顺序来进行扩展,如果扩展到 i 层时,前 $i$ 个分数之和为 $\frac{c}{d}$,而第 $i$ 个分数为 $\frac{1}{e}$ ,则接下来至少还需要 $\frac{\frac{a}{b}-\frac{c}{d}}{\frac{1}{e}}$ 个分数,总和才能达到 $\frac{a}{b}$ 。 例如,当前搜索到 $\frac{19}{45}=\frac{1}{5}+\frac{1}{100}+\cdots$ ,则后面的分数每个最大为 $\frac{1}{101}$,至少需要 $\frac{\frac{19}{45}-\frac{1}{5}}{\frac{1}{101}}=23$ 项总和才能达到 $\frac{19}{45}$ ,因此前 $22$ 次迭代是根本不会考虑这棵子树的。这里的关键在于:可以估计至少还要多少步才能出解。 

注意,这里的估计都是乐观的,因为用了\textbf{至少}这个词。 说得学术一点,设深度上限为 $maxd$,当前结点 $n$ 的深度为 $g(n)$,乐观估价函数为 $h(n)$,则当 $g(n)+h(n)>maxd$ 时应该剪枝。 这样的算法就是 IDA。 当然,在实战中不需要严格地在代码里写出 $g(n)$ 和 $h(n)$ ,只需要像刚才那样设计出乐观估价函数,想清楚在什么情况下不可能在当前的深度限制下出解即可。

\begin{QUOTE}{}{}
如果可以设计出一个乐观估价函数,预测从当前结点至少还需要扩展几层结点才有可能得到解,则迭代加深搜索变成了 IDA 算法。
\end{QUOTE}

\textbf{代码}

\begin{cppcode}
// 埃及分数问题
#include <algorithm>
#include <cassert>
#include <cstdio>
#include <cstring>
#include <iostream>
using namespace std;

int a, b, maxd;

typedef long long LL;

LL gcd(LL a, LL b) { return b == 0 ? a : gcd(b, a % b); }

// 返回满足1/c <= a/b的最小c
inline int get_first(LL a, LL b) { return b / a + 1; }

const int maxn = 100 + 5;

LL v[maxn], ans[maxn];

// 如果当前解v比目前最优解ans更优,更新ans
bool better(int d) {
  for (int i = d; i >= 0; i--)
    if (v[i] != ans[i]) {
      return ans[i] == -1 || v[i] < ans[i];
    }
  return false;
}

// 当前深度为d,分母不能小于from,分数之和恰好为aa/bb
bool dfs(int d, int from, LL aa, LL bb) {
  if (d == maxd) {
    if (bb % aa) return false;  // aa/bb必须是埃及分数
    v[d] = bb / aa;
    if (better(d)) memcpy(ans, v, sizeof(LL) * (d + 1));
    return true;
  }
  bool ok = false;
  from = max(from, get_first(aa, bb));  // 枚举的起点
  for (int i = from;; i++) {
    // 剪枝:如果剩下的maxd+1-d个分数全部都是1/i,加起来仍然不超过aa/bb,则无解
    if (bb * (maxd + 1 - d) <= i * aa) break;
    v[d] = i;
    // 计算aa/bb - 1/i,设结果为a2/b2
    LL b2 = bb * i;
    LL a2 = aa * i - bb;
    LL g = gcd(a2, b2);  // 以便约分
    if (dfs(d + 1, i + 1, a2 / g, b2 / g)) ok = true;
  }
  return ok;
}

int main() {
  int kase = 0;
  while (cin >> a >> b) {
    int ok = 0;
    for (maxd = 1; maxd <= 100; maxd++) {
      memset(ans, -1, sizeof(ans));
      if (dfs(0, get_first(a, b), a, b)) {
        ok = 1;
        break;
      }
    }
    cout << "Case " << ++kase << ": ";
    if (ok) {
      cout << a << "/" << b << "=";
      for (int i = 0; i < maxd; i++) cout << "1/" << ans[i] << "+";
      cout << "1/" << ans[maxd] << "\n";
    } else
      cout << "No solution.\n";
  }
  return 0;
}
\end{cppcode}

\subsection{练习题}

\href{https://www.luogu.org/problem/show?pid=uva1343}{旋转游戏 UVa1343}
