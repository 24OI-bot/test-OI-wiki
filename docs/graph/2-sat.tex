
\begin{QUOTE}{}{}
SAT 是适定性( Satisfiability )问题的简称 。一般形式为 k - 适定性问题,简称 k-SAT 。而当 $k>2$ 时该问题为 NP 完全的。所以我们之研究 $k=2$ 的情况。
\end{QUOTE}

\subsection{定义}

2-SAT ,简单的说就是给出 $n$ 个集合,每个集合有两个元素,已知若干个 $<a,b>$ ,表示 $a$ 与 $b$ 矛盾(其中 $a$ 与 $b$ 属于不同的集合)。然后从每个集合选择一个元素,判断能否一共选 $n$ 个两两不矛盾的元素。显然可能有多种选择方案,一般题中只需要求出一种即可。

\subsection{现实意义}

比如邀请人来吃喜酒,夫妻二人必须去一个,然而某些人之间有矛盾(比如 A 先生与 B 女士有矛盾,C 女士不想和 D 先生在一起),那么我们要确定能否避免来人之间没有矛盾,有时需要方案。这是一类生活中常见的问题。

使用布尔方程表示上述问题。设 $a$ 表示 A 先生去参加,那么 B 女士就不能参加($\neg a$);$b$ 表示 C 女士参加,那么 $\neg b$ 也一定成立(D 先生不参加)。总结一下,即 $(a \vee b)$(变量 $a, b$ 至少满足一个) 。 对这些变量关系建有向图,则有:$\neg a\Rightarrow b\wedge\neg b\Rightarrow a$($a$ 不成立则 $b$ 一定成立;同理,$b$ 不成立则 $a$ 一定成立)。建图之后,我们就可以使用缩点算法来求解 2-SAT 问题了。

\subsection{常用解决方法}

\subsubsection{Tarjan \href{/graph/scc}{SCC 缩点}}

算法考究在建图这点,我们举个例子来讲:

假设有 ${a1,a2}$ 和 ${b1,b2}$ 两对,已知 $a1$ 和 $b2$ 间有矛盾,于是为了方案自洽,由于两者中必须选一个,所以我们就要拉两条条有向边 $(a1,b1)$ 和 $(b2,a2)$ 表示选了 $a1$ 则必须选 $b1$ ,选了 $b2$ 则必须选 $a2$ 才能够自洽。

然后通过这样子建边我们跑一遍 Tarjan SCC 判断是否有一个集合中的两个元素在同一个 SCC 中,若有则输出不可能,否则输出方案。构造方案只需要把几个不矛盾的 SCC 拼起来就好了。

输出方案时可以通过变量在图中的拓扑序确定该变量的取值。如果变量 $\neg x$ 的拓扑序在 $x$ 之后,那么取 $x$ 值为真。应用到 Tarjan 算法的缩点,即 $x$ 所在 SCC 编号在 $\neg x$ 之前时,取 $x$ 为真。因为 Tarjan 算法求强连通分量时使用了栈,所以 Tarjan 求得的 SCC 编号相当于反拓扑序。

显然地, 时间复杂度为 $O(n+m)$。

\subsubsection{爆搜}

就是沿着图上一条路径,如果一个点被选择了,那么这条路径以后的点都将被选择,那么,出现不可行的情况就是,存在一个集合中两者都被选择了。

那么,我们只需要枚举一下就可以了,数据不大,答案总是可以出来的。

\paragraph{爆搜模板}

下方代码来自刘汝佳的白书:

\begin{cppcode}
// 来源:白书第 323 页
struct Twosat {
  int n;
  vector<int> g[maxn * 2];
  bool mark[maxn * 2];
  int s[maxn * 2], c;
  bool dfs(int x) {
    if (mark[x ^ 1]) return false;
    if (mark[x]) return true;
    mark[x] = true;
    s[c++] = x;
    for (int i = 0; i < (int)g[x].size(); i++)
      if (!dfs(g[x][i])) return false;
    return true;
  }
  void init(int n) {
    this->n = n;
    for (int i = 0; i < n * 2; i++) g[i].clear();
    memset(mark, 0, sizeof(mark));
  }
  void add_clause(int x, int y) {  // 这个函数随题意变化
    g[x].push_back(y ^ 1);         // 选了 x 就必须选 y^1
    g[y].push_back(x ^ 1);
  }
  bool solve() {
    for (int i = 0; i < n * 2; i += 2)
      if (!mark[i] && !mark[i + 1]) {
        c = 0;
        if (!dfs(i)) {
          while (c > 0) mark[s[--c]] = false;
          if (!dfs(i + 1)) return false;
        }
      }
    return true;
  }
};
\end{cppcode}

\subsection{例题}

\subsubsection{\textbf{HDU3062 \href{http://acm.hdu.edu.cn/showproblem.php?pid=3062}{Party}}}

\begin{QUOTE}{}{}
题面:有 n 对夫妻被邀请参加一个聚会,因为场地的问题,每对夫妻中只有 $1$ 人可以列席。在 $2n$ 个人中,某些人之间有着很大的矛盾(当然夫妻之间是没有矛盾的),有矛盾的 $2$ 个人是不会同时出现在聚会上的。有没有可能会有 $n$ 个人同时列席?
\end{QUOTE}

这是一道多校题,裸的 2-SAT 判断是否有方案,按照我们上面的分析,如果 $a1$ 中的丈夫和 $a2$ 中的妻子不合,我们就把 $a1$ 中的丈夫和 $a2$ 中的丈夫连边,把 $a2$ 中的妻子和 $a1$ 中的妻子连边,然后缩点染色判断即可。

\begin{cppcode}
// 作者:小黑 AWM
#include <algorithm>
#include <cstdio>
#include <cstring>
#include <iostream>
#define maxn 2018
#define maxm 4000400
using namespace std;
int Index, instack[maxn], DFN[maxn], LOW[maxn];
int tot, color[maxn];
int numedge, head[maxn];
struct Edge {
  int nxt, to;
} edge[maxm];
int sta[maxn], top;
int n, m;
void add(int x, int y) {
  edge[++numedge].to = y;
  edge[numedge].nxt = head[x];
  head[x] = numedge;
}
void tarjan(int x) {  // 缩点看不懂请移步强连通分量上面有一个链接可以点。
  sta[++top] = x;
  instack[x] = 1;
  DFN[x] = LOW[x] = ++Index;
  for (int i = head[x]; i; i = edge[i].nxt) {
    int v = edge[i].to;
    if (!DFN[v]) {
      tarjan(v);
      LOW[x] = min(LOW[x], LOW[v]);
    } else if (instack[v])
      LOW[x] = min(LOW[x], DFN[v]);
  }
  if (DFN[x] == LOW[x]) {
    tot++;
    do {
      color[sta[top]] = tot;  // 染色
      instack[sta[top]] = 0;
    } while (sta[top--] != x);
  }
}
bool solve() {
  for (int i = 0; i < 2 * n; i++)
    if (!DFN[i]) tarjan(i);
  for (int i = 0; i < 2 * n; i += 2)
    if (color[i] == color[i + 1]) return 0;
  return 1;
}
void init() {
  top = 0;
  tot = 0;
  Index = 0;
  numedge = 0;
  memset(sta, 0, sizeof(sta));
  memset(DFN, 0, sizeof(DFN));
  memset(instack, 0, sizeof(instack));
  memset(LOW, 0, sizeof(LOW));
  memset(color, 0, sizeof(color));
  memset(head, 0, sizeof(head));
}
int main() {
  while (~scanf("%d%d", &n, &m)) {
    init();
    for (int i = 1; i <= m; i++) {
      int a1, a2, c1, c2;
      scanf("%d%d%d%d", &a1, &a2, &c1, &c2);  // 自己做的时候别用 cin 会被卡
      add(2 * a1 + c1,
          2 * a2 + 1 - c2);  // 我们将 2i+1 表示为第 i 对中的,2i 表示为妻子。
      add(2 * a2 + c2, 2 * a1 + 1 - c1);
    }
    if (solve())
      printf("YES\n");
    else
      printf("NO\n");
  }
  return 0;
}
\end{cppcode}

\subsection{练习题}

HDU1814 \href{http://acm.hdu.edu.cn/showproblem.php?pid=1814}{和平委员会}

POJ3683 \href{http://poj.org/problem?id=3683}{牧师忙碌日}
