
\subsection{在树 / 图上 DFS}

前置知识:\href{/search/dfs}{DFS 基础}

\subsubsection{树上 DFS}

在树上 DFS 是这样的一个过程:先访问根节点,然后分别访问根节点每个儿子的子树。

可以用来求出每个节点的深度、父亲等信息。

\subsubsection{DFS 序列}

DFS 序列是指 DFS 调用过程中访问的节点编号的序列。

我们发现,每个子树都对应 DFS 序列中的连续一段(一段区间)。

\subsubsection{括号序列}

DFS 进入某个节点的时候记录一个左括号 \texttt{(},退出某个节点的啥时候记录一个右括号 \texttt{)}。

每个节点会出现两次。相邻两个节点的深度相差 1。

\subsubsection{二叉树上 DFS}

(图待补)

\paragraph{先序遍历}

先访问根,再访问子节点。

\paragraph{中序遍历}

先访问左子树,再访问根,再访问右子树。

\paragraph{后序遍历}

先访问子节点,再访问根。

已知中序遍历和另外一个可以求第三个。

\subsubsection{一般图上 DFS}

对于非连通图,只能访问到起点所在的连通分量。

对于连通图,DFS 序列通常不唯一。

注:树的 DFS 序列也是不唯一的。

在 DFS 过程中,通过记录每个节点从哪个点访问而来,可以建立一个树结构,称为 DFS 树。 DFS 树是原图的一个生成树。

DFS 树有很多性质,比如用来求 \href{/graph/scc}{强连通分量}

\subsection{BFS}

前置知识:\href{/search/bfs}{BFS 基础}

\subsubsection{树上 BFS}

从树根开始,严格按照层次来访问节点。

BFS 过程中也可以顺便求出各个节点的深度和父亲节点。

\subsubsection{BFS 序列}

类似 BFS 序列,BFS 序列是指在 BFS 过程中访问的节点编号的序列。

\subsubsection{一般图上 BFS}

同样,如果原图不连通,只能访问到起点所在的连通分量。

BFS 序列通常也不唯一。

类似的我们也可以定义 BFS 树:在 BFS 过程中,通过记录每个节点从哪个点访问而来,可以建立一个树结构,即为 BFS 树。
