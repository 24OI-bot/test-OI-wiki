
 \textbf{ 差分约束系统 } 是一种特殊的 $n$ 元一次不等式组,它包含 $n$ 个变量 $x_1,x_2,...,x_n$ 以及 $m$ 个约束条件,每个约束条件是由两个其中的变量做差构成的,形如 $x_i-x_j\leq c_k$ ,其中 $c_k$ 是常数(可以是非负数,也可以是负数)。我们要解决的问题是:求一组解 $x_1=a_1,x_2=a_2,...,x_n=a_n$ ,使得所有的约束条件得到满足,否则判断出无解。

差分约束系统中的每个约束条件 $x_i-x_j\leq c_k$ 都可以变形成 $x_i\leq x_j+c_k$ ,这与单源最短路中的三角形不等式 $dist[y]\leq dist[x]+z$ 非常相似。因此,我们可以把每个变量 $x_i$ 看做图中的一个结点,对于每个约束条件 $x_i-x_j\leq c_k$ ,从结点 $j$ 向结点 $i$ 连一条长度为 $c_k$ 的有向边。 

注意到,如果 $\{a_1,a_2,...,a_n\}$ 是该差分约束系统的一组解,那么对于任意的常数 $d$ , $\{a_1+d,a_2+d,...,a_n+d\}$ 显然也是该差分约束系统的一组解,因为这样做差后 $d$ 刚好被消掉。

设 $dist[0]=0$ 并向每一个点连一条边,跑单源最短路,若图中存在负环,则给定的差分约束系统无解,否则, $x_i=dist[i]$ 为该差分约束系统的一组解。

一般使用 Bellman-Ford 或队列优化的 Bellman-Ford(俗称 SPFA,在某些随机图跑得很快) 判断图中是否存在负环,最坏时间复杂度为 $O(nm)$ 。 

\subsection{常用变形技巧}

\subsubsection{例题 \href{https://www.luogu.org/problemnew/show/P1993}{ luogu P1993 小 K 的农场 }}

题目大意:求解差分约束系统,有 $m$ 条约束条件, 每条都为形如 $x_a-x_b\geq c_k$ , $x_a-x_b\leq c_k$ 或 $x_a=x_b$ 的形式,判断该差分约束系统有没有解。

\begin{tabular}{ccc}
\hline
题意& 转化& 连边\\x_a - x_b \geq c& x_b - x_a \leq -c& add(a, b, -c);\\x_a - x_b \leq c& x_a - x_b \leq c& add(b, a, c);\\x_a = x_b& x_a - x_b \leq 0, \space x_b - x_a \leq 0& add(b, a, 0), add(a, b, 0);\\\hline
\end{tabular}

跑判断负环,如果不存在负环,输出 \texttt{Yes} ,否则输出 \texttt{No}。

给出一种用 DFS-SPFA 实现的判负环(时间复杂度极度不稳定):

\begin{cppcode}
#include <algorithm>
#include <cstdio>
#include <cstring>
using namespace std;
const int maxn = 400010;
int n, m, op, u, v, we, cur, h[maxn], nxt[maxn], p[maxn], w[maxn], dist[maxn];
bool tf[maxn], ans;
inline void add_edge(int x, int y, int z) {
  cur++;
  nxt[cur] = h[x];
  h[x] = cur;
  p[cur] = y;
  w[cur] = z;
}
void dfs(int x) {
  tf[x] = true;
  for (int j = h[x]; j != -1; j = nxt[j])
    if (dist[p[j]] > dist[x] + w[j]) {
      if (tf[p[j]] || ans) {
        ans = 1;
        break;
      }
      dist[p[j]] = dist[x] + w[j];
      dfs(p[j]);
    }
  tf[x] = false;
}
int main() {
  cur = 0;
  ans = false;
  memset(h, -1, sizeof h);
  memset(dist, 127, sizeof dist);
  scanf("%d%d", &n, &m);
  while (m--) {
    scanf("%d%d%d", &op, &u, &v);
    if (op == 1)
      scanf("%d", &we), add_edge(u, v, -we);
    else if (op == 2)
      scanf("%d", &we), add_edge(v, u, we);
    else if (op == 3)
      add_edge(u, v, 0), add_edge(v, u, 0);
  }
  for (int i = 1; i <= n; i++) {
    dfs(i);
    if (ans) break;
  }
  if (ans)
    printf("No\n");
  else
    printf("Yes\n");
  return 0;
}
\end{cppcode}

\subsubsection{例题 \href{https://www.luogu.org/problemnew/show/P4926}{P4926 \textbackslash{}[1007\textbackslash{}] 倍杀测量者}}

不考虑二分等其他的东西,这里只论述差分系统 $\frac{x_i}{x_j}\leq c_k$  的求解方法。

对每个 $x_i,x_j$ 和 $c_k$ 取一个 $\log$ 就可以把乘法变成加法运算,即 $\log x_i-\log x_j \leq \log c_k$  ,这样就可以用差分约束解决了。

\subsection{Bellman-Ford 判负环代码实现}

下面是用 Bellman-Ford 算法判断图中是否存在负环的代码实现,请在调用前先保证图是联通的。

\begin{cppcode}
bool Bellman_Ford() {
  for (int i = 0; i < n; i++) {
    bool jud = false;
    for (int j = 1; j <= n; j++)
      for (int k = h[j]; ~k; k = nxt[k])
        if (dist[j] > dist[p[k]] + w[k])
          dist[j] = dist[p[k]] + w[k], jud = true;
    if (!jud) break;
  }
  for (int i = 1; i <= n; i++)
    for (int j = h[i]; ~j; j = nxt[j])
      if (dist[i] > dist[p[j]] + w[j]) return false;
  return true;
}
\end{cppcode}

\subsection{习题}

\href{https://www.lydsy.com/JudgeOnline/problem.php?id=1715}{ bzoj 1715: \textbackslash{}[Usaco2006 Dec\textbackslash{}] Wormholes 虫洞 } 

\href{https://www.lydsy.com/JudgeOnline/problem.php?id=2330}{ bzoj 2330: \textbackslash{}[SCOI2011\textbackslash{}] 糖果 }

\href{http://poj.org/problem?id=1364}{ POJ 1364 King }

\href{http://poj.org/problem?id=2983}{ POJ 2983 Is the Information Reliable? }
