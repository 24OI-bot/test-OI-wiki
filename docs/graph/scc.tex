
\subsection{简介}

在阅读下列内容之前,请务必了解 \href{/graph/basic}{图论基础} 部分。

强连通的定义是:有向图 G 强连通是指,G 中任意两个结点连通。

强连通分量(Strongly Connected Components,SCC)的定义是:极大的强连通子图。

这里想要介绍的是如何来求强连通分量。

\subsection{Tarjan 算法}

Robert E. Tarjan (1948\textasciitilde{}) 美国人。

Tarjan 发明了很多很有用的东西,下到 NOIP 上到 CTSC 难度的都有。

【举例子:Tarjan 算法,并查集,Splay 树,Tarjan 离线求 lca(Lowest Common Ancestor,最近公共祖先)等等】

我们这里要介绍的是图论中的 Tarjan 算法,用来处理各种连通性相关的问题。

\subsubsection{定义}

方便起见,我们先定义一些东西。

\texttt{dfn[x]}:结点 x 第一次被访问的时间戳 (dfs number)

\texttt{low[x]}:结点 x 所能访问到的点的 dfn 值的最小值

这里的树指的是 DFS 树

所有结点按 dfn 排序即可得 dfs 序列

\subsubsection{DFS 树的性质}

一个结点的子树内结点的 dfn 都大于该结点的 dfn。 

从根开始的一条路径上的 dfn 严格递增。

一棵 DFS 树被构造出来后,考虑图中的非树边。

前向边 (forward edge):祖先→儿子

后向边 (backward edge):儿子→祖先

横叉边 (cross edge):没有祖先—儿子关系的

注意:横叉边只会往 dfn 减小的方向连接

注意:在无向图中,没有横叉边(为什么?)

\subsubsection{实现}

\begin{cppcode}
dfs(x) {
  dfn[x] = low[x] = ++index;
  S.push(x);
  instack[x] = true;
    for
      each edge(x, y) {
        if (!dfn[y]) {
          dfs(y);
          low[x] = min(low[x], low[y]);
        } else if (instack[y]) {
          low[x] = min(low[x], dfn[y]);
        }
      }
    if (dfn[x] == low[x]) {
      while (1) {
        t = S.pop();
        instack[t] = false;
        if (t == x) break;
      }
    }
}
\end{cppcode}

(转自维基:\href{https://en.wikipedia.org/wiki/Tarjan%27s_strongly_connected_components_algorithm}{} )

时间复杂度 $O(n + m)$

\subsection{Kosaraju 算法}

Kosaraju 算法依靠两次简单的 dfs 实现。

第一次 dfs,选取任意顶点作为起点,遍历所有为访问过的顶点,并在回溯之前给顶点编号,也就是后序遍历。

第二次 dfs,对于反向后的图,以标号最大的顶点作为起点开始 dfs。这样遍历到的顶点集合就是一个强连通分量。对于所有未访问过的结点,选取标号最大的,重复上述过程。

两次 dfs 结束后,强连通分量就找出来了,Kosaraju 算法的时间复杂度为 $O(n+m)$

\subsubsection{实现}

\begin{cppcode}
// g 是原图,g2 是反图

void dfs1(int u) {
  vis[u] = true;
  for (int v : g[u])
    if (!vis[v]) dfs1(v);
  s.push_back(v);
}

void dfs2(int u) {
  color[u] = sccCnt;
  for (int v : g2[u])
    if (!color[v]) dfs2(v);
}

void kosaraju() {
  sccCnt = 0;
  for (int i = 1; i <= n; ++i)
    if (!vis[i]) dfs1(i);
  for (int i = n; i >= 1; --i)
    if (!color[s[i]]) {
      ++sccCnt;
      dfs2(s[i])
    }
}
\end{cppcode}

\subsection{Garbow 算法}

\subsection{应用}

我们可以将一张图的每个强连通分量都缩成一个点。

然后这张图会变成一个 DAG(为什么?)。

DAG 好啊,能拓扑排序了就能做很多事情了。

举个简单的例子,求一条路径,可以经过重复结点,要求经过的不同结点数量最多。

\subsection{推荐题目}

\href{https://www.lydsy.com/JudgeOnline/problem.php?id=1051}{USACO Fall/HAOI 2006 受欢迎的牛}

\href{http://poj.org/problem?id=1236}{POJ1236 Network of Schools}
