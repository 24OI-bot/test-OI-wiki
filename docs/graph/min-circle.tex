
\subsection{问题}

给出一个图,问其中的有 $n$ 个节点构成的边权和最小的环 $(n\ge 3)$ 是多大。

\subsubsection{暴力解法}

设 $u$ 和 $v$ 之间有一条边长为 $w$ 的边,$dis(u,v)$ 表示删除 $u$ 和 $v$ 之间的连边之后,$u$ 和 $v$ 之间的最短路。

那么最小环是 $dis(u,v)+w$。

总时间复杂度 $O(n^2m)$。

\subsubsection{Dijkstra}

枚举所有边,每一次求删除一条边之后对这条边的起点跑一次 Dijkstra,道理同上。

时间复杂度 $O(m(n+m)logn)$。

\subsubsection{Floyd}

最小环是自己到自己的距离,所以我们强迫最短路出去跑一遍就行了。

怎么强迫?

对于所有的 $i$,使它自己到自己的距离为 $\infty$,也就是

\begin{cppcode}
dis[i][i] = (1 << 30);
\end{cppcode}

然后利用 Floyd 的性质,跑完之后对所有的 $dis[i][i]$ 取 $\min$ 即可。

时间复杂度:$O(n^3)$

\subsection{例题}

GDOI2018 Day2 巡逻

给出一张 $n$ 个点的无负权边无向图,要求执行 $Q$ 个操作,三种操作

\begin{enumerate}
\item 删除一个图中的点以及与它有关的边
\item 恢复一个被删除点以及与它有关的边
\item 询问点 $x$ 所在的最小环大小
\end{enumerate}

对于 50\% 的数据,有 $N,Q \le 100$

对于每一个点 $x$ 所在的简单环,都存在两条与 $x$ 相邻的边,删去其中的任意一条,简单环将变为简单路径。

那么枚举所有与 $x$ 相邻的边,每次删去其中一条,然后跑一次 Dijkstra。

或者直接对每次询问跑一遍 Floyd 求最小环,$O(qn^3)$

对于 100\% 的数据,有 $N,Q \le 400$

还是利用 Floyd 求最小环的算法。

若没有删除,删去询问点将简单环裂开成为一条简单路。

然而第二步的求解改用 Floyd 来得出。

那么答案就是要求出不经过询问点 $x$ 的情况下任意两点之间的距离。

怎么在线?

强行离线,利用离线的方法来避免删除操作。

将询问按照时间顺序排列,对这些询问建立一个线段树。

每个点的出现时间覆盖所有除去询问该点的时刻外的所有询问,假设一个点被询问 $x$ 次,则它的出现时间可以视为 $x + 1$ 段区间,插入到线段树上。

完成之后遍历一遍整棵线段树,在经过一个点时存储一个 Floyd 数组的备份,然后加入被插入在这个区间上的所有点,在离开时利用备份数组退回去即可。
