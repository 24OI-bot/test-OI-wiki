
\textbf{图论(graph theory)} 是数学的一个分支,它以 \textbf{图} 为研究的对象。

图论本身是应用数学的一部分,历史上图论曾经被很多数学家各自独立建立过。关于图论的最早文字记载最早出现在欧拉 1736 年的论著中,也就是著名的柯尼斯堡(Konigsberg)问题(七桥问题)。

\subsection{图的定义}

一个图 $G$ 是一个二元组,即序偶 $\langle V,E\rangle$,或记作 $G= \langle V,E\rangle$,其中 $V$ 是有限非空集合,称为 $G$ 的顶点集, $V$ 中的元素称为顶点或结点; $E$ 称为 $G$ 的边的集合, $\forall e_i \in E$ ,都有 $V$ 中的结点与之对应,称 $e_i$ 为 $G$ 的边。

简单来说,就是图 $G$ 就是一个结点的集合 $V$ 和边的集合 $E$,其中任意一条边都可以表示为两个结点之间的关系。若 $e_i\in E$ 表示为 $\langle u,v\rangle$ ,则有 $u\in V , v\in V$。

\subsection{有向边和无向边}

以上定义的结点对 \textbf{可以是有序的,也可以是无序的} 。如果边 $e_i$ 和结点无序对 $(u,v)$ 相对应,则称 $e_i$ 为无向边,记作 $e_i=(u,v)$,称 $u,v$ 为边 $e_i$ 的两个端点。

如果边 $e_i$ 和结点有序对 $\langle u,v\rangle$ 相对应,则称 $e_i$ 为有向边,记为 $e_i= \langle u,v\rangle$,称 $u$ 为边 $e_i$ 的 \textbf{始点} ,$v$ 为该边的终点。

简单来说,如果边对结点的关系是双向的,那么这条边是无向边;如果是单向的,那么这条边是有向边。

\subsection{图的基本概念}

无向图:每条边都是无向边的图。

有向图:每条边都是有向边的图。

混合图:在一个图中,有些边是有向边,另一些边是无向边,则该图为混合图。

有限图:一个图的点集和边集都是有穷集的图。

零图:边集为空集的图。

平凡图:仅有一个结点而没有边构成的图。

关联:若有 $e_i=(u,v)$ 且 $e_i\in E$ ,则称 $u$ 是和 $v$ 相关联的。

孤立点:无边关联的点。

自环:若一条边所关联的两个结点重合,则称此边为自环。

邻接:关联于同一条边的两个点 $u$ 和 $v$ 称为邻接的;关联于同一个点的两条边 $e_1$ 和 $e_2$ 是邻接的(或相邻的)。

\subsection{结点的度数}

设图 $G= \langle V,E\rangle$ 为一个有向图, $v\in V$ ,关联于结点 $v$ 的 \textbf{边} 的条数,称为点 $v$ 的度数,记作 $deg(v)$ 。

注意:一个自环为它的端点增加 2 度。

当图 $G= \langle V,E\rangle$ 为一个有向图, $v\in V$ ,称以 $v$ 作为始点的边数之和称为结点 $v$ 的出度,记为 $deg^{+} (v)$。将以 $v$ 作为终点的边数之和称为结点 $v$ 的入度,记为 $deg^{-} (v)$ 。称以 $v$ 作为端点的边数之和为结点 $v$ 的度数或度,记为 $deg(v)$ 。

显然, $\forall v\in V,deg(v)=deg^{+} (v)+deg^{-} (v)$ 。

\subsubsection{定理 1}

$\sum_{v\in V} deg(v)=2\times |E|$ 

推论:在任意图中,度数为奇数的点必然有偶数个。

\subsubsection{定理 2}

$\sum_{v\in V} deg^{+} (v)=\sum_{v\in V} deg^{-} (v)=|E|$

即所有点入度之和等于出度之和。

\subsection{子图的概念}

设有图 $G= \langle V,E\rangle$ 和图 $G'= \langle V',E'\rangle$ 。

如果 $V'\subseteq V,E'\subseteq E$,则称 $G'$ 是 $G$ 的子图,记作 $G'\subseteq G$。

如果 $G'\subsetneqq G$,即 $V'\subset V$ 或 $E'\subset E$ , 则称 $G'$ 是 $G$ 的真子图,记作 $G'\subset G$。

如果 $V'=V,E'\subseteq E$,则称 $G'$ 是 $G$ 的生成子图。

如果 $V''\subseteq V$ 且 $V'' \neq \varnothing$ ,以 $V''$ 为结点集,以两端点均在 $V''$ 中的边为边集的 $G$ 的子图,称为 $V''$ 导出的 $G$ 的子图,简称为 $V''$ 的导出子图。

如果 $G''= \langle V'',E''\rangle$ 使得 $E''=E-E'$ , 且 $V''$ 中仅包含 $E''$ 中的边所关联的结点,则称 $G''$ 是子图 $G'$ 相对于原图 $G$ 的补图。

\subsection{特殊的图}

树:边数比结点数少一的连通图。更多内容,详见 \href{/graph/tree-basic/}{树相关基础}。

森林:由 $m$ 棵( $m\ge 0$ )互不相交的树组成的图。

基环树:边数和点数相等的连通图。

仙人掌:每个结点至多在一个简单环上的图。

在无向图中,关联一对顶点的边多于 1 条,则称这些边为重边(平行边),重边的条数称为重数。

简单图:不含重边和自环的图。

多重图:含重边的图。

完全图:每对不同的顶点之间都恰连有一条边相连的简单无向图。容易证明, $n$ 个顶点的完全图有 $\frac{n\times (n-1)}{2}$ 条边。

竞赛图:通过在完全图中为每条边分配方向而获得的有向图。

\subsection{参考资料}

离散数学(修订版), 田文成 周禄新 编著, 天津文学出版社, P184-187
