
\subsection{树的重心}

\subsubsection{定义}

以这个点为根,那么所有的子树(不算整个树自身)的大小都不超过整个树大小的一半。

找到一个点,其所有的子树中最大的子树节点数最少, 那么这个点就是这棵树的重心。

删去重心后,生成的多棵树尽可能平衡。

\subsubsection{性质}

树中所有点到某个点的距离和中,到重心的距离和是最小的;如果有两个重心,那么他们的距离和一样。

把两个树通过一条边相连得到一个新的树,那么新的树的重心在连接原来两个树的重心的路径上。

把一个树添加或删除一个叶子,那么它的重心最多只移动一条边的距离。

\subsubsection{求法}

树的重心可以通过简单的两次搜索求出。

\begin{enumerate}
\item 第一遍搜索求出每个结点的子结点数量 $sz[u]$
\item 第二遍搜索找出使 $max\{sz[u],n-sz[u]-1\}$ 最小的结点。
\end{enumerate}

实际上这两步操作可以在一次遍历中解决。对结点 u 的每一个儿子 v,递归的处理 v,求出 sz[v],然后判断是否是结点数最多的子树,处理完所有子结点后,判断 u 是否为重心。

(代码来自叉姐)

\begin{cppcode}
struct CenterTree {
  int n;
  int ans;
  int siz;
  int son[maxn];
  void dfs(int u, int pa) {
    son[u] = 1;
    int res = 0;
    for (int i = head[u]; i != -1; i = edges[i].next) {
      int v = edges[i].to;
      if (v == pa) continue;
      if (vis[v]) continue;
      dfs(v, u);
      son[u] += son[v];
      res = max(res, son[v] - 1);
    }
    res = max(res, n - son[u]);
    if (res < siz) {
      ans = u;
      siz = res;
    }
  }
  int getCenter(int x) {
    ans = 0;
    siz = INF;
    dfs(x, -1);
    return ans;
  }
}
\end{cppcode}

\subsection{参考}

\href{http://fanhq666.blog.163.com/blog/static/81943426201172472943638/}{}

\href{https://www.cnblogs.com/zinthos/p/3899075.html}{}
