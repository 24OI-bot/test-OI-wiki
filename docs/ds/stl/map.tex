
\subsubsection{\texttt{map} 是啥鬼?}

\texttt{map} 是利用红黑树实现的。

当你在写程序的时候,可能需要存储一些信息,例如存储学生姓名对应的分数,例如:\texttt{Tom 0},\texttt{Bob 100},\texttt{Alan 100}。

但是由于数组下标只能为非负整数,所以无法用姓名来存储,这个时候最简单的办法就是使用 STL 的 \texttt{map} 了!

\texttt{map} 可任意类型为下标(在 \texttt{map} 中叫做 \texttt{key},也就是索引),下面是 \texttt{map} 的模型:

\begin{cppcode}
map<类型名, 类型名> 你想给map起的名字
\end{cppcode}

其中两个类型名第一个是 \texttt{key}(索引,可以理解为数组的下标),第二个是 \texttt{value}(对应的元素)。例如上面的例子,我们可以这样的存储:

\begin{cppcode}
map<string, int> mp
\end{cppcode}

是不是感觉很神奇?

\subsubsection{\texttt{map}  具体怎么使用?}

\begin{itemize}
\item \texttt{map} 添加元素
\end{itemize}

\begin{enumerate}
\item 直接存,例如 \texttt{mp["Tom"]=0}
\item 通过插入,例如 \texttt{mp.insert(pair<string,int>("Alan",100));}
\item 初始化( C++11 及以上)和数组差不多:
\end{enumerate}

\begin{cppcode}
map<string, int> mp = {{"Tom", 0}, {"Bob", "100"}, {"Alan", 100}};
\end{cppcode}

\begin{itemize}
\item \texttt{map} 查找删除元素
\end{itemize}

\begin{enumerate}
\item 在你知道查找元素是啥的时候直接来就可以了,例如:\texttt{int grade=mp["Tom"]}
\item 如果你知道了元素的下标,但是想知道这个元素是否已经存在 \texttt{map} 中,可以使用 \texttt{find} 函数。
\end{enumerate}

格式:\texttt{if(mp.find()==mp.end())},意思是是否返回的是 \texttt{map} 的末尾,因为 \texttt{map} 如果没有查找到元素,迭代器会返回末尾。

其中 \texttt{mp.end()} 返回指向 map 尾部的迭代器, 另外 也可以用 \texttt{mp.count(__key) != 0} 来判断

\begin{enumerate}
\item 如果你想知道 map 里全部的元素,那么最正确的做法使用迭代器了,如果你还不会,请查阅之前文章中的迭代器。
\end{enumerate}

\begin{cppcode}
for (iter = mp.begin(); iter != mp.end(); iter++)
  cout << iter->first << " " << iter->second << endl;
\end{cppcode}

其中 \texttt{mp.begin()} 返回指向 \texttt{map} 头部的迭代器

当然,如果使用 C++11 (及以上)你还可以使用 C++11 的新特性 ,如下

\begin{cppcode}
for (auto &i : mp) {
  printf("Key : %d, Value : %d\n", i.first, i.second);
}
\end{cppcode}

\texttt{iter->first} 是 \texttt{key} 索引,例如 \texttt{Tom},而 \texttt{iter->second} 是 \texttt{value}。

如果你想删除 \texttt{Tom} 这个元素,则可以利用 \texttt{find} 函数找到 \texttt{Tom} ,然后再 \texttt{erase} 如下

\begin{cppcode}
map<string, int>::iterator it;
it = mp.find("Tom");
mp.erase(it)
\end{cppcode}

如果你想清空所有的元素,可以直接 \texttt{mp.clear()}

\begin{itemize}
\item 其他
\end{itemize}

我们刚才介绍了最常用的,下面是其他比较常用的:

\begin{itemize}
\item \texttt{count()} 返回指定元素出现的次数 ,例如 \texttt{mp.count()}
\item \texttt{swap()} 可以交换两个 \texttt{map} ,例如 \texttt{swap(m1,m2)}
\item \texttt{size()} 返回 \texttt{map} 中元素的个数
\item \texttt{empty()} 如果 \texttt{map} 为空则返回 \texttt{true},例如 \texttt{mp.empty()}。
\end{itemize}

\subsubsection{\texttt{map} 常数靠得住吗?}

一般情况下是可以的。无论查询,插入,删除的复杂度都是 $O(\log N)$,遍历是 $O(N)$。

不过有的时候不会满足啊!我只想查询元素,插入元素,但是时间不够咋办?请往下看!

\begin{itemize}
\item 由于 NOIP 不资瓷吸氧(开启 O2 优化),所以 NOIP 要注意是否会被卡
\end{itemize}

\subsubsection{更快:基于 \texttt{Hash} 实现的 \texttt{map}!}

\begin{NOTE}{note}{}
C++11 及以后使用 \texttt{std::unordered_map},在 \texttt{<unordered_map>} 头文件中之前的版本可以使用 \texttt{std::tr1::unordered_map},在 \texttt{<tr1/unordered_map>} 头文件中
\end{NOTE}


这个 \texttt{map} 的名字就是 \texttt{unordered_map} 了,它的查询,插入,删除的复杂度几乎是 $O(1)$ 级别(所有的操作几乎和 \texttt{map}一样(注意 \texttt{unordered_map} 用迭代器遍历是无序的)。

但是在最坏情况下(产生大量 hash 冲突时),\texttt{unordered_map}的各项操作的时间复杂度可达$O(n^2)$ 。\href{http://codeforces.com/blog/entry/62393}{ (详情见 Codeforces 上发表的一篇卡 unordered\_map 的文章) } 而且它的遍历速度会很慢,空间占用的会更大。
