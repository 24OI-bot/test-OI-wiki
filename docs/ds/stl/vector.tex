
\section{std :: vector}

\subsection{为什么要用 vector}

作为 OIer ,对程序效率的追求远比对工程级别的稳定性要高得多,而 vector 由于其较静态数组复杂很多的原因,时间效率在大部分情况下都要满慢于静态数组,所以在一般的正常存储数据的时候,我们是不选择 vector 的, 下面给出几个 vector 优秀的特性,在需要用到这些特性的情况下,vector 能给我们带来很大的帮助

\subsubsection{vector 重写了比较运算符}

vector 以字典序为关键字重载了 6 个比较运算符,这使得我们可以方便的判断两个容器是否相等   (复杂度与容器大小成线性关系)

\subsubsection{vector 的内存是动态分配的}

由于其动态分配的特性, 所以在调用内存的常数上在很多情况下是要快于静态数组的。

很多时候我们不能提前开好那么大的空间(eg :预处理 1\textasciitilde{}n 中所有数的约数)我们知道数据总量在空间允许的级别,但是单份数据还可能非常大,这种时候我们就需要 vector 来保证复杂度。

\subsubsection{vector 可以用赋值运算符来进行初始化}

由于 vector 重写了 \texttt{=} 运算符,所以我们可以方便的初始化。

\subsection{vector 的构造函数}

参见如下代码

\begin{cppcode}
void Vector_Constructor_Test() {
  // 1. 创建空vector v0;  常数复杂度
  std::vector<int> v0;
  // 2. 创建一个初始空间为3的vector v1,其元素的默认值是0; 线性复杂度
  std::vector<int> v1(3);
  // 3. 创建一个初始空间为5的vector v2,其元素的默认值是2; 线性复杂度
  std::vector<int> v2(5, 2);
  // 4. 创建一个初始空间为3的vector
  // v3,其元素的默认值是1,并且使用v2的空间配置器 线性复杂度
  std::vector<int> v3(3, 1, v2.get_allocator());
  // 5. 创建一个v2的拷贝vector v4, 其内容元素和v2一样; 线性复杂度
  std::vector<int> v4(v2);
  // 6. 创建一个v4的拷贝vector v5,其内容是v4的[__First, __Last)区间 线性复杂度
  std::vector<int> v5(v4.begin() + 1, v4.begin() + 3);
  // 以下是测试代码,有兴趣的同学可以自己编译运行一下本代码。
  std::cout << "v1 = ";
  std::copy(v1.begin(), v1.end(), std::ostream_iterator<int>(std::cout, " "));
  std::cout << std::endl;
  std::cout << "v2 = ";
  std::copy(v2.begin(), v2.end(), std::ostream_iterator<int>(std::cout, " "));
  std::cout << std::endl;
  std::cout << "v3 = ";
  std::copy(v3.begin(), v3.end(), std::ostream_iterator<int>(std::cout, " "));
  std::cout << std::endl;
  std::cout << "v4 = ";
  std::copy(v4.begin(), v4.end(), std::ostream_iterator<int>(std::cout, " "));
  std::cout << std::endl;
  std::cout << "v5 = ";
  std::copy(v5.begin(), v5.end(), std::ostream_iterator<int>(std::cout, " "));
  std::cout << std::endl;
  // 移动v2到新创建的vector v6;
  std::vector<int> v6(move(v2));
  std::cout << "v6 = ";
  std::copy(v6.begin(), v6.end(), std::ostream_iterator<int>(std::cout, " "));
  std::cout << std::endl;
};
\end{cppcode}

可以利用上述的方法构造一个 vector, 足够我们使用了。

\subsection{vector 元素访问}

vector 提供了如下几种方法进行访问元素

\begin{enumerate}
\item \texttt{at()}
使用方法 :\texttt{v.at(pos)} 返回 vector 中下标为 \texttt{pos} 的引用。如果数组越界抛出 \texttt{std::out_of_range} 类型的异常。
\item \texttt{operator[]}
使用方法 :\texttt{v[pos]} 返回 vector 中下标为 \texttt{pos} 的引用。不执行越界检查。
\item \texttt{front()}
使用方法 :\texttt{v.front()} 返回首元素的引用
\item \texttt{back()}
使用方法 :\texttt{v.back()} 返回末尾元素的引用
\item \texttt{data()}
使用方法 :\texttt{v.data()} 返回指向数组第一个元素的指针。
\end{enumerate}

\subsection{vecort 迭代器}

vector 提供了如下几种迭代器

\begin{enumerate}
\item \texttt{begin() / cbegin()}
返回指向首元素的迭代器,其中 \texttt{*begin = front}
\item \texttt{end() / cend()}
返回指向数组尾端占位符的迭代器,注意是没有元素的。
\item \texttt{rbegin() / rcbegin()}
返回指向逆向数组的首元素的逆向迭代器, 可以理解为正向容器的末元素
\item \texttt{rend() / rcend()}
返回指向逆向数组末元素后一位置的迭代器,对应容器首的前一个位置, 没有元素。
\end{enumerate}

以上列出的迭代器中,含有字符 \texttt{c} 的为只读迭代器,你不能通过只读迭代器去修改 vector 中的元素的值。如果一个 vector 本身就是只读的,那么它的一般迭代器和只读迭代器完全等价。只读迭代器自 C++11 开始支持。

\subsection{vector 容量}

vector 有如下几种返回容量的函数

\begin{enumerate}
\item \texttt{empty()}
返回一个 \texttt{bool} 值,即 \texttt{(v.begin() == v.end())} True 为空,False 为非空
\item \texttt{size()}
返回一个元素数量,即 \texttt{(std :: distance(v.begin(), v.end()))}
\item \texttt{shrink_to_fit()} (C++11)
释放未使用的内存来减少内存使用
\end{enumerate}

此外,还有 \texttt{max_size()}, \texttt{reserve()}, \texttt{capacity()} 等 OIer 很难用到的函数,不做介绍。

\subsection{vector 修改器}

\begin{itemize}
\item \texttt{clear()} 清除所有元素
\item \texttt{insert()} 支持在某个迭代器位置插入元素、可以插入多个\textbf{此操作是与 \texttt{pos} 距离末尾长度成线性而非常数的}
\item \texttt{erase()} 删除某个迭代器或者区间的元素,返回最后被删除的迭代器。
\item \texttt{push_back()} 在末尾插入一个元素。
\item \texttt{pop_back()} 删除末尾元素。
\item \texttt{swap()} 与另一个容器进行交换,此操作是\textbf{常数复杂度}而非线性的。
\end{itemize}

\subsection{vector 特化 \texttt{std::vector<bool>}}

标准库提供对 bool 的 vector 优化,其空间占用与 bitset 一样,每个 \texttt{bool} 只占 1bit,且支持动态内存

注意,\texttt{vector<bool>}没有 bitset 的位运算重载,所以适用情况与 bitset 并不完全重合,请选择食用
