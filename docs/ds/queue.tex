
队列,英文名是 queue,在 C++ STL 中有 \href{https://en.cppreference.com/w/cpp/container/queue}{std::queue} 和 \href{https://en.cppreference.com/w/cpp/container/priority_queue}{std::priority\_queue}。

先进入队列的元素一定先出队列,因此队列通常也被称为先进先出(first in first out)表,简称 FIFO 表。

注:\texttt{std::stack} 和 \texttt{std::queue} 都是容器适配器,默认底层容器为 \texttt{std::deque}(双端队列)。

通常用一个数组模拟一个队列,用两个指针:front 和 rear 分别表示队列头部和尾部。

在入队的时候将 rear 后移,在出队的时候将 front 后移。

这样会导致一个问题:随着时间的推移,整个队列会向数组的尾部移动,一旦到达数组的最末端,即使数组的前端还有空闲位置,再进行入队操作也会导致溢出。(这种数组上实际有空闲位置而发生了上溢的现象称为是 “假溢出”。

解决假溢出的办法是采用循环的方式来组织存放队列元素的数组,即将数组下标为 0 的位置看做是最后一个位置的后继。(\texttt{x} 的后继为 \texttt{(x + 1) % Size})。这样就形成了循环队列。
