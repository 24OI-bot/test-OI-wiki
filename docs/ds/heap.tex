
\subsection{堆}

堆是一种数据结构,维护一个数的集合(或者,一个支持比较的元素的集合)。

主要功能有:insert, getmin, deletemin, decreasekey。

注意:简单起见,我们这里讨论的都是维护最小值的堆,也叫小根堆,与之相对的叫做大根堆。

一些功能强大的堆还能(高效地)支持 merge 等操作。

一些功能更强大的堆还支持可持久化,也就是对任意历史版本进行查询或者操作,产生新的版本。

\subsection{堆的分类}

一个有趣的事实是,这些堆都是用基于树的数据结构实现的。

在 NOIP 中,我们只要求一个能支持主要操作的堆就行,也就是二叉堆。

\begin{itemize}
\item 二叉堆 {\em (binary heap) }
\end{itemize}

最基础的堆,不支持 merge 和可持久化,所有操作的复杂度都是 $O(\log n)$ 的。

\begin{itemize}
\item 二项堆 {\em (binomial heap) }
\end{itemize}

支持 merge 的堆,(也能可持久化),所有操作的复杂度都是 $O(\log n)$。

\begin{itemize}
\item Fib 堆 {\em (Fibonacci heap) }
\end{itemize}

除了不能可持久化,支持全部功能,而且除了 deletemin 以外都是均摊 $O(1)$ 的。

\subsection{二叉堆}

\subsubsection{结构}

从二叉堆的结构说起,它是一棵二叉树,并且是完全二叉树,每个结点中存有一个元素(或者说,有个权值)。

堆性质:父亲的权值不大于儿子的权值 (小根堆)。

由堆性质,树根存的是最小值 (getmin 操作就解决了)。

\subsubsection{插入操作}

首先,要保证插入后也是一棵完全二叉树。

最简单的方法就是,最下一层最右边的叶子之后插入。

如果最下一层已满,就新增一层。

插入之后可能会不满足堆性质?

向上调整:如果这个结点的权值大于它父亲的权值,就交换,重复此过程直到不满足或者到根。

可以证明,插入之后向上调整后,没有其他结点会不满足堆性质。

向上调整的时间复杂度是 $O(\log n)$ 的。

\subsubsection{删除操作}

删除根结点。

如果直接删除,则变成了两个堆,难以处理。

所以不妨考虑插入操作的逆过程,设法将根结点移到最后一个结点,然后直接删掉。

然而实际上不好做,我们通常采用的方法是,把根结点和最后一个结点直接交换。

于是直接删掉(在最后一个结点处的)根结点,但是新的根结点可能不满足堆性质……

向下调整:在该结点的所有儿子中,找一个最小的,与该结点交换,重复此过程直到底层。

可以证明,删除并向下调整后,没有其他结点不满足堆性质。

时间复杂度 $O(\log n)$。

\subsubsection{减小某个点的权值}

很显然,直接修改后,向上调整一次即可,时间复杂度为 $O(\log n)$。

\subsubsection{实现}

我们发现,上面介绍的几种操作主要依赖于两个核心:向上调整和向下调整。

(伪代码)

\begin{cppcode}
up(x) {
  while (x > 1 && h[x] > h[x / 2]) {
    swap(h[x], h[x / 2]);
    x /= 2;
  }
}
down(x) {
  while (x * 2 <= n) {
    t = x * 2;
    if (t + 1 <= n && h[t + 1] < h[t]) t++;
    if (h[t] >= h[x]) break;
    swap(h[x], h[t]);
    x = t;
  }
}
\end{cppcode}

\subsubsection{建堆}

考虑这么一个问题,从一个空的堆开始,插入 $n$ 个元素,不在乎顺序。

直接一个一个插入需要 $O(n \log n)$ 的时间,有没有更好的方法?

\paragraph{方法一:使用 decreasekey(即,向上调整)}

从根开始,按 BFS 序进行.

\vskip 0.2 in
\texttt{
build_heap_1() {\\	for (i = 1; i <= n; i++) up(i);\\}}
\vskip 0.2 in

为啥这么做:对于第 $k$ 层的结点,向上调整的复杂度为 $O(k)$ 而不是 $O(\log n)$。

总复杂度:$\log 1 + \log 2 + \cdots + \log n = \Theta(n \log n)$。

(在「基于比较的排序」中证明过)

\paragraph{方法二:使用向下调整}

这时换一种思路,从叶子开始,逐个向下调整

\vskip 0.2 in
\texttt{
build_heap_2() {\\	for (i = n; i >= 1; i--) down(i);\\}}
\vskip 0.2 in

换一种理解方法,每次「合并」两个已经调整好的堆,这说明了正确性。

注意到向下调整的复杂度,为 $O(\log n - k)$。

$$
\begin{aligned}
总复杂度 & = n \log n - \log 1 - \log 2 - \cdots - \log n \\\\
& \leq n \log n - 0 \times 2^0 - 1 \times 2^1 -\cdots - (\log n - 1) \times \frac{n}{2} \\\\
& = n \log n - (n-1) - (n-2) - (n-4) - \cdots - (n-\frac{n}{2}) \\\\
& = n \log n - n \log n + 1 + 2 + 4 + \cdots + \frac{n}{2} \\\\
& = n - 1 \\\\ &  = O(n)
\end{aligned}
$$

之所以能 $O(n)$ 建堆,是因为堆性质很弱,二叉堆并不是唯一的。

要是像排序那样的强条件就难说了。
