
WBLT,全称 Weight Balanced Leafy Tree,一种不常见的平衡树写法,但是具有常数较小,可以当做可并堆使用的优点。

类似于 WBL(weight-balanced trees,加权平衡树),WBLT 体现了 leafy 的性质, 即节点多,怎么多呢?

对于 n 个数,不同于 treap 等,WBLT 会建立 2n 个节点,每个节点的权值为其右儿子的权值,且右儿子的权值大于等于左儿子

每次插入,类似于堆,逐次向下交换并向上 pushup 更新即可,删除也是同理

当然,如果输入数据递增或递减,WBLT 会退化成链状,于是我们采用旋转来维护平衡。

因为 WBLT 同时满足堆的性质,我们可以用它来实现堆和可并堆。

而在旋转的过程中,会产生很多垃圾节点,我们采用垃圾回收的方式就可以回收废弃节点,将建立节点的操作稍作修改即可。

附上普通平衡树代码:

\begin{cppcode}
#include <cstdio>
#include <iostream>

using namespace std;

const int maxn = 400005;

const int ratio = 5;
int n, cnt, fa, root;
int size[maxn], ls[maxn], rs[maxn], val[maxn];

void newnode(int &cur, int v) {
  cur = ++cnt;
  size[cur] = 1;
  val[cur] = v;
}
void copynode(int x, int y) {
  size[x] = size[y];
  ls[x] = ls[y];
  rs[x] = rs[y];
  val[x] = val[y];
}
void merge(int l, int r) {
  size[++cnt] = size[l] + size[r];
  val[cnt] = val[r];
  ls[cnt] = l, rs[cnt] = r;
}
void rotate(int cur, bool flag) {
  if (flag) {
    merge(ls[cur], ls[rs[cur]]);
    ls[cur] = cnt;
    rs[cur] = rs[rs[cur]];
  } else {
    merge(rs[ls[cur]], rs[cur]);
    rs[cur] = cnt;
    ls[cur] = ls[ls[cur]];
  }
}
void maintain(int cur) {
  if (size[ls[cur]] > size[rs[cur]] * ratio)
    rotate(cur, 0);
  else if (size[rs[cur]] > size[ls[cur]] * ratio)
    rotate(cur, 1);
  if (size[ls[cur]] > size[rs[cur]] * ratio)
    rotate(ls[cur], 1), rotate(cur, 0);
  else if (size[rs[cur]] > size[ls[cur]] * ratio)
    rotate(rs[cur], 0), rotate(cur, 1);
}
void pushup(int cur) {
  if (!size[ls[cur]]) return;
  size[cur] = size[ls[cur]] + size[rs[cur]];
  val[cur] = val[rs[cur]];
}
void insert(int cur, int x) {
  if (size[cur] == 1) {
    newnode(ls[cur], min(x, val[cur]));
    newnode(rs[cur], max(x, val[cur]));
    pushup(cur);
    return;
  }
  maintain(cur);
  insert(x > val[ls[cur]] ? rs[cur] : ls[cur], x);
  pushup(cur);
}
void erase(int cur, int x) {
  if (size[cur] == 1) {
    cur = ls[fa] == cur ? rs[fa] : ls[fa];
    copynode(fa, cur);
    return;
  }
  maintain(cur);
  fa = cur;
  erase(x > val[ls[cur]] ? rs[cur] : ls[cur], x);
  pushup(cur);
}
int find(int cur, int x) {
  if (size[cur] == x) return val[cur];
  maintain(cur);
  if (x > size[ls[cur]]) return find(rs[cur], x - size[ls[cur]]);
  return find(ls[cur], x);
}
int rnk(int cur, int x) {
  if (size[cur] == 1) return 1;
  maintain(cur);  // asdasdasdasd
  if (x > val[ls[cur]]) return rnk(rs[cur], x) + size[ls[cur]];
  return rnk(ls[cur], x);
}
int main() {
  scanf("%d", &n);
  newnode(root, 2147383647);  //使根不改变
  while (n--) {
    int s, a;
    scanf("%d %d", &s, &a);
    if (s == 1) insert(root, a);
    if (s == 2) erase(root, a);
    if (s == 3) printf("%d\n", rnk(root, a));
    if (s == 4) printf("%d\n", find(root, a));
    if (s == 5) printf("%d\n", find(root, rnk(root, a) - 1));
    if (s == 6) printf("%d\n", find(root, rnk(root, a + 1)));
  }
  return 0;
}
\end{cppcode}
