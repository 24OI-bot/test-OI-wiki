
\subsection{哈希表}

哈希表是又称散列表,一种以 "key-value" 形式存储数据的数据结构。所谓以 "key-value" 形式存储数据,是指任意的 key 都唯一对应到内存中的某个位置。只需要输入查找的值 key,就可以快速地找到其对应的 value。可以把哈希表理解为一种高级的数组,这种数组的下标可以是很大的整数,浮点数,字符串甚至结构体。

\subsection{哈希函数}

要让 key 对应到内存中的位置,就要为 key 计算索引,也就是计算这个数据应该放到哪里。这个根据 key 计算索引的函数就叫做哈希函数,也称散列函数。举个例子,比如 key 是一个人的身份证号码,哈希函数就可以是号码的后四位,当然也可以是号码的前四位。生活中常用的 “手机尾号” 也是一种哈希函数。在实际的应用中,key 可能是更复杂的东西,比如浮点数、字符串、结构体等,这时候就要根据具体情况设计合适的哈希函数。哈希函数应当易于计算,并且尽量使计算出来的索引均匀分布。

在 OI 中,最常见的情况应该是 key 为整数的情况。当 key 的范围比较小的时候,可以直接把 key 作为数组的下标,但当 key 的范围比较大,比如以 10\textasciicircum{}9 范围内的整数作为 key 的时候,就需要用到哈希表。一般把 key 模一个较大的质数作为索引,也就是取 $f(x)=x \mod M$ 作为哈希函数。另一种比较常见的情况是 key 为字符串的情况,在 OI 中,一般不直接把字符串作为 key,而是先算出字符串的哈希值,再把其哈希值作为 key 插入到哈希表里。

能为 key 计算索引之后,我们就可以知道每个 value 应该放在哪里了。假设我们用数组 a 存放数据,哈希函数是 f,那键值对 (key,value) 就应该放在 a[f(key)] 上。不论 key 是什么类型,范围有多大,f(key) 都是在可接受范围内的整数,可以作为数组的下标。

\subsection{冲突}

如果对于任意的 key,哈希函数计算出来的索引都不相同,那只用根据索引把 (key,value) 放到对应的位置就行了。但实际上,常常会出现两个不同的 key,他们用哈希函数计算出来的索引是相同的。这时候就需要一些方法来处理冲突。在 OI 中,最常用的方法是拉链法。

\subsubsection{拉链法}

拉链法是在每个存放数据的地方开一个链表,如果有多个 key 索引到同一个地方,只用把他们都放到那个位置的链表里就行了。查询的时候需要把对应位置的链表整个扫一遍,对其中的每个数据比较其 key 与查询的 key 是否一致。如果索引的范围是 1\textasciitilde{}M ,哈希表的大小为 N ,那么一次插入 / 查询需要进行期望 $O(\frac{N}{M})$ 次比较。

\subsection{实现}

\subsubsection{拉链法}

\begin{cppcode}
const int SIZE = 1000000;
const int M = 999997;
struct HashTable {
  struct Node {
    int next, value, key;
  } data[SIZE];
  int head[M], size;
  int f(int key) { return key % M; }
  int get(int key) {
    for (int p = head[f(key)]; p; p = data[p].next)
      if (data[p].key == key) return data[p].value;
    return -1;
  }
  int modify(int key, int value) {
    for (int p = head[f(key)]; p; p = data[p].next)
      if (data[p].key == key) return data[p].value = value;
  }
  int add(int key, int value) {
    if (get(key) != -1) return -1;
    data[++size] = (Node){head[f(key)], value, key};
    head[f(key)] = size;
    return value;
  }
};
\end{cppcode}

\subsection{例题}

\href{https://www.lydsy.com/JudgeOnline/problem.php?id=2761}{\textbackslash{}[JLOI2011\textbackslash{}] 不重复数字}
