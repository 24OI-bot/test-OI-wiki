
\section{并查集}

并查集是一种树形的数据结构,顾名思义,它用于处理一些不交集的\textbf{合并}及\textbf{查询}问题。  

它支持两种操作:

\begin{itemize}
\item 查找 (Find):确定某个元素处于哪个子集;
\item 合并(Union):将两个子集合并成一个集合。
\end{itemize}

\subsection{初始化}

\begin{cppcode}
void makeSet(int size) {
  for (int i = 0; i < size; i++) {
    fa[i] = i;  // i就在它本身的集合里
  }
  return;
}
\end{cppcode}

\subsection{查找}

\begin{NOTE}{举个例子}{}
几个家族进行宴会,但是家族普遍长寿,所以人数众多。由于长时间的分离以及年龄的增长,这些人逐渐忘掉了自己的亲人,只记得自己的爸爸是谁了,而最长者(称为「祖先」)的父亲已经去世,他只知道自己是祖先。为了确定自己是哪个家族,他们想出了一个办法,只要问自己的爸爸是不是祖先,一层一层的向上问,直到问到祖先。如果要判断两人是否在同一家族,只要看两人的祖先是不是同一人就可以了。  
\end{NOTE}


在这样的思想下,并查集的查找算法诞生了。我们可以用代码模拟这个过程。

\begin{cppcode}
int fa[MAXN];  //记录某个人的爸爸是谁,特别规定,祖先的爸爸是他自己
int find(int x)  //寻找x的祖先
{
  if (fa[x] == x)  //如果x是祖先则返回
    return x;
  else
    return find(fa[x]);  //如果不是则x的爸爸问x的爷爷
}
\end{cppcode}

显然这样最终会返回 $x$ 的祖先。

\subsubsection{路径压缩}

这样的确可以达成目的,但是显然效率实在太低。为什么呢?因为我们使用了太多没用的信息,我关心的是我祖先是谁,我爸爸是谁没什么关系,这样一层一层找太浪费时间,不如我直接当祖先的儿子,问一次就可以出结果了。甚至祖先是谁都无所谓,只要这个人可以代表我们家族就能得到想要的效果。\textbf{把在路径上的每个节点都直接连接到根上},这就是路径压缩。 

于是用代码实现它。

\begin{cppcode}
int find(int x) {
  if (x != fa[x])  // x不是自身的父亲,即x不是该集合的代表
    fa[x] = find(fa[x]);  //查找x的祖先直到找到代表,于是顺手路径压缩
  return fa[x];
}
\end{cppcode}

\subsection{合并}

宴会上,一个家族的祖先突然对另一个家族说: 我们两个家族交情这么好,不如合成一家好了。另一个家族也欣然接受了。  

我们之前说过,并不在意祖先究竟是谁,所以只要其中一个祖先变成另一个祖先的儿子就可以了。

\begin{cppcode}
void unionSet(int x, int y)  // x与y所在家族合并
{
  x = find(x);
  y = find(y);
  if (x == y)  //原本就在一个家族里就不管了
    return;
  fa[x] = y;  //把x的祖先变成y的祖先的儿子
}
\end{cppcode}

\subsubsection{启发式合并(按秩合并)}

一个祖先突然抖了个机灵:「你们家族人比较少,搬家到我们家族里比较方便,我们要是搬过去的话太费事了。」  

启发式合并是将深度小的集合合并到深度大的集合(也称为\textbf{按秩合并}),但是笔者认为路径压缩之后它就失去意义了,或者不如按照节点数量合并,这样还可以减少下次路径压缩的工作量。(反正启发式合并用得很少,路径压缩已经够快了。)

\begin{cppcode}
int size[N];  //记录子树的大小
void unionSet(int x, int y) {
  int xx = find(x), yy = find(y);
  if (xx == yy) return;
  if (size[xx] > size[yy])  //保证小的合到大的里
    swap(xx, yy);
  fa[xx] = yy;
  size[yy] += size[xx];
}
\end{cppcode}

\subsection{时间复杂度及空间复杂度}

\subsubsection{时间复杂度}

同时使用路径压缩和启发式合并之后,并查集的每个操作平均时间仅为 $O(\alpha(n))$ ,其中 $\alpha$ 为 \href{https://en.wikipedia.org/wiki/Ackermann_function}{阿克曼函数} 的反函数,其增长极其缓慢,也就是说其平均运行时间可以认为是一个很小的常数。 

\subsubsection{空间复杂度}

显然为 $O(n)$。

\subsection{经典题目}

\href{https://www.lydsy.com/JudgeOnline/problem.php?id=4195}{\textbackslash{}[NOI2015\textbackslash{}] 程序自动分析}

\href{https://www.lydsy.com/JudgeOnline/problem.php?id=1015}{\textbackslash{}[JSOI2008\textbackslash{}] 星球大战}

\href{https://www.luogu.org/problemnew/show/P2024}{\textbackslash{}[NOI2001\textbackslash{}] 食物链}

\href{https://www.luogu.org/problemnew/show/P1196}{\textbackslash{}[NOI2002\textbackslash{}] 银河英雄传说}

\subsection{其他应用}

\href{/graph/mst}{最小生成树} Kruskal 是基于并查集的算法。
