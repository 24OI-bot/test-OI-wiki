
\subsection{\textbackslash{}\_\textbackslash{}\_gnu\_pbds :: priority\_queue}

附 :\href{https://gcc.gnu.org/onlinedocs/libstdc++/ext/pb_ds/pq_performance_tests.html#std_mod1}{官方文档地址——复杂度及常数测试}

\begin{cppcode}
#include <ext/pb_ds/priority_queue.hpp>
using namespace __gnu_pbds;
__gnu_pbds ::priority_queue<T, Compare, Tag, Allocator>
\end{cppcode}

\subsection{模板形参}

\begin{itemize}
\item \texttt{T} : 储存的元素类型
\item \texttt{Compare} : 提供严格的弱序比较类型
\item \texttt{Tag} : 是 \texttt{__gnu_pbds} 提供的不同的五种堆,Tag 参数默认是 \texttt{pairing_heap_tag}
  五种分别是 :
\begin{itemize}
\item \texttt{pairing_heap_tag}:配对堆
官方文档认为在非原生元素 (如自定义结构体 / \texttt{std :: string} / \texttt{pair}) 中,配对堆表现最好
\item \texttt{binary_heap_tag}:二叉堆 
官方文档认为在原生元素中二叉堆表现最好,不过我测试的表现并没有那么好
\item \texttt{binomial_heap_tag}:二项堆
二项堆在合并操作的表现要优于配对堆  但是其取堆顶元素的
\item \texttt{rc_binomial_heap_tag}:冗余计数二项堆
\item \texttt{thin_heap_tag}:除了合并的复杂度都和 Fibonacci 堆一样的一个 tag
\end{itemize}
\item \texttt{Allocator}:空间配置器,由于 OI 中很少出现,故这里不做讲解
\end{itemize}

由于本篇文章只是提供给学习算法竞赛的同学们,故对于后四个 tag 只会简单的介绍复杂度,第一个会介绍成员函数和使用方法。

经作者本机 Core i5@3.1 GHz On macOS 测试堆的基础操作,结合 GNU 官方的复杂度测试,Dijkstra 测试,都表明:

至少对于 OIer 来讲,除了配对堆的其他 4 个 tag 都是鸡肋,要么没用,要么常数大到不如 std 的,且有可能造成 MLE,故这里只推荐用默认的配对堆。同样,配对堆也优于 \texttt{algorithm} 库中的 \texttt{make_heap()}。

\subsection{构造方式}

要注明命名空间因为和 std 的类名称重复。

\vskip 0.2 in
\texttt{
__gnu_pbds ::priority_queue<int> __gnu_pbds::priority_queue<int, greater<int> >\\__gnu_pbds ::priority_queue<int, greater<int>, pairing_heap_tag>\\__gnu_pbds ::priority_queue<int>::point_iterator id; // 迭代器\\// 迭代器是一个内存地址,在modify和push的时候都会返回一个迭代器,下文会详细的讲使用方法\\id = q.push(1);}
\vskip 0.2 in

\subsection{成员函数}

\begin{enumerate}
\item \texttt{push()}: 向堆中压入一个元素, 返回该元素位置的迭代器
\item \texttt{pop()}: 将堆顶元素弹出
\item \texttt{top()}: 返回堆顶元素
\item \texttt{size()}返回元素个数
\item \texttt{empty()}返回是否非空
\item \texttt{modify(point_iterator, const key)} : 把迭代器位置的 key 修改为传入的 key,并对底层储存结构进行排序
\item \texttt{erase(point_iterator)} : 把迭代器位置的键值从堆中擦除
\item \texttt{join(__gnu_pbds :: priority_queue &other)}: 把 other 合并到  this 并把 other 清空。
\end{enumerate}

使用的 \texttt{tag} 决定了每个操作的时间复杂度:

\begin{tabular}{cclccc}
\hline
 & push& pop& modify& erase& Join\\Pairing_heap_tag& O(1)& 最坏& 最坏& 最坏& O(1)\\Binary_heap_tag& 最坏& 最坏& \Theta(n)& \Theta(n)& \Theta(n)\\Binomial_heap_tag& 最坏& \Theta(\log(n))& \Theta(\log(n))& \Theta(\log(n))& \Theta(\log(n))\\Rc_Binomial_heap_tag& O(1)& \Theta(\log(n))& \Theta(\log(n))& \Theta(\log(n))& \Theta(\log(n))\\Thin_heap_tag& O(1)& 最坏& 最坏& 最坏& \Theta(n)\\\hline
\end{tabular}

\subsection{示例}

\begin{cppcode}
#include <algorithm>
#include <cstdio>
#include <ext/pb_ds/priority_queue.hpp>
#include <iostream>
using namespace __gnu_pbds;
// 由于面向OIer, 本文以常用堆 : pairing_heap_tag作为范例
// 为了更好的阅读体验,定义宏如下 :
#define pair_heap __gnu_pbds ::priority_queue<int>
pair_heap q1;  //大根堆, 配对堆
pair_heap q2;
pair_heap ::point_iterator id;  // 一个迭代器
int main() {
  id = q1.push(1);
  // 堆中元素 : [1];
  for (int i = 2; i <= 5; i++) q1.push(i);
  // 堆中元素 :  [1, 2, 3, 4, 5];
  std ::cout << q1.top() << std ::endl;
  // 输出结果 : 5;
  q1.pop();
  // 堆中元素 : [1, 2, 3, 4];
  id = q1.push(10);
  // 堆中元素 : [1, 2, 3, 4, 10];
  q1.modify(id, 1);
  // 堆中元素 :  [1, 1, 2, 3, 4];
  std ::cout << q1.top() << std ::endl;
  // 输出结果 : 4;
  q1.pop();
  // 堆中元素 : [1, 1, 2, 3];
  id = q1.push(7);
  // 堆中元素 : [1, 1, 2, 3, 7];
  q1.erase(id);
  // 堆中元素 : [1, 1, 2, 3];
  q2.push(1), q2.push(3), q2.push(5);
  // q1中元素 : [1, 1, 2, 3], q2中元素 : [1, 3, 5];
  q2.join(q1);
  // q1中无元素,q2中元素 :[1, 1, 1, 2, 3, 3, 5];
}
\end{cppcode}
