
\subsection{KMP 自动机}

为了介绍 AC 自动机这种神奇的算法,先介绍自动机和 KMP 自动机

有限状态自动机 (DFA):字符集,有限状态控制,初始状态,接受状态

KMP 自动机:一个不断读入待匹配串,每次匹配时走到接受状态的 DFA

共有 $m$ 个状态,第 $i$ 个状态表示已经匹配了前 $i$ 个字符

$$
trans[i][x] =
\begin{cases}
i + 1,  & \text{if $b[i] = x$} \\[2ex]
trans[next[i]][x], & \text{else}
\end{cases}
$$

(约定 $next[0]=0$)

我们发现 $trans[i]$ 只依赖于之前的值,所以可以跟 \href{/string/kmp}{KMP} 一起求出来

时间和空间复杂度:$O(m|∑|)$

一些细节:走到接受状态之后立即转移到该状态的 $next$

\subsection{AC 算法就是 Trie 上的自动机}

注意在 \href{/search/bfs}{BFS} 的同时求出 $trans$ 即可

可以解决多串匹配问题

注意细节:$O(n+匹配次数)$ 还是 $O(n)$

前者能找到每次匹配,后者只能求出匹配次数(通过合并接受状态)

\subsection{AC 自动机的实现}
