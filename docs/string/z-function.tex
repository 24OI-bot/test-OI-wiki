
假设我们有一个长度为 $n$ 的字符串 $s$。该字符串的\textbf{Z 函数}为一个长度为 $n$ 的数组,其中第 $i$ 个元素为满足从位置 $i$ 开始且为 $s$ 前缀的字符串的最大长度。

换句话说,$z[i]$ 是 $s$ 和从 $i$ 开始的 $s$ 的后缀的最大公共前缀长度。

\textbf{注意}:为了避免歧义,在这篇文章中下标从 $0$ 开始,即 $s$ 的第一个字符下标为 $0$,最后一个字符下标为 $n - 1$。

Z 函数的第一个元素,$z[0]$,通常不是良定义的。在这篇文章中我们假定它是 $0$(虽然在算法实现中这没有任何影响)。

国外一般将计算该数组的算法称为\textbf{Z Algorithm},而国内则称其为\textbf{扩展 KMP}。

这篇文章包含在 $O(n)$ 时间复杂度内计算 Z 函数的算法以及其各种应用。

\subsection{样例}

下面若干样例展示了对于不同字符串的 Z 函数:

\begin{itemize}
\item $Z(\mathtt{aaaaa}) = [0, 4, 3, 2, 1]$
\item $Z(\mathtt{aaabaab}) = [0, 2, 1, 0, 2, 1, 0]$
\item $Z(\mathtt{abacaba}) = [0, 0, 1, 0, 3, 0, 1]$
\end{itemize}

\subsection{朴素算法}

Z 函数的形式化定义可被表述为下列基础的 $O(n^2)$ 实现。

\begin{cppcode}
vector<int> z_function_trivial(string s) {
  int n = (int)s.length();
  vector<int> z(n);
  for (int i = 1; i < n; ++i)
    while (i + z[i] < n && s[z[i]] == s[i + z[i]]) ++z[i];
  return z;
}
\end{cppcode}

我们做的仅仅为循环每个位置 $i$,并通过下述做法更新每个 $z[i]$:从 $z[i] = 0$ 开始,只要我们没有失配(并且没有到达末尾)就将其加 $1$。

诚然,这并不是一个高效的实现。我们接下来将展示一个高效实现的构造过程。

\subsection{计算 Z 函数的高效算法}

为了得到一个高效算法,我们将以 $i = 1$ 到 $n - 1$ 的顺序计算 $z[i]$,但在计算一个新值的同时,我们将尝试尽最大努力使用之前已经计算好的值。

为了简便起见,定义\textbf{匹配段}为同 $s$ 一个前缀相同的那些子串。举例来说,所求 Z 函数的第 $i$ 个元素 $z[i]$ 为从位置 $i$ 开始的匹配段的长度(其终止位置位于 $i + z[i] - 1$)。

为了达成目标,我们将始终保持\textbf{$[l;r]$  为最靠右的匹配段}。也就是说,在所有已探测到的匹配段中,我们将保持结尾最靠右的那一个。另一方面,下标 $r$ 可被认为是字符串 $s$ 已被算法扫描的边界;任何超过该点的字符都是未知的。

假设当前下标为 $i$(即我们要计算的下一个 Z 函数值的下标),则有两种情况:

\begin{itemize}
\item $i > r$ -- 当前位置在我们已处理位置\textbf{之外}。
我们接下来使用\textbf{朴素算法}(即一个一个的比较字符)来计算 $z[i]$。注意如果最后 $z[i] > 0$,我们需要更新最靠右的匹配段的下标,因为新的 $r = i + z[i] - 1$ 一定比之前的 $r$ 优。
\item $i \le r$ -- 当前位置位于当前匹配段 $[l;r]$ 之内。
那么我们可以用已计算过的 Z 函数值来 “初始化” $z[i]$ 至某值(至少比 “从零开始” 要好),甚至可能是某些较大的值。
为了做到这一点,我们注意到子串 $s[l\dots r]$ 和 $s[0 \dots r - l]$ 匹配。这意味着作为 $z[i]$ 的一个初始近似,我们可以直接使用对应于段 $s[0 \dots r - l]$ 的已计算过的 Z 函数值,也即 $z[i - l]$。
然而,$z[i - l]$ 可能太大了:将其应用到位置 $i$ 结果可能超过下标 $r$。这种做法并不合法,原因在于我们对 $r$ 右侧的字符一无所知:他们可能并不满足要求。
此处给出一个相似场景的\textbf{例子}:
$$
s=\mathtt{aaaabaa}
$$
当我们尝试计算末尾位置($i = 6$)的值时,当前匹配的段为 $[5;6]$。位置 $6$ 会匹配位置 $6 - 5 = 1$,其 Z 函数值为 $z[1] = 3$。显然,我们不能将 $z[6]$ 初始化为 $3$,因为这完全不对。我们可以初始化的最大值为 $1$ -- 因为这是使我们不超过段 $[l;r]$ 的边界 $r$ 的最大可能取值。
因此,我们可以放心的将下列值作为 $z[i]$ 的一个初始近似:
$$
z_0[i] = \min(r - i + 1, z[i - l])
$$
当将 $z[i]$ 初始化为 $z_0[i]$ 后,我们尝试使用\textbf{朴素算法}增加 $z[i]$ 的值 -- 因为宏观来讲,对于边界 $r$ 之后的事情,我们无法得知段是否会继续匹配还是失配。
\end{itemize}

综上所述,整个算法被划分成两种情况,他们只在设置 $z[i]$ 的\textbf{初始值}时有所不同:在第一种情况下,其被认为为 $0$,在第二种情况下它由先前已计算过的值确定(使用前述公式)。之后,该算法的两个分支都被规约为实现\textbf{朴素算法}。当我们设置完初始值后,该算法即开始执行。

该算法看起来非常简单。尽管在每轮迭代都会运行朴素算法,但我们已经取得了巨大进步:获得了一个时间复杂度为线性的算法。之后我们会证明这一点。

\subsection{实现}

实现相对来说十分简明:

\begin{cppcode}
vector<int> z_function(string s) {
  int n = (int)s.length();
  vector<int> z(n);
  for (int i = 1, l = 0, r = 0; i < n; ++i) {
    if (i <= r) z[i] = min(r - i + 1, z[i - l]);
    while (i + z[i] < n && s[z[i]] == s[i + z[i]]) ++z[i];
    if (i + z[i] - 1 > r) l = i, r = i + z[i] - 1;
  }
  return z;
}
\end{cppcode}

\subsubsection{对该实现的注释}

整个解法被作为一个函数给出。该函数返回一个长度为 $n$ 的数组 -- $s$ 的 Z 函数。

数组 $z$ 被初始化为全 $0$。当前最右的匹配段被假定为 $[0;0]$(一个故意为之的不包含任何 $i$ 的小段)。

在循环内,对于 $i=1\dots n - 1$,我们首先确定 $z[i]$ 的初始值 -- 其要么保持为 $0$ 或者使用前述公式计算。

之后,朴素算法尝试尽可能多的增加 $z[i]$ 值。

最后,如果必要(即如果 $i + z[i] - 1 > r$),我们更新最右匹配段 $[l;r]$。

\subsection{算法的渐进行为}

我们将证明上述算法的运行时间关于字符串长度呈线性 -- 即其时间复杂度为 $O(n)$。

该证明十分简单。

我们只关心内层 \texttt{while} 循环,因为其余部分在一次循环中只是一堆常数次操作,其时间复杂度总和为 $O(n)$。

我们将证明 \texttt{while} 的\textbf{每次迭代}都将增加匹配段的右边界 $r$。

为了做到这一点,我们将考虑算法的所有分支:

\begin{itemize}
\item $i > r$
在这种情况下,要么 \texttt{while} 循环不进行任何迭代(如果 $s[0] \neq s[i]$),要么其将从位置 $i$ 开始进行若干次迭代,其中每次迭代将向右移动一个字符。每次迭代后,右边界 $r$ 必定被更新。
因此我们证明了,当 $i > r$ 时,\texttt{while} 循环的每轮迭代都会使新的 $r$ 增加 $1$。
\item $i \le r$
在这种情况下,我们将 $z[i]$ 初始化为由前述公式给出的某个具体 $z_0$。将 $z_0$ 和 $r - i + 1$ 比较,可能有三种情况:
\begin{itemize}
\item $z_0 < r - i + 1$
我们证明在这种情况下 \texttt{while} 循环不会进行任何迭代。
这是十分容易证明的,比如通过反证法:如果 \texttt{while} 循环进行了至少一次迭代,这意味着初始近似 $z[i] = z_0$ 是不准确的(小于匹配的实际长度)。但是由于 $s[l\dots r]$ 和 $s[0\dots r - l]$ 是一样的,这推出 $z[i - l]$ 的值是错误的(比其该有的值小)。
所以,因为 $z[i - l]$ 是正确的且其值小于 $r - i + 1$,故该值同所求的 $z[i]$ 是相同的。
\item $z_0 = r - i + 1$
在这种情况下,\texttt{while} 循环可能会进行若干次迭代。因为我们从 $s[r + 1]$ 开始比较,而其位置已经超过了区间 $[l;r]$,故每次迭代都会使 $r$ 增加。
\item $z_0 > r - i + 1$
根据 $z_0$ 的定义,这种情况是不可能的。
\end{itemize}
\end{itemize}

综上,我们已经证明了内层循环的每次迭代都会使 $r$ 向右移动。由于 $r$ 不可能超过 $n - 1$,这意味着内层循环至多进行 $n - 1$ 轮迭代。

因为该算法的剩余部分显然时间复杂度为 $O(n)$,所以我们已经证明了计算 Z 函数的整个算法时间复杂度为线性。

\subsection{应用}

我们现在来考虑在若干具体情况下 Z 函数的应用。

这些应用在很大程度上同\href{./prefix-function.md}{前缀函数}的应用类似。

\subsubsection{查找子串}

为了避免混淆,我们将 $t$ 称作\textbf{文本},将 $p$ 称作\textbf{模式}。所给出的问题是:寻找在文本 $t$ 中模式 $p$ 的所有出现(occurrence)。

为了解决该问题,我们构造一个新的字符串 $s = p + \diamond + t$,也即我们将 $p$ 和 $t$ 连接在一起,但是在中间放置了一个分割字符 $\diamond$(我们将如此选取  $\diamond$ 使得其必定不出现在 $p$ 和 $t$ 中)。

首先计算 $s$ 的 Z 函数。接下来,对于在区间 $[0; \operatorname{length}(t) - 1]$ 中的任意 $i$,我们考虑其对应的值 $k = z[i + \operatorname{length}(p) + 1]$。如果 $k$ 等于 $\operatorname{length}(p)$,那么我们知道有一个 $p$ 的出现位于 $t$ 的第 $i$ 个位置,否则没有 $p$ 的出现位于 $t$ 的第 $i$ 个位置。

其时间复杂度(同时也是其空间复杂度)为 $O(\operatorname{length}(t) + \operatorname{length}(p))$。

\subsubsection{一个字符串中本质不同子串的数目}

给定一个长度为 $n$ 的字符串 $s$,计算 $s$ 的本质不同子串的数目。

我们将迭代的解决该问题。也即:在知道了当前的本质不同子串的数目的情况下,在 $s$ 末尾添加一个字符后重新计算该数目。

令 $k$ 为当前 $s$ 的本质不同子串数量。我们添加一个新的字符 $c$ 至 $s$。显然,会有一些新的子串以新的字符 $c$ 结尾(换句话说,那些以该字符结尾且我们之前未曾遇到的子串)。

构造字符串 $t = s + c$ 并将其反转(以相反顺序书写其字符)。我们现在的任务是计算有多少 $t$ 的前缀未在 $t$ 的其余任何地方出现。让我们计算 $t$ 的 Z 函数并找到其最大值 $z_{\max}$。显然,$t$ 的长度为 $z_{\max}$ 的前缀出现在 $t$ 中间的某个位置。自然的,更短的前缀也出现了。

所以,我们已经找到了当将字符 $c$ 添加至 $s$ 后新出现的子串数目为 $\operatorname{length}(t) - z_{\max}$。

作为其结果,该解法对于一个长度为 $n$ 的字符串的时间复杂度为 $O(n^2)$。

值得注意的是,我们可以用同样的方法在 $O(n)$ 时间内,重新计算在头部添加一个字符,或者移除一个字符(从尾或者头)时的本质不同子串数目。

\subsubsection{字符串压缩}

给定一个长度为 $n$ 的字符串 $s$,找到其最短的 “压缩” 表示,即:寻找一个最短的字符串 $t$,使得 $s$ 可以被 $t$ 的一份或多份拷贝的拼接表示。

其中一种解法为:计算 $s$ 的 Z 函数,从小到大循环所有满足 $i$ 整除 $n$ 的 $i$。在找到第一个满足 $i + z[i] = n$ 的 $i$ 时终止。那么该字符串 $s$ 可被压缩为长度 $i$ 的字符串。

该事实的证明同应用\href{./prefix-function.md}{前缀函数}的解法证明一样。

\subsection{练习题目}

\begin{itemize}
\item \href{http://codeforces.com/problemset/problem/126/B}{Codeforces - Password \textbackslash{}[Difficulty: Easy\textbackslash{}]}
\item \href{http://uva.onlinejudge.org/index.php?option=onlinejudge&page=show_problem&problem=396}{UVA \# 455 "Periodic Strings" \textbackslash{}[Difficulty: Medium\textbackslash{}]}
\item \href{http://uva.onlinejudge.org/index.php?option=onlinejudge&page=show_problem&problem=1963}{UVA \# 11022 "String Factoring" \textbackslash{}[Difficulty: Medium\textbackslash{}]}
\item \href{http://uva.onlinejudge.org/index.php?option=com_onlinejudge&Itemid=8&category=24&page=show_problem&problem=2470}{UVa 11475 - Extend to Palindrome}
\item \href{https://icpcarchive.ecs.baylor.edu/index.php?option=com_onlinejudge&Itemid=8&category=588&page=show_problem&problem=4450}{LA 6439 - Pasti Pas!}
\item \href{https://www.codechef.com/problems/CHSTR}{Codechef - Chef and Strings}
\item \href{http://codeforces.com/problemset/problem/432/D}{Codeforces - Prefixes and Suffixes}
\end{itemize}

\hr

\textbf{本页面主要译自博文 \href{http://e-maxx.ru/algo/z_function}{Z-функция строки и её вычисление} 与其英文翻译版 \href{https://cp-algorithms.com/string/z-function.html}{Z-function and its calculation} 。其中俄文版版权协议为 Public Domain + Leave a Link;英文版版权协议为 CC-BY-SA 4.0。}
