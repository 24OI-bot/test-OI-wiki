
\subsection{字符串是啥?}

字符串可以看作是字符序列。

\subsection{字符集}

字符集是符号和文字组成的集合,在 OI 中,处理字符串时计算复杂度往往要考虑到字符集大小带来的常数影响。

举个栗子,如果一道题只包含'A' \textasciitilde{} 'Z' 意味着字符集大小是 26 。 如果再加上 '0' ~ '9' 字符集大小就变成了 36

计算复杂度时,字符集大小带来的常数往往要用 $\alpha$ 表示。

\subsection{如何存字符串}

可以开一个 \texttt{char} 数组 , 如 \texttt{char a[100]}

也可以用 \texttt{vector} 如  \texttt{vector<char> v}

同时 STL 中也提供了字符串容器 \texttt{std :: string} 
